%\enlargethispage*{2cm}

\paragraph{Research Team}
Ben Godde (Professor of Neuroscience), Klaus Sch�mann (Professor of Sociology), Ursula M. Staudinger (Professor of Psychology), Claudia Voelcker-Rehage (Lecturer, Human Performance), Heike Heidemeier (Postdoctoral fellow, Psychology)

\paragraph{Description of the project}
The aging and shrinking of the labor force in modern industrialized nations requires that companies and employees optimize their productivity across their life spans in order to both avoid jeopardizing economic success, and to enable individuals to successfully navigate through their extended life span. With increasing age there are certain gains and a greater number of losses that may interfere with the maintenance of individuals' productivity. Job mobility has become an increasingly important feature of the modern labor market. In addition, it may be that job mobility helps to compensate for some salient age-related declines, and thus can be considered a natural intervention to strengthening  productivity across life spans.




\paragraph{Research objectives}
Our main aim is to investigate the cumulative effects of voluntary job mobility on (productivity) and its interaction with individual developmental trajectories of adaptive competencies. We will focus on the large cohort of the so-called "baby boomers", who are currently entering into midlife and will constitute the future cohort of older workers.

Job mobility is defined as encompassing voluntary changes of the employer or company, as well job changes within a company. Extending traditional job mobility research, which has concentrated on single spells of mobility, we investigate (using sequence-analytical methods) the cumulative effects of job mobility. Adaptive competence comprises a cognitive as well as a personality component. In the first domain, our study deals with the capacity to adapt perceptual and motor functioning, as well as higher cognitive functions, to new or changing tasks or challenges. This takes into account the neurobiological mechanisms underlying such learning or adaptation processes.
The important dimension of adaptive competence in the domain of personality characteristics, on which we will focus, is the openness for new experiences, the readiness to take risks, and rigidity. As mediating and moderating variables we are interested in self-esteem, internal control beliefs, and accommodative as well as assimilative skills. Further, job-specific aspects like job commitment and job satisfaction play an important role.
We assume a strong relationship between both aspects of adaptive competencies on the one hand and job mobility on the other hand. We have two main hypotheses that may both be true. First, adaptive competences might strengthen job mobility, as adaptive workers are more able and ready to learn new tasks or to adapt to new working environments, and thus to take over new positions within their firm or in other firms. An adaptive personality, characterized by openness, may have a stronger motivation for changes and mobility.
Second, it may also be expected that voluntary job mobility, in turn, is able to increase adaptive competencies. Changing working tasks and experiences of job mobility "train" cognitive adaptivity, as well as adaptivity in the domain of personality, and thus avoid the usual age-related decline or even stimulate increases. On the other hand, routinized work and a lack of new challenges for several years might have negative impacts on adaptive competences; even they must be trained continuously.
Interested in cognitive as well as personality developmental processes, our multidisciplinary approach combining both is well suited not only to provide new evidence within the different domains of adaptivity, but also to investigate interactions between personality characteristics and underlying cognitive processes at the neural level. Moreover, we expect that findings in either domain allow predictions for the other.



\paragraph{Outlook} 
  In an important first empirical step, using secondary-analytical methods, we will recruit matching samples from the data of the German Socioeconomic Panel (SOEP).
These samples will differ with respect to their job mobility (high or low mobility) and - accounting for gender-specific careers - in terms of gender, resulting in a design with four different groups of participants. Groups have to be matched for confounding variables like job profile, social participation, health, qualification, life satisfaction, etc. Propensity Score Matching allows filtering persons with nearly identical characteristics differing only in respect to the specific factors of interest (job mobility and gender).
As the second empirical step, we will use these samples to investigate the two facets of adaptive competence development as antecedence and/or consequence of job mobility. 
In the domain of cognitive functions we will use neurophysiological and neuropsychological methods (functional MRI, behavioral tests) to examine adaptivity and learning abilities. Test parameters are, for example, the ability to adapt perception under fast switching conditions, the capacity to transfer learned skills to new situations, and the extent of cognitive flexibility and short-term cortical plasticity. 
In the domain of personality, questionnaires will be used to assess openness to new experience and rigidity, as well as internal control beliefs and accommodative/assimilative coping styles. Further, job-specific aspects like job commitment and job satisfaction are also investigated. 
Sample size is restricted by the SOEP data set itself but also by the applied methods. For the neurophysiological and -psychological tests a sample size of n=20 per group is intended. The total sample size to correlate personality characteristics with job mobility will be extended to all persons available in the dataset and willing to participate. First calculations suggest that we need to cumulate 4-5 birth cohorts (born between 1962 and 1966) in order to arrive at a final sample of 200 participants (50 in each group) that has complete information for 20 years; this is taking into account a 25% recruitment success rate. 
The current study suggests that it is useful to pool expertise from sociology, psychology and neuroscience in order to test whether voluntary job mobility may be considered a highly successful "natural" intervention when it comes to optimizing productivity across the life course. For this purpose, the study combines the unique longitudinal SOEP data set with behavioral and neuroscience methods, and utilizes the unique potential of secondary data analysis. The practical implications of the proposed study are obvious: given we find the alleged compensatory and facilitative effects of voluntary job mobility, this will send a strong signal to companies and employees alike to pursue the route of transforming traditionally stable work careers in Germany.


\paragraph{Grant} 
 This project is funded by Volkswagen Foundation.
 
