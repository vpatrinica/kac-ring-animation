
\index{Staudinger, M. Ursula}

\paragraph{Research Team}
Ursula M. Staudinger (Professor), Jessica D\"orner (Doctoral Fellow; graduated 05/2006; postdoctoral fellow until 08/2006).

Life insight, and in particular wisdom as its highest form, represent the prototype of growth in adulthood. One may think that such phenomena defy empirical study. But a reliable and valid measurement paradigm has been developed (Baltes \& Staudinger, 2000). Consistent with cultural-historical writings about wisdom, there is empirical evidence that wisdom is indeed a phenomenon that depends on the successful integration of mind and character (Staudinger, Lopez, \& Baltes, 1997). The analysis and evaluation of important life experiences play an important part in that integration process. Thus, it does not come as a surprise that it is not enough to grow older to become wiser. In the last two years, a differentiation between personal wisdom and self-insight on the one and general wisdom and life insight on the other hand has been introduced and a measurement paradigm has been developed. This distinction is based on the assumption that distinguishable psychological phenomena are concerned when you have to be wise about a problem of your own life or life in general.

\null
\textbf{Research Highlights 2006}

Using the two new measures of self-insight or personality maturity that had been developed before, we started to investigate their age trajectories during the last year. As expected, based on the literature on personality growth, we found that neither self-related wisdom nor self-concept maturity normatively show positive age differences. In terms of self-related wisdom, we even found negative age differences in particular with regard to the three meta criteria of self-related wisdom, that is, tolerance, embeddedness and tolerance of ambiguity. This finding is consistent with the interpretation that the last developmental task of life, that is, integrating our lives and finding peace in the face of human terminality, may be juxtaposed to the goal of pursuing self-related wisdom.

\paragraph{Collaborations}

\begin{itemize}
\item Max Planck Institute for Human Development, Berlin\\ Prof. Dr. Paul B. Baltes ($\dagger$)
\item Oregon State University\\ Prof. Karen Hooker, PhD
\item IUB, JCLL\\ Prof. Dr. Ute Kunzmann
\item IUB, JCLL\\ Prof. Dr. Sven V\"olpel
\item University of Florida, Gainsville, FLA\\ Prof. Manfred Diehl, PhD
\item Universit\"at Frankfurt\\ Prof. Dr. Tilmann Habermas
\item Universit\"at Wien\\ Prof. Dr. Judith Gl\"uck
\end{itemize}

\begin{bibunit}[apalike]
\nocite{*}
\putbib[profUrsulaStaudinger1]
\end{bibunit}

\paragraph{Grants}
\begin{itemize}
\item DFG STA 540/3 -1/2(PI: U.M. Staudinger): Is it possible to promote self-insight? (2001-2005)
\end{itemize}