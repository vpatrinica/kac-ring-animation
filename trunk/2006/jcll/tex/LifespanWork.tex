
\index{Ro\ss nagel, Christian}

\paragraph{Research Team}
Christian Ro\ss nagel (Professor, since 09/2006), Melanie Schulz (Doctoral Fellow, starting February 2007).

\enlargethispage{0.5cm}
 This research line aims at describing age-related changes in individual workplace learning. Our basic assumption is that organizational and personal factors will interact so that age will moderate the influence of organizational factors on the perceived learning context (personal-subjective factors), and thus lead to different learning success. Subjective factors are assumed to be more important to learning success than objective factors. Clarifying the regularities of the interplay between organizational and personal factors will broaden our understanding of adult development at the workplace, and help develop and extend age-adequate HRD practices. Learning success is generically defined as the maximization of desired learning outcomes and minimization of undesired learning outcomes.

 As a first empirical step, we will develop age-differentiated instruments to assess motivational and affective consequences of learning episodes, and the learning climate of organizations. These questionnaires will be drafted from literature reviews, similar instruments and calibrated in semi-structured interviews. Construct validity will be assessed in a web-based survey. The survey will address participants who have recently participated in Training and Development measures. In a second step, we will experimentally validate the assumptions of the theory to be developed by investigating the interplay of age and \textit{developmental interactions} at the workplace. The experiments will test for the impact of different adviser styles on advisees' learning success. The topic of advice will be a computer task that will be novel to participants, but related to their everyday work, and likely to be used both by younger and older participants.

\paragraph{Collaborations}
\begin{itemize}
\item University of Heidelberg, Psychology Dept \\ I/O Psychology Unit \\ Prof. Dr. Guido Hertel; Dr. Ralf Stegmaier
\end{itemize}

\enlargethispage{0.5cm}
\begin{bibunit}[apalike]
\nocite{*}
\putbib[profChristianRossnagel]
\end{bibunit}

\paragraph{Grants}

\begin{itemize}
\item BMBF (PI: JCLL). C. Ro\ss nagel, K. Sch\"omann: subproject ``Learning'' within the joint research project ``Effects of Matches/Mismatches between Aspects of Human and Social Capital, Corporate Strategy and Work Organization on the Physical and Mental Well-Being of Employees''. 
\item DFG (submitted, PI: C. Ro\ss nagel with U.M. Staudinger in the framework of the DFG focus program 1293). Age-differentiated analysis of the competency of self-regulated professional learning.
\end{itemize}
