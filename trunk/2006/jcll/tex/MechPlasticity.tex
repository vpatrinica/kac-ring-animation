

\index{Staudinger, M. Ursula}

\paragraph{Research Team}
Ursula M. Staudinger (Professor), Jessica D\"orner (Doctoral Fellow until 05/2006), Eva-Marie Kessler (Doctoral Fellow, graduated in January 2006; now Postdoctoral Fellow)

Human development is not determined but plastic. This is one the central contentions of lifespan psychology. Based on biological, psychological and external resources any observed development trajectory can be changed within limits. Such changes can be directed towards an improvement or a deterioration of the normally observed developmental course. It is argued that it is theoretically and empirically useful to distinguish these two kinds of plasticity. In conjunction with negative deviations from the normal developmental course the concept of resilience has been introduced into the literature. It refers to the maintenance or regaining of normal levels of functioning in the face of stressors. In contrast, positive deviations from the developmental trajectory should be subsumed under the notion of growth. When considering development as such, rather than positive or negative deviations from an expected developmental course, it again may be meaningful to differentiate between two kinds of positive developments. One is primarily geared towards the achievement of growth and the other towards adjustment. This latter distinction has been a focus of our work in 2005.

\null
\textbf{Research Highlights 2006}

Does personality stay stable after young adulthood or is there continued change throughout middle and later adulthood? For decades, this question caused heated debate. Over the last couple of years, a consensus has emerged based on recent cross-cultural as well as longitudinal evidence. This consensus confirms that indeed there is personality change in middle and later adulthood. Many authors have labeled this change personality maturation or growth. In somewhat simplified terms the observed pattern is as follows: Neuroticism declines, conscientiousness and agreeableness increase. At the same time it has been argued that this pattern of personality change is the result of coping with the developmental tasks of adulthood and thus increased adjustment. We critically examined this practice of equating developmental adjustment with growth and pointed to the necessity of defining personality growth by reference to theories of personality development as well as lifespan theory. 

\paragraph{Collaborations}

\begin{itemize}
\item Universit\"at Hildesheim\\ Prof. Dr. Werner Greve
\item University Trier\\ Prof. Dr. Sigrun-Heide Filipp
\item IUB, JCLL\\ Prof. Dr. Ute Kunzmann
\end{itemize}

\hyphenation{Ent-wick-lungs-psycho-lo-gie Hassel-horn}
\begin{bibunit}[apalike]
\nocite{*}
\putbib[profUrsulaStaudinger4]
\end{bibunit}