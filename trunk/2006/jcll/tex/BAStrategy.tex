\index{Voelpel, Sven}

\paragraph{Research Team}
Sven Voelpel (Professor), Polina Isichenko (Doctoral Fellow), Jan-Dirk Fr\"uchtenicht (Doctoral Fellow since 09/2006).

Strategy is important to - systematically and efficiently - lead organizations to success. Knowledge and wisdom, on the other hand, are the most important interrelated sources for innovation. Together with the key lever energy, they determine organizations' success. These ``resources'' need to be aligned within an integrated organizational strategy. New strategic management approaches and tools are quested to enable rapid, discontinuous organizational innovation and change capabilities with which to create and sustain wise organizations. Research into strategy entails projects on strategic (change) management, strategic organizational fitness, business model reinvention, managing a culturally diverse workforce.

 Research covering the area of ``Aging'' and ``Strategy'' introduces the new topic of the 'competitive workforce' to the body of scientific knowledge. Recently, many companies have become aware that the demographic shift, which is altering the average age of society and the available workforce alike globally, holds the potential to become a serious threat for the innovative competitiveness of organizations. So far, extant research and coherent managerial guidelines in that field are scarce and preliminary. Thus, it is of high significance to fill this gap to provide sound measures.
 
 The research in the area of strategy and aging focuses especially on (1) managing an aging workforce towards competitiveness, (2) identifying future market and/or customer opportunities resulting from the demographic shift and the aging of the workforce, (3) best practices of knowledge management in China, (4) retention and rejuvenation of organizational knowledge.

 The rise of Asia's economy during the last decades has stunned the world. Many international firms have already planned Asia on the top of their long-term development agenda. What are their objectives in Asia? How have they developed policies and implemented these objectives? These are some of the questions Asia strategists seek answers to. Strategic management is dynamic and requires on-going, continuous reassessment and reformation. The fast moving Asian context further accelerates the dynamism of industrial and market environment an organization has to cope with. Research on strategy and Asia focuses on (1) entry strategies, (2) growth strategies, (3) integration of local and global strategy of a firm, (4) dynamic capability.

\null
\textbf{Research Highlights 2006}

 Organizations increasingly have to face major challenges arising from the demographic shift of the workforce. One of the main questions is, how leaders will be able to manage work and people to achieve increasing performance under changed demographic conditions in the 21st century. So far, governments and management alike either do not realize the potential problems, or they ignore them, for there are no easy solutions at hand. Entirely new skill sets of how to manage and lead organizations are required, and it is not appropriate to postpone adequate measures. Despite the fact that older employees have unique wisdom and experience that they have accumulated during their working life, so far leaders in management are lacking the necessary know-how to leverage on these qualities. 

 With regard to this topic, Sven Voelpel and Chris Streb have been serving as track conveners of the research workshops at the 66th Annual Academy of Management Meeting, August 2006, Atlanta, USA on ``Managing the Aging Workforce - Leadership towards a new Weltanschauung'' where they were able to gather the leading researchers in this field from Asia, Europe and the USA. 

\newpage
\paragraph{Collaborations}
\begin{itemize}
\item Harvard University, USA \\ Prof. Emeritus Michael Beer, PhD
\item University of St. Gallen, Switzerland \\ Prof. Dr. Georg von Krogh 
\item University of Stellenbosch, South Africa \\ Prof. Dr. Marius Leibold 
\item Cranfield University School of Management \\ Visiting Prof. John-Christopher Spender, PhD
\end{itemize}

\begin{bibunit}[apalike]
\nocite{*}
\putbib[profSvenVoelpel3]
\end{bibunit}

\paragraph{Grants}
\begin{itemize}
\item DAAD (Funding for PhD Thesis of Polina Isichenko).
\item Lufthansa (Funding for PhD Thesis of Jan-Dirk Fr\"uchtenicht).
\item VHB Travel Grant for Polina Isichenko for Academy of Management Conference 2006.
\end{itemize}