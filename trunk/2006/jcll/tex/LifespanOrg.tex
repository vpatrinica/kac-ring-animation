
\index{Ro\ss nagel, Christian}

\paragraph{Research Team}
Christian Ro\ss nagel (Professor, since 09/2006), Philipp Martzog (Doctoral Student).

 This line of research complements the research outlined in Section 9.1 that uses the concept of \textit{successful} workplace learning. Learning success will be defined not only with regards to its objective learning outcomes, but also with respect to the psychic costs it incurs. Initial informal surveys in a co-operating clinic (see 9.3.2) showed that the need for lifelong learning may turn to ``learning pressure'' and lead to severe health problems.

 We are planning to use the instruments developed in the workplace learning research (see Section 9.1) together with semi-structured interviews with patients to uncover the joint contribution of age differences and organizational factors to vocational adjustment disorders. These findings will broaden our understanding of successful workplace learning and will contribute to development of HR development and intervention strategies.

\paragraph{Collaborations}
\begin{itemize}
\item Baar Clinic for Behavioral Medicine, Donaueschingen \\ Bernd Haves (Medical Director)
\end{itemize}
