

\index{Voelpel, Sven}

\paragraph{Research Team}
Sven Voelpel (Professor), Zheng Han (Postdoctoral Fellow until 09/2006), Chris Streb (Doctoral Fellow), Jan Meyer (Doctoral Fellow), Chunli Zhao (Doctoral Fellow).

 Wisdom creation research has been developed from well-known streams of management research, such as information and knowledge management. Knowledge accordingly is considered as companies' basic ``resource''. Since organizations are not only driven by intellects, but also by humans, successful management research and practice also need to include less concrete constructs, such as emotional qualities. One aspect of these wisdom-related qualities would be the ability to abstract from experience knowledge and apply it to new situations.

 Research in psychology shows that wisdom is a phenomenon that depends on the successful integration of mind and character (Baltes \& Staudinger, 2000). This can be applied beyond the individual to the group and organizational level in order to create and sustain a wise organization.

 Research in the area of wisdom involves projects on: (1) knowledge management and mobilization, (2) lifelong learning, (3) knowledge acquisition for leaving/retiring employees and (4) the ``wisdom creating company''.

 With mass retirement of mature and skilled aged professionals, organizations forfeit their chances to profit from and utilize a huge pool of valuable experience, which has been accumulated by seniors during their long working career. It is becoming clear that companies risk losing wisdom and with it competitive advantage, if they let their mature workforce depart. In the light of growing organizational interest in how best to manage mature employees with rich accumulated wisdom, the paucity of research in this area represents a serious gap in academic knowledge.

 Research in the area of wisdom and aging focuses especially on: (1) knowledge management: Driving knowledge sharing and creation by supporting the natural (often social) knowledge processes with organizational and IT tools; (2) managing a multi-generational workforce: integration, participation and collaboration; (3) lifelong learning with an aging workforce; (4) knowledge acquisition for retiring employees. 

 It is a long journey starting from pure Information to Knowledge to finally reach Wisdom. The management of knowledge makes up the principal portion of this journey. Within the Asian context, Knowledge Management (KM), a well-established management research concept in the West, has been introduced not long ago. The context, however, is a totally different one due to significant discrepancies between Asian and Western cultures. Culture shapes assumptions about what knowledge is, defines the relationship between individual and organizational knowledge, creates the context for social interaction in which knowledge will be used, and shapes the processes by which new knowledge is created, leveraged and distributed in organizations. Consequently, it is of high significance to build up effective KM strategies within the Asian context.

 Research in the area of Wisdom and Asian focuses on: (1) knowledge creation in the Asian context, (2) cultural influences on the behavior of knowledge sharing, (3) best practices of knowledge management in China, (4) applicability of Western knowledge management tools within the Asian context.

\null
\textbf{Research Highlights 2006}

 In order to be able to fuse the latest streams of research in the area of knowledge management and wisdom, the research group made an effort to review the latest findings, which resulted in a number of publications (see below). Particularly noteworthy is the special issue ``Becoming critical on Intellectual Capital'' in the Journal of Intellectual Capital co-edited by David O'Donnell, Lars Bo Henriksen and Sven Voelpel. This special issue achieved to review contributions from the leading authors in the field of intellectual capital. Another extraordinary achievement is the review article on the progress in the major management field knowledge management during the last 15 years in collaboration with two of the leading researchers of the field (Nonaka, I., von Krogh, G., and Voelpel, S., 2006).

 Following efforts from last year, the group contributed to the field of Intellectual capital (IC), which is concerned with the valuation and management of intangibles such as knowledge, human capital and brands. Drawing on Habermas' theory of communicative action and an illustrative series of interviews at Meyer Werft, O'Donnell et al. (2006) describe how large parts of organizational knowledge are tacit, thus tied to individuals, yet collectively created and maintained by communicative action. This tacit background knowledge is an important driver for individual action (e.g. when professional pride motivates innovative behavior) and thus opens up promising venues for indirect (``second-order'') management of these motivators.  As central argument the authors show limits in accessing the background knowledge stemming from its truly tacit nature and thus also imposing limits on valuing the full spectrum of organizational knowledge for the purpose of cost accounting and financial reporting. Jan Meyer presented the paper at the 2nd Workshop on Visualizing, Measuring and Managing Intangibles and Intellectual Capital, Maastricht (NL).

 As preparation for further data acquisition, a survey was worked out, which links different activities (routine tasks, innovation related tasks, problem solving efforts etc.) to knowledge types and knowledge tools use from the perspective of individual employees. Here ``knowledge tools'' include not only IT-tools but also human centered tools such as meetings and master/apprentice co-working arrangements. The aim is to understand tool usage patterns as basis for optimizing the use of knowledge tools within organizations. The pilot study for next year's survey is currently conducted at Meyer Werft.

\newpage
\paragraph{Collaborations}
\begin{itemize}
\item Intellectual Capital Institute of Ireland  \\ David O'Donnell 
\item Harvard Business School, USA \\  Dorothy Leonard
\item Hitotsubashi University, Japan \\ Prof. Ikujiro Nonaka
\item Babson College, USA \\  Prof. Thomas H. Davenport, PhD
\item University of St. Gallen, Switzerland \\  Prof. Dr. Georg von Krogh
\item University of Stellenbosch, South Africa \\  Prof. Dr. Marius Leibold
\item IUB, JCLL \\  Prof. Dr. Christian Ro\ss nagel
\end{itemize}

\begin{bibunit}[apalike]
\nocite{*}
\putbib[profSvenVoelpel1]
\end{bibunit}


\paragraph{Grants}

\begin{itemize}
\item BMBF (PI: JCLL). S. Voelpel, C. Schwender: subproject ``Communication and Experience Management'' within the joint research project ``Effects of Matches/Mismatches between Aspects of Human and Social Capital, Corporate Strategy and Work Organization on the Physical and Mental Well-Being of Employees''. 
\item DAAD ``STIBET-Doktorandenf\"orderung'' PhD Seminar (Sven Voelpel, Zheng Han). 
\item Meyer Werft (Funding for PhD Thesis of Jan Meyer).
\item Stiftung der Deutschen Wirtschaft (sdw) Scholarship (Chunli Zhao). 
\item DFG Travel Grant for Sven Voelpel for Academy of Management Conference 2006, European Academy of Management Conference 2006.
\item DFG Travel Grant for Zheng Han, IAMOT Conference 2006.
\item Advanced Institute of Management Travel Grant for Sven Voelpel and Zheng Han for Workshop on the Practice of Dynamic Capabilities.
\item Singapore Management University Travel Grant for Sven Voelpel and Zhang Han for the 3rd International Research Conference on Chinese Entrepreneurship and Asian Business Networks.
\end{itemize}



