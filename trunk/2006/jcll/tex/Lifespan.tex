\section{A Lifespan Perspective on Organizational Behavior}
\shorttitle{A Lifespan Perspective on Organizational Behavior}

Organizational behavior research addresses the interplay of organizational context - expressed in an organization's culture, climates and goals - and organization members' KSAOs - the I/O Psychology shorthand for knowledge, skills, abilities and other characteristics (e.g., motives, personality, values). Until recently, age-related KSAO changes have not been much of a concern. The business place has been - and still is - youth-centered, and research projects have reflected this. In the wake of demographic change, however, a growing number of older (50+ yrs) workers will live longer work lives. At the same time, fewer younger workers will be available. As globalization and technological change pace permanent adaptation, continuous learning will become an integral part of organizations' operating systems. 

 On this background, an important overarching question will be how the developments of an organization and of its members may best be aligned for the benefit of both sides. More specifically, a fruitful research agendum will be to apply concepts and findings from lifespan psychology to develop an age-differentiated theory that describes how organizational and individual learning needs and abilities can best be matched. With a view on individuals, this will require to uncover the system of age-related changes in workplace behaviors. With a view on companies, it will be necessary to develop competency management approaches that take the dynamics of individual development into account.

\subsection{Successful Workplace Learning across the Lifespan}


\index{Ro\ss nagel, Christian}

\paragraph{Research Team}
Christian Ro\ss nagel (Professor, since 09/2006), Melanie Schulz (Doctoral Fellow, starting February 2007).

\enlargethispage{0.5cm}
 This research line aims at describing age-related changes in individual workplace learning. Our basic assumption is that organizational and personal factors will interact so that age will moderate the influence of organizational factors on the perceived learning context (personal-subjective factors), and thus lead to different learning success. Subjective factors are assumed to be more important to learning success than objective factors. Clarifying the regularities of the interplay between organizational and personal factors will broaden our understanding of adult development at the workplace, and help develop and extend age-adequate HRD practices. Learning success is generically defined as the maximization of desired learning outcomes and minimization of undesired learning outcomes.

 As a first empirical step, we will develop age-differentiated instruments to assess motivational and affective consequences of learning episodes, and the learning climate of organizations. These questionnaires will be drafted from literature reviews, similar instruments and calibrated in semi-structured interviews. Construct validity will be assessed in a web-based survey. The survey will address participants who have recently participated in Training and Development measures. In a second step, we will experimentally validate the assumptions of the theory to be developed by investigating the interplay of age and \textit{developmental interactions} at the workplace. The experiments will test for the impact of different adviser styles on advisees' learning success. The topic of advice will be a computer task that will be novel to participants, but related to their everyday work, and likely to be used both by younger and older participants.

\paragraph{Collaborations}
\begin{itemize}
\item University of Heidelberg, Psychology Dept \\ I/O Psychology Unit \\ Prof. Dr. Guido Hertel; Dr. Ralf Stegmaier
\end{itemize}

\enlargethispage{0.5cm}
\begin{bibunit}[apalike]
\nocite{*}
\putbib[profChristianRossnagel]
\end{bibunit}

\paragraph{Grants}

\begin{itemize}
\item BMBF (PI: JCLL). C. Ro\ss nagel, K. Sch\"omann: subproject ``Learning'' within the joint research project ``Effects of Matches/Mismatches between Aspects of Human and Social Capital, Corporate Strategy and Work Organization on the Physical and Mental Well-Being of Employees''. 
\item DFG (submitted, PI: C. Ro\ss nagel with U.M. Staudinger in the framework of the DFG focus program 1293). Age-differentiated analysis of the competency of self-regulated professional learning.
\end{itemize}


\subsection{Dynamic Modeling of Lifelong Learning Competency}

\enlargethispage*{1cm}

\index{Ro\ss nagel, Christian}

\paragraph{Research Team}
Christian Ro\ss nagel (Professor, since 09/2006), Michael Kohler (Diploma Student).

 Competency models are widely used by Human Resource (HR) practitioners in the selection, placement and development of employees. However, only little competency modeling research has been conducted, and the issue of age-related competency changes has hardly ever been addressed. In collaboration with Bosch Rexroth AG, we are developing a competency model that incorporates employees' age-related changes in ability and motivation factors. The model is intended to help identify HR development requirements in the mid-term (5 years).

 Using a goodness-of-fit framework, we will match supervisors' ratings of their employees on competencies described in existing company-specific competency models against employees' ratings of their KSAOs and their perception of how these will change over the next five years. Fit indices will be computed for ability and non-ability-related dimensions to identify mismatches as a starting point for HR development measures.

\textbf{Collaborations}
\begin{itemize}
\item Bosch Rexroth AG, Homburg Plant \\ Personnel Department \\ Claudia Neunzig
\end{itemize}


\subsection{Organizational Determinants of Vocational Adjustment Disorders}


\index{Ro\ss nagel, Christian}

\paragraph{Research Team}
Christian Ro\ss nagel (Professor, since 09/2006), Philipp Martzog (Doctoral Student).

 This line of research complements the research outlined in Section 9.1 that uses the concept of \textit{successful} workplace learning. Learning success will be defined not only with regards to its objective learning outcomes, but also with respect to the psychic costs it incurs. Initial informal surveys in a co-operating clinic (see 9.3.2) showed that the need for lifelong learning may turn to ``learning pressure'' and lead to severe health problems.

 We are planning to use the instruments developed in the workplace learning research (see Section 9.1) together with semi-structured interviews with patients to uncover the joint contribution of age differences and organizational factors to vocational adjustment disorders. These findings will broaden our understanding of successful workplace learning and will contribute to development of HR development and intervention strategies.

\paragraph{Collaborations}
\begin{itemize}
\item Baar Clinic for Behavioral Medicine, Donaueschingen \\ Bernd Haves (Medical Director)
\end{itemize}


\subsection{Other Professional Activities}

\textit{Editorial Board Membership}

\begin{itemize}
\item Journal of Behavioural Sciences.
\end{itemize}

\textit{Ad-hoc Reviews}

\begin{itemize}
\item Journal of Behavioural Sciences.
\item Swiss National Fund.
\item Zeitschrift f\"ur Personalpsychologie.
\item Zeitschrift f\"ur Sozialpsychologie.
\end{itemize}




