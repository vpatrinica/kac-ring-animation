%Publications for Statistics and Methodoligy
\newpage


\subsection{Publications}




	\paragraph{Books} \textit{ }
\bigskip
	
Rippl, S., Baier, D., \& Boehnke, K. (2006). \textit{Auf dem Weg nach rechts? EU-Osterweiterung als Kontext f\"{u}r die Mobilisierung rechter Einstellungen - eine vergleichende Studie in Deutschland, Polen und der Tschechischen Republik} (unter Mitarbeit von A. Kindervater \& A. Hadjar). Wiesbaden: VS Verlag, in press.\\




\paragraph{Articles \& Chapters}\textit{ }

\bigskip

Baier, D., \& Boehnke, K. (2006). Marek Fuchs, Siegfried Lamnek, Jens Luedtke und Nina Baur: Gewalt an Schulen. 1994 - 1999 - 2004. Wiesbaden: VS Verlag, 2005. \textit{K\"{o}lner Zeitschrift f\"{u}r Soziologie und Sozialpsychologie, 58}, 367-369.\\

Baier, D., \& Boehnke, K. (2006). Das Risiko der Jugendforschung. Sammelbesprechung. \textit{Soziologische Revue, 29}, 154-165.\\

Boehnke, K., \& Welzel, C. (2006). Wertetransmission und Wertewandel - Eine explorative Drei-Generationen-Studie. \textit{Zeitschrift f\"{u}r Soziologie der Erziehung und Sozialisation, 26 }(4), 341-360.\\

Boehnke, K., \& Boehnke, M. (2006). Atomare Katastrophenangst als Motor f\"{u}r politisches Engagement? Analysen einer 20-j\"{a}hrigen L\"{a}ngsschnittstudie. \textit{Umweltpsychologie, 10 }(1), 62-85.\\

Fuss, D. (2006). Exklusiv vs. inklusiv? Einstellungen gegen\"{u}ber Fremden im Kontext nationaler und europ\"{a}ischer Identit\"{a}t. \textit{Zeitschrift f\"{u}r Soziologie der Erziehung und Sozialisation, 26 }(1), 69-85. \\

Fuss, D., \& Grosser, M.A. (2006). What makes young Europeans feel European? Results from a cross-cultural research project. In I.P. Karolewski, \& V. Kaina (Eds.), \textit{European Identity: Theoretical Perspectives and Empirical Insights} (pp. 209-241). Hamburg: LIT-Verlag.\\

Idemudia, E.H., \& Boehnke, K. (2006). An assessment of African migrants' attitudes in Germany. \textit{Psychopathologie africaine, 23}, 5-20.\\

Keeves, J.P., Lietz, P., Gregory, K., \& Darmawan, I.G.N. (2006). Some Problems in the analysis of cross-national survey data. \textit{International Education Journal, 7} (2), 110-126.\\

Kornyeyeva, O. (2006).  First-form boy on the extended-day class (in Russian). \textit{Family World}, Issue 1, 68-70.\\

Kornyeyeva, O. (2006). "Help! I am getting crazy being a young mom" Case study: Post-natal depression and changing woman-mother identity. How to overcome the difficulties (in Russian). \textit{Family World}, Issue 6, 20-22.\\

Kornyeyeva, O. (2006). How to teach a child to take care of his/her personal things? (in Russian). \textit{Aljenka and Serezhka Journal for Parents}, Issue 1, 12-13.\\

Kornyeyeva, O. (2006). How to teach a child to keep his room in order? (in Russian). \textit{Aljenka and Serezhka Journal for Parents}, Issue 2, 10-11.\\

Kornyeyeva, O. (2006). How to teach child manners during a meal? (in Russian). \textit{Aljenka and Serezhka Journal for Parents}, Issue 3, 10-11.\\

Kornyeyeva, O. (2006). How to teach a child to behave in community? (in Russian). \textit{Aljenka and Serezhka Journal for Parents}, Issue 5, 10-11.\\

Kornyeyeva, O. (2006). How to teach a child to behave in adult society and to communicate with adults? (in Russian). \textit{Aljenka and Serezhka Journal for Parents}, Issue 5, 16-17.\\

Kornyeyeva, O. (2006). "Sorry, your daddy is temporarily unavailable..." (in Russian). \textit{Aljenka and Serezhka Journal for Parents}, Issue 6, 56-59.\\

Leonidou L.C., Barnes B., \& Talias, M. (2006). Exporter-importer relationship quality: The inhibiting role of uncertainty, distance and conflict. \textit{Industrial Marketing Management, 35 }(5), 576-588.\\

Lietz, P. (2006). Quantitative Auswertung: Multivariate Statistik. In N. Groeben, \& B. Hurrelmann (Eds.) \textit{Empirische Unterrichtsforschung in der Literatur- und Lesedidaktik }(pp. 481-512). Weinheim: Juventa.\\

Lietz, P. (2006) Issues in the change in gender differences in reading achievement in cross-national research studies since 1992: A meta-analytic view. \textit{International Education Journal, 7} (2), 127-149.\\

Lietz, P. (2006).). A meta-analysis of gender differences in reading achievement at the secondary school level. \textit{Studies in Educational Evaluation, 32}, 317-344.\\

Lietz, P., \& Kotte, D. (2006). Quantitative Auswertung: Uni- undbivariate Statistik. In N. Groeben, \& B. Hurrelmann (Eds.) \textit{Empirische Unterrichtsforschung in der Literatur- und Lesedidaktik }(pp. 443-480). Weinheim: Juventa.\\

Lietz, P., \& Kotte, D. (2006). Science and education: Analyses of science-related variables of the PISA-2000 survey. In B. Dooley (Ed.), \textit{Energy and culture }(pp. 113-128). New York: Springer.\\

Oda\v{g}, \"{O}. (2006). Sozialpsychologie des Unterrichts. In N. Groeben, \& B. Hurrelmann (Eds.), \textit{Empirische Unterrichtsforschung in der Literatur- und Lesedidaktik }(pp. 255-272). Weinheim: Juventa.\\

Rippl, S. \& Boehnke, K. (2006). Europas Jugend: Protagonisten f\"{u}r Integration oder Nationalismus? \textit{Aus Politik und Zeitgeschehen, 47}, 37-46.\\

Schmitz-Justen, F.J., \& Wilhelm, A.F.X. (2006). An empirical study of factors impacting on knowledge processes in online forums: Factors of interest and model outline. \textit{International Journal of Web Based Communities (IJWBC), 2} (3), in press.\\

Schmitz-Justen, F.J., \& Wilhelm, A.F.X. (2006). An empirical study of factors impacting on knowledge processes in online forums: Structural equationmodeling analysis and results. \textit{International Journal of Web Based Communities (IJWBC)}, in press.\\

Schreier, M. (2006). (Subjective) Well-Being. In J. Bryant, \& P. Vorderer (Eds.), \textit{Psychology of entertainment }(pp. 389-404). Mahwah, NJ: Erlbaum.\\
\newpage
Schreier, M. (2006). Qualitatives Untersuchungsdesign. In N. Groeben, \& B. Hurrelmann (Eds.), \textit{Empirische Unterrichtsforschung in der Literatur- und Lesedidaktik} (pp. 343-360). Weinheim: Juventa.\\

Schreier, M. (2006). Experimentelle/quasi-experimentelle Untersuchungsplanung. In N. Groeben, \& B. Hurrelmann (Eds.), \textit{Empirische Unterrichtsforschung in der Literatur- und Lesedidaktik }(pp. 307-342). Weinheim: Juventa.\\

Schreier, M. (2006). Qualitative Verfahren der Datenerhebung. In N. Groeben, \& B. Hurrelmann (Eds.), \textit{Empirische Unterrichtsforschung in der Literatur- und Lesedidaktik }(pp. 399-420). Weinheim: Juventa.\\

Schreier, M. (2006). Qualitative Auswertungsverfahren. In N. Groeben, \& B. Hurrelmann (Eds.), \textit{Empirische Unterrichtsforschung in der Literatur- und Lesedidaktik }(pp. 421-442). Weinheim: Juventa.\\

Schreier, M.(2006). Qualitative Methoden. In J. Funke, \& P. A. Frensch (Eds.), \textit{Handbuch der Allgemeinen Psychologie - Kognition }(pp. 769-774). G\"{o}ttingen etc.: Hogrefe.\\

Schreier, M., \& Lietz, P. (2006). Quantitative Datenerhebungsverfahren. In N. Groeben, \& B. Hurrelmann (Eds.), \textit{Empirische Unterrichtsforschung in der Literatur- und Lesedidaktik }(pp. 361-397). Weinheim: Juventa.\\



Schreier, M. (2006). The "nation" in crime: does the reader care? In I. Amodeo, \& E. Erdmann (Eds.), \textit{Crime and nation}. Trier: WVT, in press.\\

Schreier, M. (2006). Belief change through fiction: how fictional narratives affect real readers. In G. Lauer, F. Jannidis, \& S. Winko (Eds.), \textit{Grenzen der Literatur}. Berlin: de Gruyter, in press.\\

Schreier, M. (2006). Textwirkungen. In T. Anz (Ed.), \textit{Handbuch Literaturwissenschaft}. Stuttgart: Metzler, in press.\\

Schreier, M. \& Oda\v{g}, \"{O}. (2006). Experiential states during reading: a gendered view. In C. Aaftink (Ed.), \textit{How literature enters life (SPIEL special issue)}, in press.\\

Stott, C., Adang, O., Livingstone, A., \& Schreiber, M. (2006). Variability in the collective behaviour of England fans at Euro2004: "Hooliganism", public order, policing, and social change. \textit{European Journal of Social Psychology }(accepted and pre-published paper on the Internet under http://www3.interscience.wiley.com/cgi-bin/abstract/112636731/ABSTRACT? CRETRY= 1\&SRETRY=0).\\

Thies, Y., \& Schreier, M. (2006). Kultivation durch Unter-haltungsangebote: Die stereotype Welt des Vielsehers von St. Angela und Co? In H. Schramm, W. Wirth, \& H. Bilandzic (Eds.), \textit{Empirische Unterhaltungsforschung: Studien zu Rezeption und Wirkung von medialer Unterhaltung }(pp. 191-214). M\"{u}nchen: Verlag Reinhard Fischer.\\
