%the function for the book publication could be as follows [it was not implemented here]
%1=author;2=year;3=title;4=publisher + comments
%\newcommand{\book}[4]{#1 (#2). \textit{#3}. #4}

\subsection{Publications}




	\paragraph{Books} \textit{ }
	
\bigskip	

Amodeo, I. (2006). \textit{"Il gusto melodrammatico". Eine medienkomparatistische Studie zum Opernhaften.} Bielefeld: Transcript, in press.\\ \\


Amodeo, I., \& Erdmann, E. (Eds.) (2006). \textit{Crime \& nation. Political and cultural mappings of criminality in traditional and in new media.} Trier: WVT, in press. \\ \\

Botar, O.A.I., \& W\"{u}nsche, I. (Eds.) (2006). \textit{Biocentrism and modernism.} Pittsburgh: Pittsburgh University Press, in press.\\ \\


Crowther, P. (2006). \textit{Defining art, creating the canon; artistic value in an era of doubt.} Oxford: Oxford University Press, in press.\\ \\

Elliot, R.K. (2006). \textit{Aesthetics, imagination, and the unity of experience.} (edited and critically introduced by Paul Crowther, pp. ix - xix). Ashgate: Aldershot.\\ \\

Peters, S., Stierstorfer, K., \& Volkmann, L. (Ed.) (2006). \textit{Drama I, Teaching Contemporary Literature and Culture.} Trier: WVT.\\ \\

Peters, S., Stierstorfer, K., \& Volkmann, L. (Ed.) (2006). \textit{Drama II, Teaching Contemporary Literature and Culture.} Trier: WVT.\\ \\

Rommel, T. (2006). \textit{50 Klassiker der Weltliteratur. B\"{u}cher lesen und verstehen.} Hamburg: merus.\\ \\

Rommel, T. (2006). \textit{Das Selbstinteresse von Mandeville bis Smith. \"{O}konomisches Denken in ausgew\"{a}hlten Texten des 18. Jahrhunderts.} Heidelberg: Winter.\\ \\

Rommel, T. (2006). \textit{Literaturspalten. B\"{u}cher lesen und verstehen.} Hamburg: merus.\\ \\

Rommel, T., \& Schreiber, M. (2006). \textit{Mapping uncertain territories. Space and place in contemporary theater and drama.} (Contemporary Drama in English; 13). Trier: WVT.\\ \\

W\"{u}nsche , I. (2006). \textit{Galka E. Scheyer \& Die Blaue Vier: Briefwechsel 1924-1945.} Bern: Benteli.\\ \\

W\"{u}nsche, I. (2006). \textit{Galka E. Scheyer \& The Blue Four: Correspondence 1924-1945.} Bern: Benteli.\\ \\



\paragraph{Articles \& Chapters}\textit{ }

\bigskip

Amodeo, I. (2006). Schreiben in mehreren Sprachen: Chiara de Manzini Himmrich - Giuseppe Giambusso - Sonja Guerrera - Piero Salabe. In V. Borso, \& H. Brohm (Eds.), \textit{Transkulturation. Literarische und mediale Grenzraume im deutsch-italienischen Kulturkontakt.} Bielefeld: Transcript, in press.\\ \\

Amodeo, I., \& Ortner-Buchberger, C. (2006). Viewing Italian colonial cinema: From propaganda machine to collective memory. \textit{Palabres. Art. Litterature. Philosophie. 5}(1), in press.\\ \\

Amodeo, I. (2006). Migrant tongues: German - and its others. In D. Dumontet \& F. Zipfel (Eds.), \textit{Ecriture Migrante / Migrant Writing / Schreiben und Migration.} Berlin: Erich Schmidt, in press.\\ \\

Amodeo, I. (2006). Wanderratten e Wolkenhunde: Il gran bazar dell'interculturalita in autori contemporanei di origine non tedesca. In D. Mugnolo (Ed.), \textit{TranScrizioni. Percorsi interculturali nella lingua e nella letteratura tedesca.} Roma: Donzelli, in press.\\ \\

Amodeo, I. (2006): "Bel paese" (ou pas) : Remarques sur les autres litt\�{e}ratures italiennes. In F. Rinner (Ed.), \textit{Identit\�{e} en m\�{e}tamorphose dans l'\�{e}criture contemporaine} (pp. 129-136). Aix-en-Provence: Publications de l'Universit\�{e} de Provence.\\ \\

Amodeo, I. (2006). "comme dans l'atmosph\�{e}re d'un autre monde". Die Oper in der Literatur als Traum und als Wirklichkeit. Ausgew\"{a}hlte Beispiele. In P. Csob�di, G. Gruber, J. K\"{u}hnel, U. M\"{u}ller, O. Panagl, \& F.V. Spechtler (Eds.), \textit{Traum und Wirklichkeit in Theater und Musiktheater. Vortr\"{a}ge und Gespr\"{a}che des Salzburger Symposions 2004} (pp. 49-57). Salzburg: Mueller-Speiser.\\ \\

Amodeo, I. (2006). "Il n'est plus question de patrie": Histoires de migration et leurs configurations esth\�{e}tiques. De la litt\�{e}rature des auteurs francophones d'origine italienne au Canada. L'exemple d'Antonio D'Alfonso. In E. Arendt (Ed.), \textit{Histoires invent\�{e}es}, (pp. 83-93). Frankfurt am Main: Peter Lang.\\ \\

Amodeo, I. (2006), La letteratura della migrazione in Germania. In A. Gnisci (Ed.), \textit{Nuovo Planetario italiano. Guida alla letteratura italiana della migrazione} (pp. 1-8). Rome: Citt� aperta edizioni.\\ \\

Amodeo, I. (2006). Le spectacle du corps. Images corporelles du XVIIe si\�{e}cle au cin\�{e}ma. In E. Erdmann, \& K. Schoell (Eds.), Le comique corporel. Mouvement et comique dans l'espace th\�{e}�tral du XVIIe si\�{e}cle (pp. 149-160). T\"{u}bingen: Gunter Narr Verlag.\\ \\

Amodeo, I. (2006). "L'isola plurale": Kulturtheoretische Konzepte in der Essayistik von Gesualdo Bufalino. In D. Reichardt (Ed.), L'Europa che comincia e finisce: la Sicilia. \textit{Approcci transculturali alla letteratura siciliana. Beitr\"{a}ge zur transkulturellen Ann\"{a}herung an die Literatur Siziliens. Contributions to a transcultural approach to Sicilian literature} (pp. 179-188). Frankfurt am Main: Peter Lang.\\ \\

Birus, H. (2006). Beim Wiederlesen von Jacques Derridas "Schibboleth - pour Paul Celan". In R. L\"{u}deke, \& I. M\"{u}lder-Bach (Eds.), \textit{Wiederholen. Literarische Funktionen und Verfahren} (= M\"{u}nchener Komparatistische Studien, vol. 7) (pp. 103-133). G\"{o}ttingen: Wallstein.\\ \\

Birus, H. (2006). "mein Sohn Zarathustra": Zur Vorgeschichte einer Namengebung. In M. Mayer (Ed.), \textit{Also wie sprach Zarathustra? West-\"{o}stliche Spiegelungen im kulturgeschichtlichen Vergleich} (= Klassische Moderne, vol. 6) (pp. 31-47). W\"{u}rzburg: Ergon.\\ \\

Birus, H. (2006). Von Kaisersaschern nach Pacific Palisades: ein Wegweiser aus dem "Tal der Ahnungslosen" in die "Freie Welt". In D. Felken (Ed.), \textit{Ein Buch, das mein Leben ver\"{a}ndert hat. Liber amicorum f\"{u}r Wolfgang Beck} (pp. 55-57). M\"{u}nchen: C.H. Beck.\\ \\

Frohne, U. (2006). D\�{e}tournement als "verwirklichte Poesie" - Der K\"{u}nstler als Kulturaktivist. In P. Weibel, K. Buol-Wischenau, \& C. Steinle (Eds.), \textit{Peter Weibel. Das offene Werk 1964-1979}, Neue Galerie am Landesmuseum Joanneum Graz (pp. 27-35). Ostfildern: Hatje Cantz.\\ \\

Frohne, U. (2006). Double Feature: Filmadaptionen in zeitgen\"{o}ssischer Kunst. In A.-K. Reulecke (Ed.), \textit{F\"{a}lschungen. Zu Autorschaft und Beweis in Wissenschaften und K\"{u}nsten} (pp. 364-389). Frankfurt am Main: Suhrkamp.\\ \\

Frohne, U. (2006). Media Wars. Strategische Bilder des Krieges. In A. J\"{u}rgens-Kirchhoff (Ed.), \textit{Warshots, Krieg - Kunst \& Medien}, Publikation der Tagung des Sonderforschungsbereichs 437: \textit{Kriegserfahrungen - Krieg und Gesellschaft in der Neuzeit, Schriftenreihe der Guernica-Gesellschaft} (pp. 161-186). Osnabr\"{u}ck: Universit\"{a}tsverlag.\\ \\

Hoffmann, K. (2006). Anwesende Abwesenheit. Kritik und Ann\"{a}herung an eine bildanthropologische Perspektive in der Medienkunst. In S. Kacunko (Ed.), \textit{Kunst- und Bildwissenschaft im Spannungsfeld von Performance und Medienkunst.} Berlin: Philo, in press.\\ \\

Hoffmann, K. (2006). Anwesende Abwesenheit. Kritik und Ann\"{a}herung an eine bildanthropologische Perspektive in der Medienkunst. In S. Kacunko, \& D. Leach (Eds.), \textit{Downdate. Publikation zur Medienkunstkonferenz 2006 in Osnabr\"{u}ck} (pp. 89-113). Berlin, in press.\\ \\

Knorpp, E. (2006). Rezension zur Ausstellung \textit{Barock im Vatikan} (Kunst- und Ausstellungshalle, Bonn) [www.portalkunstgeschichte.de, last accessed: January 19, 2006].\\ \\

Knorpp, E. (2006). Rezension des Ausstellungskatalogs Kunst- und Ausstellungshalle der Bundesrepublik Deutschland [Hrsg.]: \textit{Barock im Vatikan. Kunst und Kultur im Rom der P\"{a}pste II 1572-1676,} Leipzig, 2005 [www.portalkunstgeschichte.de, last accessed: February 22, 2006].\\ \\

Knorpp, E. (2006). Rezension zur Ausstellung \textit{"F\"{u}r immer und ewig? - The world's most photographed"} [www.portalkunstgeschichte.de, last accessed: April 18, 2006].\\ \\

Knorpp, E. (2006). Rezension des Ausstellungskatalogs Max Hollein / Christa Steinle [Hg.]: \textit{Religion Macht Kunst. Die Nazarener,} K\"{o}ln: Verlag der Buchhandlung Walther K\"{o}nig, 2005. \textit{sehepunkte} 6 (2006) [www.sehepunkte. de/2006/06/8705.html, last accessed: June 15, 2006].\\ \\

Knorpp, E. (2006). Rezension des Ausstellungskatalogs Gerhard Kolberg [Hg.]: \textit{Salvador Dal�. La Gare de Perpignan Pop, Op, Yes-yes, Pompier,} Ostfildern: Hatje Cantz, 2006 [www.portalkunstgeschichte.de, last accessed: June 26, 2006].\\ \\

Peters, S (2006). Derek Walcott: Pantomime. In S. Peters, K. Stierstorfer, \& L. Volkmann (Eds.), \textit{Drama II}. (Teaching Contemporary Literature and Culture, vol. 1, pp. 527-544). Trier: WVT.\\ \\

Peters, S. (2006). Marie Jones: Stones in his pocket. In S. Peters, K. Stierstorfer, \& L. Volkmann (Eds.), \textit{Drama I}, (Teaching Contemporary Literature and Culture, vol. 1, pp. 203-219). Trier: WVT.\\ \\

Peters, S. (2006). Richard III. In R. Petersohn, \& L. Volkmann (Eds.), \textit{Shakespeare didaktisch II: Ausgew\"{a}hlte Dramen und Sonette f\"{u}r den Unterricht} (pp. 121-134). T\"{u}bingen: Stauffenburg.\\ \\

Pfander, G. \& W\"{u}nsche, I. (2006). Exploring infinity: Number sequences in modern art. \textit{Arkhai}, in press.\\ \\

Rommel, T. (2006). Cynthia Dereli, \textit{A war culture in action. A study of the literature of the Crimean War period.} Oxford: Lang, 2003. Zeitschrift f\"{u}r Anglistik und Amerikanistik, 54 (3), 317-318.\\ \\

Rommel, T. (2006). Francis Hutcheson. In D. Herz, \& V. Weinberger (Eds.), \textit{Lexikon \"{o}konomischer Werke. 650 wegweisende Schriften von der Antike bis ins 20. Jahrhundert} (pp. 212-213). D\"{u}sseldorf: Verlag Wirtschaft \& Finanzen.\\ \\

Rommel, T. (2006). \"{u}ber "Wahre S\"{u}dseegeschichten" und andere Erfindungen. Literarische Fiktionalisierung der S\"{u}dsee. In G. Febel, A. Hamilton, M. Blumberg, H. de Souza, \& C. Sandten (Eds.), \textit{Zwischen Kontakt und Konflikt. Perspektiven der Postkolonialismus-Forschung} (pp. 165-175). Trier: WVT.\\ \\

W\"{u}nsche, I. (2006). Biocentric Modernism: The other side of the avant-garde. In V. Lehoda (Ed.), \textit{Local Strategies. International Ambitions: Modern Art in Central Europe, 1918 -1968, Conference Proceedings} (pp. 125-132). Prague: Academy of Sciences.\\ \\

W\"{u}nsche, I. (2006). Lebendige Formen und bewegte Linien: Organische Abstraktionen in der Kunst der klassischen Moderne. In U. Lehmann (Ed.), \textit{Floating Forms: Abstract Art Now} (pp. 10-22). Bielefeld: Kerber.\\ \\





