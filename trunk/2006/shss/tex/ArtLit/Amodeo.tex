\subsection{Prof. Dr. Immacolata Amodeo}


\textbf{Main Research Interests}\\[-0.25cm]
\begin{enumerate}
\item[$\bullet$]	Comparative Literature
\item[$\bullet$]	Comparative Media Studies
\item[$\bullet$]	Literary and Cultural Theory
\item[$\bullet$]	Intercultural and Transcultural Studies
\item[$\bullet$]	Literature in its Relations to other Media (mainly Opera, Music, and Film)
\item[$\bullet$]	Multilingualism and Literature
\item[$\bullet$]	Migration Literatures
\item[$\bullet$]	Aesthetic Aspects of Cultural Conflicts and Cultural Encounters
\end{enumerate}

\vspace{0.6cm}
\textbf{Research Activities}\\[-0.25cm]

Immacolata Amodeo has continued and diversified her research activities in the field of migrant literatures: (a) she worked on gender-related questions and literary texts written by migrant women writers of Italian origin in Germany; (b)  she studied the works of migrant writers with Spanish as first language (in the Federal Republic of Germany and the GDR); (c) she has further developed the comparative study of migrant literatures; d) she has analyzed linguistic aspects of migrant literatures. The analysis of linguistic aspects of migrant literatures led to the organization and establishment of an international and interdisciplinary research group (scholars from Germany, Italy, France and several African countries are involved) under her direction and a joint grant application with a research proposal on multilingualism and migrant literatures in Germany, Italy, and Canada from a comparative perspective (VolkswagenStiftung, program \textit{Studiengruppen zu Migration und Integration}, decision pending). Furthermore, she has started translating (from German into Italian) a collection of short-stories by the author of Italian origin Franco Biondi living in Germany and writing in German which she is committed to editing for publication in Italy in 2007. Immacolata Amodeo has also refined her analysis of political and cultural mappings of criminality and nationality in traditional and new media. This research (in collaboration with Eva Erdman, Universit�t Erfurt) lead to the idea of an online atlas of crime fiction and to data collection on the basis of contemporary crime novels. 


\vspace{0.6cm}
\textbf{Organization of Scientific Conferences}\\[-0.25cm]
\begin{enumerate}

\item[$\bullet$] March 2006\newline
 International University Bremen\newline
"Contemporary Indian Literature Conference"\newline (organized together with IUB students Annika Carlson, Cristina Galusca, Christin H�ne, Saskia Schirmann)\newline
funded by the Indian Embassy in Germany\newline
international participants: 40
\end{enumerate}



 

\vspace{0.6cm}
\textbf{Other Professional Activities}\\[-0.25cm]
\begin{enumerate}
\item[$\bullet$]	Committee work and reviewing for foundations and other institutions (the accreditation agency ACQUIN, the Alexander von Humboldt Foundation, the Studienstiftung des Deutschen Volkes, DAAD etc.)
\item[$\bullet$]	Member of the jury for the International Literature Prize \textit{Albatros} of the G�nter Grass-Stiftung Bremen
\item[$\bullet$]	Member of INPUTS (Institut f�r postkoloniale und transkulturelle Studien, Universit�t Bremen)
\item[$\bullet$]  Member of the advisory board of the exhibition \textit{Annelino. Zur Geschichte der Italiener in Ludwigshafen}, Kultur-Rhein-Neckar e.V. in collaboration with the Stadtmuseum Ludwigshafen (December 2005-March 2006)
\item[$\bullet$]  Moderation of the round table discussion with the authors Silvia Di Natale, Marisa Fenoglio, Lisa Mazzi, Venera Tirreno Schneider, Istituto Italiano di Cultura di Francoforte / Coordinamento Donne Francoforte, Frankfurt am Main, 17 March 2006
\item[$\bullet$]  Coordinator (together with Prof. Dr. Nicola Spakowski) of the Humanities Graduate Program \textit{Intercultural Humanities}, International University Bremen
\item[$\bullet$]  External PhD examiner University of Bayreuth, Germany

\end{enumerate}


\vspace{0.6cm}
\textbf{PhD-Students}\\[-0.25cm]

Irina Chkhaidze \newline
\textit{Posthuman Bodies: Conceptualising Hybrids}\\[-0.15cm]

Heidrun Gudrun H�rner \newline
\textit{China's New Literature is Taking Place Abroad: How Authors of Chinese Origin in France and Germany Write - A Comparative Analysis}\\[-0.15cm]

Brigitte Kienle \newline
\textit{On "New World Literatures": Globalization, Migration and Aesthetics in the Literature of Migrant African Authors in Italy}\\[-0.15cm]

