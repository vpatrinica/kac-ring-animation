\subsection[Prof. Dr. Ursula Anna Frohne] {Prof. Dr. Ursula Anna Frohne {\normalfont\normalsize\newline(Left IUB in March 2006)}}
 


\textbf{Main Research Interests}\\[-0.25cm]
\begin{enumerate}
\item[$\bullet$]	$19^{\rm th} - 21^{\rm st}$ Centuries Art (American and European Art, Self-Referentiality, Art in Public Space, Ephemeral Art Practices, Exhibition History)
\item[$\bullet$]	New Media (Photography, Film, Video, Electronic Media, Multi-Media Installations and Media Transcriptions)
\item[$\bullet$]  Science of the Image (Visual Culture and Cultural Techniques)
\item[$\bullet$]	Art and Politics (Politics and Ethics of Visibility, Art and Agency, Iconicity of Social Reality)
\item[$\bullet$]  Art Market Phenomena and Value Transformations in Art
\end{enumerate}


\vspace{0.6cm}
\textbf{Research Activities}\\[-0.25cm]

Ursula Frohne's longstanding project addresses self-referential practices in art since the mid-$20^{\rm th}$ century. The book emerging from this work explores the conceptual dimensions of diverse appropriative artistic methods and their inter-medial transcriptions. It theorizes the aesthetic and cultural significance of mimetic repetition, recycling, copying, and re-staging of visual material in the field of art and the role of new media technologies in this. A second project is dedicated to the investigation of reflections of filmic space, particularly in contemporary media installations, consisting of illusionary spatial projections with extensions into the exhibition space. It is based on a collaborative research initiative between the Art History Department of the Universit�t Frankfurt am Main and the Media Studies Department of the Universit�t Jena; an application for DFG-funding has been submitted in June 2006. In the context of the Graduiertenkolleg \textit{Bild - K\"{o}rper - Medium. Eine anthropologische Perspektive} at the Staatliche Hochschule f\"{u}r Gestaltung Karlsruhe to which Ursula Frohne is affiliated as an external professor since 2003, she has dedicated her research to image theory, visual culture and politics. Questions concerning visual representations of social reality, ethics and politics of visibility, the potential of images to initiate agency and intercultural dimensions of image transfers are of major interest in this context. Another project is based on a research collaboration with the Neues Museum Weserburg Bremen. Its preliminary investigations of collaborative structures and process-based concepts in the field of art since the 1960s will be materializing in a grant application to further historically contextualize and reconstruct ephemeral art tendencies. Finally Ursula Frohne is dedicated to the investigation of significant shifts in the value system of the modern and contemporary art market, entailing the relation between the aesthetic and the economic value of the art work. Special emphasis is given to the value producing systems (price politics of galleries, auction houses and the role of annual art fairs for the international art market, Corporate Collecting, etc.) and the emergence of new art markets in zones of political transformation. 

\vspace{0.6cm}


\textbf{Funded Projects}\\[-0.25cm]
\begin{enumerate}
\item[$\bullet$]   Participation in the collaborative grant application for the Graduiertenkolleg \textit{Bild - K\"{o}rper - Medium. Eine anthropologische Perspektive} at the Staatliche Hochschule f\"{u}r Gestaltung Karlsruhe concerning the third funding phase 2006 - 2009.\newline
 DFG-funding was granted in August 2006.
\end{enumerate}


\vspace{0.6cm}
\textbf{Organization of Scientific Conferences}\\[-0.25cm]
\begin{enumerate}
\item[$\bullet$]	May 2006\newline
	International University Bremen\newline
  "Art and Metaphysics in the Twentieth Century and Beyond"\newline
  (together with Prof. Dr. Paul Crowther \& Prof. Dr. Isabel W�nsche, IUB)\newpage
	funded by the DFG\newline
	international participants: 75
\item[$\bullet$]	July 2006\newline
	Universit\"{a}t Karlsruhe\newline
	"Politische Kunst - Politik der Kunst"\newline
	(in collaboration with Prof. Dr. Jutta Held)\newline
	funded by the Universit�t Karlsruhe and the Guernica Gesellschaft\newline
	international participants: 50
\end{enumerate}


\vspace{0.6cm}
\textbf{Other Professional Activities}\\[-0.25cm]
\begin{enumerate}
\item[$\bullet$] Editorial Board Member of the book series \textit{Interfaces. Studies in Visual Culture}, University of New England Press (Editors: Prof. Dr. Adrian Randolph und Prof. Dr. Mark J. Williams)
\item[$\bullet$] Editorial Board Member of the electronic journal \textit{Vectors: Culture and Technology in a Dynamic Vernacular}, University of Southern California
\item[$\bullet$] Member of the Scientific Board of the Research Association of the Archive for Small Press \& Communication / ASPC, Neues Museum Weserburg, Bremen
\item[$\bullet$] Selection Committee Member at the Fulbright Foundation
\item[$\bullet$] Member of the Board of the Haus im Park at the Klinikum Bremen-Ost
\item[$\bullet$] Member of the Editorial Board of the Jahrbuch des Wallraf-Richartz-Museum and Museum Ludwig, K\"oln
\item[$\bullet$] Member of the Lehrekommission at the Universit\"{a}t K\"{o}ln
\item[$\bullet$] Member of the Exzellenzclusterinitiative \textit{Media: Material Conditions and Cultural Practice} at the Universit\"{a}t K\"{o}ln
\item[$\bullet$] Member of the committee for the development of a concept for a Graduate School \textit{Wissen - Rezeption - Transkulturalit\"{a}t} at the University of Cologne under the thematic focus "Medialit\"{a}t[en]: Prozesse der Formierung und Transformation"
\item[$\bullet$] Member of the Scientific Board of conferences organized within the Graduiertenkolleg "Bild - K\"{o}rper - Medium", Staatliche Hochschule f�r Gestaltung Karlsruhe
\end{enumerate}




\vspace{0.6cm}
\textbf{PhD-Students}\\[-0.25cm]


Julia Bannenberg\newline
\textit{"Deutschlandbilder": Ein deutsch-deutsches Ausstellungskonzept}\\[-0.15cm]

Kirsten Fitzke\newline
\textit{Lebende Denkm\"{a}ler: Darstellungen kriegsversehrter K\"{o}rper im 20. Jahrhundert}\\[-0.15cm]

Sophie Gerlach\newline
\textit{Neo Rauch - Bilder 1984 - 2005. Maler und Werk als Projektionsfl\"{a}che gesellschaftspolitischer Tendenzen}\\[-0.15cm]

Katja Hoffmann\newline
\textit{Zur Relevanz des 'iconic turn' in der Kunstwissenschaft: \"{U}berlegungen zur Ph\"{a}nomenologie, Theorie und Historiografie eines Paradigmas - am Beispiel der Documenta 11 (2002) und der Ausstellung Film und Fotografie (1929)}\\[-0.15cm]

Charlotte Kraft\newline
\textit{Rosemarie Trockel: Identit\"{a}ten}\\[-0.15cm]

Sebastian Neusser\newline
\textit{Die submediale Pr�senz des K�rpers in der Kunst}\\[-0.15cm]

\newpage
Stefanie Zobel\newline
\textit{(Working title) Zeitgen�ssische (Re)Pr\"{a}sentation von Medienkunst: Ausstellungskonzepte und -formate}\\[-0.15cm]

Doroth\'{e}e Brill\newline
\textit{Shock of the Void. The Senseless as Strategy in Dada and Fluxus}


\vspace{0.6cm}
\textbf{Research Personnel}\\[-0.25cm]

Katja Hoffmann \newline Research Associate
