\subsection[PD Dr. Susanne Peters] {PD Dr. Susanne Peters {\normalfont\normalsize\newline (Visiting Professor in Fall 2006)}}


\textbf{Main Research Interests}\\[-0.25cm]
\begin{enumerate}
\item[$\bullet$]	Sense Perception in Literature
\item[$\bullet$]	Orality and Literacy
\item[$\bullet$]	Intercultural Communication
\item[$\bullet$]	Intermediality
\item[$\bullet$]	History and Theory of Drama and Theatre
\item[$\bullet$]	Censorship
\end{enumerate}


\vspace{0.6cm}
\textbf{Research Activities}\\[-0.25cm]

After her habilitation with a major study on the role and function of letters in drama from the $16^{\rm th}$ century to the present, Susanne Peters has focused her research on the relationship between oral and written discourse in fiction and drama and the creation and distortion of public and private spheres in literature. She also continued to enquire into the role of sense perception in literature as well as film that she started with her dissertation on the works of James Joyce. Further areas of research were intercultural communication with a special interest in intercultural incompetence, and literatures from English-speaking countries. With two colleagues she has recently founded and co-edits a new series of scholarly books on teaching contemporary literature and culture, of which the first volume in two parts, offering readings of 31 plays, has just been published. The series presents accessible readings of major works of contemporary literature in English for schools and universities, with an emphasis on teachability. The next volume, again in two parts, dealing with the teaching of novels, is currently in preparation and will appear in 2007. Susanne Peters has held visiting professorships at the universities of Leipzig, D�sseldorf, and Stuttgart. She joined IUB in spring 2006 as lecturer and is currently visiting professor.
\vspace{0.6cm}

\textbf{Other Professional Activities}\\[-0.25cm]
\begin{enumerate}
\item[$\bullet$] Member of Deutscher Anglistenverband
\item[$\bullet$] Member of Deutsche Shakespeare-Gesellschaft
\item[$\bullet$] Member of Deutsch-Britische Gesellschaft
\item[$\bullet$] Member of Gesellschaft f�r die Neuen Englischsprachigen Literaturen (GNEL/ASNEL)
\item[$\bullet$] Member of The German Society for Contemporary Drama and Theatre in English (CDE)
\end{enumerate}

