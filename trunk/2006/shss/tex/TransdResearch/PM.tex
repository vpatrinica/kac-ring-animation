\subsection{"Priorisierung in der Medizin" (Prioritizing in Medicine)
FOR 655\\
Deutsche Forschungsgemeinschaft}


\paragraph{Speaker/Coordinator:} Adele Diederich


\vspace{0.6cm}
Researchers from different fields (Economics, Law, Medicine, Philosophy, and Psychology) initiated a research group on ``Priorisierung in der Medizin:Eine theoretische und empirische Analyse unter besonderer Ber�cksichtigung der Gesetzlichen Krankenversicherung (GKV)'' (Prioritizing in Medicine: A Theoretical and Empirical Analysis  in consideration of the public health insurance system).


\vspace{0.6cm}
Priorities have always been set in health care, often implicitly and without openly presenting the values and principles on which these priorities are based. There are no clear and established procedures how to generate and process information to reach such important decisions. The goal of the research group is to identify these principles and answer questions related to prioritizing both theoretically and empirically. In particular, we distinguish between vertical and horizontal prioritizing. Vertical prioritizing refers to establishing a hierarchy within an area of specialty or disease category whereas horizontal prioritizing refers to setting priorities among different areas of specialty or among different disease groups.


\vspace{0.6cm}
Three main aspects are of special important: a) Empirical investigations, i.e., different from any current approaches we focus systematically on needs and interest of patients, physicians, and the general public. Several methods are utilized, from qualitative interviews to a representative survey to conjoint analysis and models such as random utility models. Both, aspects of vertical and horizontal prioritizing are considered. Closely intertwined with vertical prioritizing is b) to establish priority criteria for specific diseases, in particular hemophilia, rheumatoid arthritis, organ allocation for kidneys, heart, liver, and c) a philosophical, juridical, and economical discussion of prioritizing.


\vspace{0.6cm}
The transdiciplinary interaction between researchers from a variety of disciplines and in particular the empirical part of the project is in itself unique and challenging.


\vspace{0.6cm}
A grant proposal was sent to the German Research Foundation (DFG) in November 2005. At hearing took place on November 23, 2006. The reviewers recommended granting the Forschergruppe.
