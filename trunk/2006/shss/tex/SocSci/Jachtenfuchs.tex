\subsection[Prof. Dr. Markus Jachtenfuchs] {Prof. Dr. Markus Jachtenfuchs{\normalfont\normalsize\newline(Left IUB in July 2006)}}


\vspace{0.3cm}
\textbf{Main Research Interests}\\[-0.25cm]
\begin{enumerate}
\item[$\bullet$]	European Integration
\item[$\bullet$]	Transformation of the State
\item[$\bullet$]	International Governance
\end{enumerate}


\vspace{0.6cm}
\textbf{Research Activities}\\[-0.25cm]

In 2006, Markus Jachtenfuchs has been involved in three main research activities. First, within the Collaborative Research Center (CRC) on Transformations of the State (Sonderforschungsbereich Staatlichkeit im Wandel, 2003-2006), he has participated in implementing the overall research agenda of the CRC and preparing the next application round in September 2006. This included the coordination of research projects in four thematic fields (resources, legitimacy, rule of law, and intervention) and the preparation of joint publications from these projects which will appear in 2006. Second, he has directed a research project on "The Internationalization of the Monopoly of Force" within the CRC dealing with the internationalization of police activity. Third, he has continued writing on the findings and potential of a governance approach to European integration.


\vspace{0.6cm}
\textbf{Funded Projects}\\[-0.25cm]
\begin{enumerate}
\item[$\bullet$]	"The Internationalization of the Monopoly of Force,"\newline
	funded by the Deutsche Forschungsgemeinschaft as part of the CRC 	"Transformations of the State"
\end{enumerate}


\vspace{0.6cm}
\textbf{Other Professional Activities}\\[-0.25cm]
\begin{enumerate}
\item[$\bullet$]	Managing editor of reviewed book series "Weltpolitik im 21. Jahrhundert" (World Politics in the 21st century), edited by the International Relations Section of the German Political Science Association
\item[$\bullet$]	Member of scientific or editorial boards: Journal of International Relations and Development, Zeitschrift f�r internationale Beziehungen, Politique Europ�enne, Institut f�r europ�ische Politik, COST A 24 Action: The Social Construction of Threats
\item[$\bullet$]	Reviewer for professional journals:\textit{ Journal of European Public Policy, Journal of Common Market Studies, European Journal of International Relations, International Organization, Zeitschrift f�r Internationale Beziehungen, Journal of International Relations and Development, Politische Vierteljahresschrift}
\item[$\bullet$]	Reviewer for research grants: Deutsche Forschungsgemeinschaft, Fritz-Thyssen-Stiftung, Volkswagenstiftung, Austrian Ministry of Science
\item[$\bullet$]	Reviewer and selection committee member for professorships at various German universities
\item[$\bullet$]	Member of accreditation committee for accrediation agency ACQUIN
\item[$\bullet$]	Co-chair of MA in International Relations (joint program IUB-Uni Bremen)
\end{enumerate}


\vspace{0.6cm}
\textbf{PhD-Students}\\[-0.25cm]

Simon Dalferth\newline
\textit{ Police Cooperation in the EU. Efficiency versus Civil Liberties}\\[-0.15cm]

Eva Herschinger\newline
\textit{ Understanding Security Policy and its Role for Identity Construction - an Analysis of the French Security Discourse}\\[-0.15cm]

Christiane Kasack\newline
\textit{ EU Justice and Home Affairs and the Transformation of the State}\\[-0.15cm]

Akos Kopper\newline
\textit{ The Meaning of Sovereignty in Immigration-related Issues among the EU and its Neighbours}\\[-0.15cm]

Moritz Wei�\newline
\textit{ Preferences and Institutional Settings. The European Union's Security Policy since the mid-1990s}


\vspace{0.6cm}
\textbf{Scientific Personnel}\\[-0.25cm]

Dr. J�rg Friedrichs\newline
Post doctoral researcher in the CRC project on "The Internationalization of the Monopoly of Force"

