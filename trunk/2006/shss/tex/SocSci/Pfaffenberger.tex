\subsection{Prof. Dr. Wolfgang Pfaffenberger}

\vspace{0.2cm}
\textbf{Main Research Interests}\\[-0.25cm]
\begin{enumerate}
\item[$\bullet$]	Energy Economics
\item[$\bullet$]	Electricity Economics
\item[$\bullet$]	Macroeconomic Aspects of Energy and Climate Policy
\item[$\bullet$]	Promotion of Energy Efficiency
\end{enumerate}


\vspace{0.5cm}
\textbf{Research Activities}\\[-0.25cm]

Network industries often display characteristics of a natural monopoly. Traditionally they were therefore organized as regional or local monopolies. In recent years in many parts of the world re-regulation took place by separating the bottleneck parts of the system (usually networks or parts of them) from the rest of the industry by various forms of regulation. For economists like Wolfgang Pfaffenberger it is of special interest to study the interaction between government regulation and the development of the markets and to review the regulatory concepts in light of the experience in the markets on the basis of empirical and modeling studies. At the same time the energy industry is a key sector for greenhouse gas reduction strategies. In this area a fundamental change of policy formation took place (international market for GHG certificates) on the basis of the global nature of GHG emissions whereas in the past direct intervention or taxation on the basis of national goals were dominant. It is of interest to observe this process of transformation. Economic models can serve as a basis for analyzing the rationale of policy making. Wolfgang Pfaffenberger undertakes research on this topic at Bremer Energie Institut within a multidisciplinary environment. Methods and approaches used depend on the subject of the project. In the first quarter of 2006 Wolfgang Pfaffenberger's research had was a focus on macroeconomic effects of renewable energy in connection with the "green roads to growth" conference. All projects are described in the annual report of the institute itself. The institute is an independent institute affiliated with Universit�t Bremen and IUB.


\vspace{0.6cm}
\textbf{Funded Projects}\\[-0.25cm]

As Director of the Bremer Energie Institut (until March 31, 2006) Wolfgang Pfaffenberger was nominal principal investigator of the following funded research projects:
\begin{enumerate}
\item[$\bullet$]	"Entwicklung des Endenergieverbrauchs f�r Heizung und Warmwasser bei Einfamilienh�usern,"\newline
funded by Bundesamt f�r Bauwesen und Raumordnung, Bonn
\item[$\bullet$]	"Ermittlung von Effekten des KfW-CO2-Geb�udesanierungsprogramms"
funded by KfW Bankengruppe, Frankfurt
\item[$\bullet$]	Implementation of Further-Education Measures on the topic of energy efficiency in buildings and "Energiepass"\newline
funded by EWE AG
\item[$\bullet$]	Coordination of the Bremer Contracting-Offensive\newline
funded by Bremer Energie-Konsens GmbH
\item[$\bullet$]	"Holzfeuerungsanlagen f�r Wohngeb�ude > 1000 m$^2$ Nutzfl�che"\newline
funded by Bundesamt f�r Bauwesen und Raumordnung (Bonn) and Bremer Energie-Konsens GmbH
\item[$\bullet$]	"Erfolgskontrolle des Einsatzes zentraler elektronischer Einzelraumremperaturregler in Wohnungen bez�glich der Akzeptanz und der Reduzierung des Energieverbrauchs"\newline
funded by Bremer Energie-Konsens GmbH
\item[$\bullet$]	"Analyses and Guidelines for Implementation of CHP Directive 2004/8/EC"\newline
funded by the European Commission, Directorate-General for Energy and Transport (GD TREN)
\item[$\bullet$]	Scientific evaluation of a model project on "Emissionshandel f�r Kleinverbraucher und Haushalte",\newline
funded by Landesinnungsverband des Schornseinfegerhandwerks Hessen
\end{enumerate}


\vspace{0.6cm}
\textbf{Organization of Scientific Conferences}\\[-0.25cm]
\begin{enumerate}
\item[$\bullet$]	June 2006\newline
Potsdam\newline
Annual international conference of the International Association for Energy Economics (IAEE)\newline
Member of the Program Committee
\end{enumerate}


\vspace{0.6cm}
\textbf{Other Professional Activities}\\[-0.25cm]
\begin{enumerate}
\item[$\bullet$]	Member of committee "Environmental and Resource Economics", Gesellschaft f�r Wirtschaftswissenschaften.
\item[$\bullet$]	Member of the board of "Gesellschaft f�r Energiewissenschaft und Energiepolitik (GEE)" (German branch of International Association for Energy Economics IAEE).
\item[$\bullet$] Kurzgutachten zu den Bestimmungsfaktoren der Gaspreise in Deutschland unter besonderer Ber\"ucksichtigung der Stadtwerke Auftraggeber: Ein Verbund norddeutscher Stadtwerk
\item[$\bullet$] Konzeption der energetischen Modernisierung eines Wohngebietes in Bremerhaven\newline Auftraggeber: Wohnungsgesellschaft ST\"AWOG Bremerhaven
\item[$\bullet$] Holzpelletheizsysteme f\"ur Ein-, Zwei- und kleine Mehrfamilienh\"auser - Technische Rahmenbedingungen und Wirtschaftlikeitsaspekte \newline Auftraggeber: Bremer Energie-Konsens GmbH
\item[$\bullet$] Detaillierung eines Konzepts zur Qualit\"atssicherung von Energieausweise \newline
Auftraggeber: F\"OGES F\"ordergemeinschaft f\"ur Geb\"aude- und Energiesysteme GmbH
\item[$\bullet$] Analyse des nationalen Potenzials f\"ur den Einsatz hocheffizienter KWK, einschlie�lich hocheffizienter Kleinst-KWK, unter Ber\"ucksichtigung der sich aus der EU-KWK-RL ergebenden Aspekte \newline
Auftraggeber: Bundesministerium f\"ur Wirtschaft und Arbeit
\item[$\bullet$] Qualit\"atssicherung Energieausweis: Gutachten zu Zielen, Nutzen, Instrumentarium und Methodik eines Systemes zur Qualit\"atssicherung von Energieausweisen \newline
Auftraggeber: Vereinigung der deutschen Zentralheizungswirtschaft e.V., Bonn
\item[$\bullet$] Einfluss des Heizsystems auf Schimmelpilze in  Wohnungen oder: Vermeiden Gasetagenheinzungen Schimmelpilze?
Auftraggerber: Bundesvereiningung der Firmen im Gas- und Wasserfach e.V. - figawa
\item[$\bullet$] Machbarkeitsstudie f\"ur eine Grasraffinerie\newline
Auftraggeber: Hanseatische Naturentwicklung GmbH (haneg)
\item[$\bullet$] Entwicklung des Endenergieverbrauchs f\"ur Heinzung und Warmwasser bei Einfamilienh\"ausern \newline
Auftraggeber: Bundesamt f\"ur Bauwesen und Raumordnung, Bonn
\end{enumerate}


\vspace{0.6cm}
\textbf{PhD-Students}\\[-0.25cm]

Stefanie Kesting\newline
\textit{Transmission Network Access Regulation in the European Gas Market}\newline
Defense: May 2006\\[-0.15cm]

Andres Ojeda\newline
\textit{Simulation Model for Electricity Pricing in a Market Environment}\newline
DAAD sandwich grant

\newpage
\vspace{0.6cm}
\textbf{Research Personnel}\\[-0.25cm]

As Director of the Bremer Energie Institut (until March 31, 2006) Wolfgang Pfaffenberger was nominal supervisor of the following research associates (all of which are funded through diverse third-party grants):\\[-0.2cm]

Dr.-Ing. Klaus-Dieter Clausnitzer\newline
Dr. rer. pol. J�rgen Gabriel\newline
Dr.-Ing. Bernd Eikmeier\newline
Dr. rer. nat. Karin Jahn\newline
Dipl.-Ing. Wolfgang Schulz\newline
Dipl.-Ing. Bernd Eikmeier
