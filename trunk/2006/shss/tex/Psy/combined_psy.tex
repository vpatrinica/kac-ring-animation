%the function for the book publication could be as follows [it was not implemented here]
%1=author;2=year;3=title;4=publisher + comments
%\newcommand{\book}[4]{#1 (#2). \textit{#3}. #4}

\subsection{Publications}




	\paragraph{Books} \textit{ }
	
\bigskip	

F\"{o}rster, J. (2006). \textit{Kleine Einf\"{u}hrung ins SchubladenDenken. Vom Nutzen und Nachteil des Vorurteils}. M\"{u}nchen: DVA/Random House, in press. 

Shields, S.A., \& Kappas, A. (Eds.) (2006). \textit{Magda B. Arnold's contribution to emotion research and theory}. New York: Psychology Press.

Strack, F., \& F\"{o}rster J. (Eds.) (2006). Social Cognition (\textit{Frontiers of Social Psychology }(9 Volumes), edited by J. Forgas, \& A. Kruglanski, Volume 7). Mahwah, NJ: Lawrence Erlbaum Associates, in press.


\paragraph{Articles \& Chapters}\textit{ }

\bigskip

Bockhorst, D., \& Diederich, A. (2006). Personelle und situative Moderatoren riskanter Entscheidungen. In H. Hecht et al. (Eds.), \textit{Experimentelle Psychologie. Beitr\"{a}ge zur 48. Tagung experimentell arbeitender Psychologen}. Lengerich: Pabst Science Publishers, in press.

Colonius, H., \& Diederich, A. (2006). The race model inequality: Interpreting a geometric measure of the amount of violation. \textit{Psychological Review, 113}, 1, 148-154.

Diederich, A. (2006). Mathematische Modellierung. In: K. Pawlik (Ed.), \textit{Handbuch Psychologie} (pp. 535-554). Heidelberg: Springer.

Diederich, A., \& Busemeyer, J.R. (2006). Modeling the effects of payoff on response bias in a perceptual discrimination task: Threshold-bound, drift-rate-change, or two-stage-processing hypothesis. \textit{Perception \& Psychophysics, 68}, 2, 194-207.

Diederich, A., \& Colonius, H. (2006). Why two "distractors" are better than one: Modeling the effect of non-target auditory and tactile stimuli on visual saccadic reaction time. \textit{Experimental Brain Research}, in press.

Diederich, A., \& Colonius, H. (2006). Spatial effects in visual-tactile saccadic reaction time. \textit{Perception \& Psychophysics}, in press (http://www.psychonomic.org/PP/forthcoming.html).

F\"{o}rster, J. (2006). Flexible Kreativit\"{a}t. \textit{Personalwirtschaft}, 12, 23-25.

F\"{o}rster, J. (2006). Approach/avoidance conflict. In \& K. Vohs, \& R. Baumeister (Eds.), \textit{Encyclopedia of Social Psychology}, in press.

F\"{o}rster, J. (2006). Unconscious influences of the living environment on creative and analytic thinking. In P. Meusburger (Ed.), \textit{Milieus of creativity}. New York: Springer, in press.

F\"{o}rster, J. (2006). Embodiment in context. In G. Semin, \& E. Smith (Eds.), \textit{Embodiment}. New York, NJ: Lawrence Erlbaum Associates, in press.

F\"{o}rster, J., \& Denzler, M. (2006). Die Theorie des regulatorischen Fokus. In V. Brandst\"{a}tter, \& J. Otto (Eds.), \textit{Handbuch der Psychologie, Band "Motivation und Emotion"}. Berlin: Hogrefe, in press.

F\"{o}rster, J., \& Denzler, M. (2006). Self-Regulation. In F. Strack, \& J. F\"{o}rster (Eds.), Social cognition (\textit{Frontiers of Social Psychology, Volume 7, edited by} J. Forgas \& A. Kruglanski). Mahwah, NJ: Erlbaum, in press.

F\"{o}rster, J., \& Denzler, M. (2006). Selbstregulation. In H.W. Bierhoff, \& D. Frey (Eds.), \textit{Handbuch der Sozialpsychologie und Kommunikationspsychologie} (pp. 33-39). G\"{o}ttingen: Hogrefe. 

F\"{o}rster, J., \& Denzler, M. (2006). Kreativit\"{a}t. In J. Funke, \& P. Frensch (Eds.), \textit{Handbuch der Psychologie, Band "Allgemeine Psychologie"} (pp. 446- 454). Berlin: Hogrefe. 

F\"{o}rster, J., Friedman, R., \"{o}zelsel, A., \& Denzler, M. (2006). Enactment of approach and avoidance behavior influences the scope of perceptual and conceptual attention. \textit{Journal of Experimental Social Psychology}, 42, 133-146.

F\"{o}rster, J., \& Liberman, N. (2006). Inhibition processes in comparisons. In D.A. Stapel, \& J. Suls (Ed.), \textit{Assimilation and contrast in social psychology}  (pp. 231-261). New York: Psychology Press. 

F\"{o}rster, J., \& Liberman, N. (2006). Knowledge activation. In E. T. Higgins, \& A. W. Kruglanski (Eds.),\textit{ Social Psychology: Handbook of basic principles} (pp. 243-271; 2nd Edition). New York: Guilford.

F\"{o}rster, J., Liberman, N., \& Friedman, R. (2006). What do we prime? Principles and experimental manipulations to distinguish between semantic, procedural and goal priming. In E. Morsella, J. Bargh, \& P. Gollwitzer (Eds.), \textit{The psychology of action and goals}. New York: Guilford, in press.

F\"{o}rster, J., \& Werth, L. (2006). Regulatory focus theory. In G. Moskowitz, \& H. Grant (Eds.), \textit{Goals}. New York: Guilford, in press.

Friedman, R., F\"{o}rster, J., \& Denzler, M. (2006). Interactive Effects of Mood and Task Framing on Creative Generation. \textit{Creativity Research Journal}, in press.

Friedman, R., \& F\"{o}rster, J. (2006). Activation and measurement of motivational states. In A.J. Elliott (Ed.), \textit{Handbook of approach and avoidance motivation}. New York: Guilford, in press.

Hannover, B., \& K\"{u}hnen, U. (2006). A connectionist model of the self or just a general learning model? Comment on "Connectionism and self: James, Mead, and the stream of enculturated consciousness" by Yoshihisa Kashima. \textit{Psychological Inquiry}, in press.

Hannover, B., \& K\"{u}hnen, U. (2006). Culture and social cognition in human interaction. In F. Strack, \& J. F\"{o}rster (Eds.), Social cognition (\textit{Frontiers of Social Psychology, Volume 7, edited by} J. Forgas \& A. Kruglanski). Mahwah, NJ: Erlbaum, in press.

Harvey, M., Olk, B., Newport, R., \& Jackson, S.R. (2006). Impaired orientation processing in hemispatial neglect. \textit{Neuroreport, 17}, in press.

Kappas, A. (2006). Geheimnisvolle Tr\"{a}nen: Das Ph\"{a}nomen des Weinens aus der Sicht der Neurosemiotik. \textit{Zeitschrift f\"{u}r Semiotik, 28}, 295-309.

Kappas, A. (2006). Appraisals are direct, immediate, intuitive, and unwitting ... and some are reflective ... . \textit{Cognition and Emotion, 20}, 952-975.

Kappas, A., \& M\"{u}ller, M.G. (2006). Bild und Emotion - Ein neues Forschungsfeld. \textit{Publizistik, 51}, 3-23.

Kingstone, A., Ristic, J., \& Olk, B. (2006). Rethinking voluntary and involuntary spatial attention. \textit{Abstracts of the Psychonomic Society, 11}, 204.

K\"{u}hnen, U. (2006). Kultur und Kognition - westliches und \"{o}stliches Denken im Vergleich. \textit{Deutsche Zeitschrift f\"{u}r Akkupunktur, 2}, 8-12.

Laudien, J.H., K\"{u}ster, D., Sojka, B., Ferstl, R., \& Pause, B.M. (2006). Central odor processing in subjects experiencing helplessness. \textit{Brain Research, 1120}, 141-150.

Liberman, N., \& F\"{o}rster, J., \& Higgins, E.T. (2006). Set/reset or inhibition after goal fulfillment? A fair test between two mechanisms producing assimilation and contrast. \textit{Journal of Experimental Social Psychology}, in press.

Liberman, N., \& F\"{o}rster, J. (2006). Inferences from decision difficulty. \textit{Journal of Experimental Social Psychology, 42}, 290-302.


Margu\�{c}, J., van Dijk, W.W., \& Gallucci, M. (2006). De invloed van angst en boosheid op informatie zoekgedrag. \textit{Jaarboek Sociale Psychologie}, in press.

Nouwen, A., Cloutier, C., Kappas, A., Warbrick, T., \& Sheffield, D. (2006). The effects of focusing and distraction on cold-pressor induced pain in chronic back pain patients and controls. \textit{Journal of Pain, 7}, 62-71.

Olk, B., Chang, E., Kingstone, A., \& Ro, T. (2006). Modulation of antisaccades by transcranial magnetic stimulation of the human frontal eye field. \textit{Cerebral Cortex, 16}, 76-82.

Olk, B., \& Harvey, M. (2006). Characterizing exploration behavior in spatial neglect: Omissions and repetitive search. \textit{Brain Research, 1118}, 106-115.

Olk, B., \& Kingstone A. (2006). Disentangling confounded measures of attention: Orienting in patients with right hemisphere lesions. \textit{Journal of Cognitive Neuroscience}, C 26 Suppl.

Olk, B., Hildebrandt, H., \& Kingstone, A. (2006). Volitional orienting in patients with right-hemisphere lesions. \textit{Zeitschrift f\"{u}r Neuropsychologie}, 96.

Olk. B. (2006). Interaction of reflexive and volitional orienting. \textit{Perception, 35}, 161, Suppl.

Rach, S., \& Diederich, A. (2006). Visual-tactile integration: Does stimulus duration influence the relative amount of response enhancement? \textit{Experimental Brain Research, 173}, 514-520.

Shields, S.A., \& Kappas, A. (2006). Magda B. Arnold's contributions to emotions research. \textit{Cognition and Emotion, 20}, 898-901.

Shimozaki, S., Kingstone, A., Olk, B., Stowe, R., \& Eckstein, M. (2006). Classification images of two right hemisphere patients: A window into the attentional mechanisms of spatial neglect. \textit{Brain Research, 1080}, 26-52.

Werth, L., \& F\"{o}rster, J. (2006). How regulatory focus influences consumer behavior. \textit{European Journal of Social Psychology, 36}, 1-19. 

Werth, L., \&  F\"{o}rster, J. (2006). Regulatorischer Fokus: Ein \"{u}berblick. \textit{Zeitschrift f\"{u}r Sozialpsychologie}, in press.

Werth, L., \& F\"{o}rster, J. (2006). The effects of regulatory focus on braking speed. \textit{Journal of Applied Social Psychology}, in press.

Werth, L., F\"{o}rster, J., \& Markel, P. (2006). The role of implicit theories on leadership evaluation. \textit{European Journal of Work and Organizational Psychology, 15}, 1-26.
