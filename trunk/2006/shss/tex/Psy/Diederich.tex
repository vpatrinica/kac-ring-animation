\subsection{Prof. Dr. Adele Diederich}


\textbf{Main Research Interests}\\[-0.25cm]
\begin{enumerate}
\item[$\bullet$]	Perception: Multisensory Interaction
\item[$\bullet$]	Decision Theory: Dynamic Models for Decision Making under Uncertainty
\item[$\bullet$]	Medical Decision Making: Conjoint Analysis (Random Utility Models) in Health Care
\item[$\bullet$]	Psychophysics: Fechnerian Scaling
\item[$\bullet$]	Mathematical Modeling of Cognitive Processes
\end{enumerate}


\vspace{0.6cm}
\textbf{Research Activities}\\[-0.25cm]

Adele Diederich has two main research foci: Perception and Decision-Making. Perceptual processes are typically investigated within one modality only. Triggered by recent physiological findings multisensory research has caught interest of cognitive psychologists as well. Two DFG grants allowed Adele Diederich to continue to study the interaction of different modalities in space and in time with experiments using elementary stimuli and measuring manual response times and saccadic eye movements. Quantitative psychological models were developed for the behavioral results taking into account results from neurophysiological studies of the lower brain structures. During the summer an extensive study with elderly from the Bremen area was conducted. The theory of Fechnerian Scaling makes available a principled way of imposing a metric representing pairwise dissimilarities on any discrete set of stimuli, given the probability with which they are discriminated from each other. Adele Diederich utilizes this approach in testing whether bimodal objects possess emergent features that cannot be predicted from their unimodal (visual and auditory) representations. In the field of decision-making processes Multi-Attribute Decision Field Theory (MDFT) is a theory developed for multiattribute decision problems which take into account both the dynamic and stochastic nature of decision making. Its goal is to describe the motivational and cognitive mechanisms that guide the deliberation process in decisions under uncertainty. Ongoing experiments investigate the influence of payoffs in perception decision tasks. Competing models are included in analyzing the data. Researchers from different fields (Economics, Law, Medicine, Philosophy, and Psychology) initiated a research group on \textit{Priorisierung in der Medizin} (described in Chapter 7 of this research report). Adele Diederich is the coordinator of this research group, meanwhile suggested for funding by evaluators of Deutsche Forschungsgemeinschaft. A book on \textit{Cognitive Modeling} together with J.R. Busemeyer, Indiana University, Bloomington, USA, is in progress. The book is targeted to graduate students in cognitive sciences. It is planned to be finished in 2007.
\vspace{0.6cm}


\textbf{Funded Projects}\\[-0.25cm]
\begin{enumerate}
\item[$\bullet$]   Experimentelle und theoretische Untersuchung r�umlicher und zeitlicher Regeln der multisensorischen Integration, funded by Deutsche Forschungsgemeinschaft
\item[$\bullet$]   Regular minimality principle in relation to decision making and categorization, funded by AFOSR (US Air Force)
\item[$\bullet$]   International Graduate School for Neurosensory Systems and Science funded by Deutsche Forschungsgemeinschaft
\item[$\bullet$]   Sokrates/Erasmus Intensive Program on "Mathematical and Computational Models in the Psychological Sciences (MCMPS)" funded by the European Commission
\end{enumerate}



\vspace{0.6cm}
\textbf{Other Professional Activities}\\[-0.25cm]
\begin{enumerate}
\item[$\bullet$] Editorial Board: Journal of Mathematical Psychology
\item[$\bullet$] Executive Board:  Society of Mathematical Psychology
\item[$\bullet$] Member of the NSF Panel Methodology, Measurement, and Statistics (MMS) 
\item[$\bullet$] Member of the research center Neurosensorik Oldenburg
\item[$\bullet$] Numerous reviews for international journals (e.g., Psychological Review, Journal of Experimental Psychology: General; Psychonomic Bulletin and Review; Journal of Mathematical Psychology; Cognitive Science, etc.) 
\item[$\bullet$] Numerous reviews for funding agencies (DFG, NSF, NWO)
\item[$\bullet$] Coordination (Sprecherin) of the Forschergruppe "Priorisierung in der Medizin"
\item[$\bullet$] Committee work 
	\begin{enumerate}
		\item[-]	Member of the University Promotion Review Committee for the School of Humanities and Social Sciences
		\item[-]	Member of the University Promotion Review Committee for the School of Engineering and Science 
		\item[-]	Member of the  Scientific Computation Committee of CLAMV
		\item[-]	Numerous ad hoc committees
		\item[-]	New Investigator Award of the Society for Mathematical Psychology
	\end{enumerate}
\item[$\bullet$] Mentor in the mentoring program of the Universit\"{a}t Bremen
\item[$\bullet$] Contact person (Vertrauensdozentin) of Friedrich Naumann Stiftung
\item[$\bullet$] Evaluation and selection of fellows for the Studienstiftung des deutschen Volkes
\end{enumerate}





\vspace{0.6cm}
\textbf{PhD-Students}\\[-0.25cm]

Daniela Bockhorst\newline
\textit{Entscheiden unter Unsicherheit: Alters- und geschlechtsspezifische Einfl�sse in alltagsnahen Wahlsituationen}\\[-0.15cm]

Stefan Rach\newline
\textit{A Two-stage Multichannel Diffusion Model of Multisensory Integration}\newline
funded by Deutsche Forschungsgemeinschaft\\[-0.15cm]

Rike Steenken\newline
\textit{Psychophysical Investigation of Unconscious Cross-Modal Priming}\newline
funded by Deutsche Forschungsgemeinschaft

\vspace{0.6cm}
\textbf{Research Personnel}\\[-0.25cm]

Dr. Annette Schomburg\newline 
PostDoc funded by Deutsche Forschungsgemeinschaft
