\subsection{Prof. Dr. Jens F\"{o}rster}


\textbf{Main Research Interests}\\[-0.25cm]
\begin{enumerate}
\item[$\bullet$]	Social Cognition (Judgment, Perception, Processing Styles, Memory)
\item[$\bullet$]	Self Regulation (Motivation, Self Control, Automatic Goal Pursuit) 
\item[$\bullet$]	Embodiment (Meaning, Expression Patterns, Attitudes, Emotion) 
\item[$\bullet$]	Prejudice (Stereotype Threat, Discrimination, Thought Suppression)
\end{enumerate}


\vspace{0.6cm}
\textbf{Research Activities}\\[-0.25cm]

Jens F\"{o}rster's research focuses on basic principles of information processing and motivation that underlie diverse social phenomena. In 2006, more than 200 experiments have been conducted with the following results: \textit{Global processing styles} (looking at the forest) enhance abstract thinking, creativity, similarity search, abstraction of meaning, and reduce prejudice, whereas {\itshape local processing styles} (looking at the trees) enhance concrete and analytic thinking, dissimilarity search, and detail perception. Global vs. local processing can be elicited by {\itshape approach/avoidance motivation, regulatory focus, positive/negative mood} and {\itshape right/left hemisphere activation}, and studies show that they have similar effects. To illustrate, a study found that if participants compared their athletic skills to high (Michael Schumacher) or low standards (Helmut Kohl), they assimilate their judgments toward the standard when in a good mood (e.g., think that they are more athletic after comparison with Schumacher and less athletic after comparison with Kohl) - whereas contrast effects occurred when in a bad mood (e.g., think they are less athletic after comparison with high and more athletic after comparison with low standards). Three papers were submitted. Further studies examined the impact of {\itshape novelty} on thinking, showing that framing events as novel leads to activation of broader categories than framing events as familiar. A paper is under revision. A cognitive model on {\itshape love} vs. {\itshape sex} was designed. Studies found that unconscious reminders of love make people more creative, enhance face recognition, lead to global processing and Halos, whereas unconscious priming of sex enhances analytic thinking, recognition of verbal information, local processing and reduces Halos. Three papers have been submitted. Moreover, Jens F\"{o}rster finished reviews on unconscious self regulation (one paper is under revision; three book chapters are in press), and two book chapters on embodiment and approach/avoidance motivation were submitted. He is currently editing a book on Social Cognition and a popular book on prejudice has been accepted by dva/Random House and will appear March 2007. 

\vspace{0.6cm}


\textbf{Funded Projects}\\[-0.25cm]
\begin{enumerate}
\item[$\bullet$]   "Perseveranzeffekte bei Kreativit\"{a}t und analytischem Denken", funded by Deutsche Forschungsgemeinschaft 
\item[$\bullet$]   "Motivational Influences on Construct Accessibility", funded by Deutsche Forschungsgemeinschaft
\end{enumerate}



\vspace{0.6cm}
\textbf{Other Professional Activities}\\[-0.25cm]
\begin{enumerate}
\item[$\bullet$] Ad hoc reviewer for the following scientific journals: \textit{Acta Psychologica, Anxiety, Stress, and Coping, Behaviour Research and Therapy, British Journal of Social Psychology, Cognition and Emotion, Cortex, Emotion, European Journal of Social Psychology, Experimental Psychology, Journal of Experimental Psychology: Learning, Memory, \& Cognition, Journal of Experimental Psychology: Human Perception \& Performance, Journal of Experimental Social Psychology, Journal of Personality, Journal of Personality and Social Psychology, Learning and Individual Differences, Personality and Social Psychology Bulletin, Psychological Science, Quarterly Journal of Experimental Psychology, Schweizerische Zeitschrift f�r Psychologie, Self and Identity, Social Cognition, Zeitschrift f�r Psychologie, Zeitschrift f�r Sozialpsychologie, Zeitschrift f�r Experimentelle Psychologie.} 
\item[$\bullet$] Associate editor of \textit{Social Cognition}
\item[$\bullet$] Member of the editorial board of the \textit{European Journal of Social Psychology}
\item[$\bullet$] Member of the editorial board of the \textit{Journal of Personality and Social Psychology}
\item[$\bullet$] Member of the editorial board of \textit{Self and Identity}
\item[$\bullet$] Reviewer for the following organisations: Deutsche Forschungsgemeinschaft (DFG) single projects, ethics guidelines for research, graduate colleges and research groups, Deutsche Gesellschaft f�r Psychologie (Ethics), Deutscher Wissenschaftsrat, Evangelische Studienstiftung Villigst, German Israeli Science Foundation (GIF), Fonds voor Wetenschappelijk Onderzoek - Vlaanderen (FWO), National Science Foundation (NSF), Nederlandse Organisatsie voor Wetenschappelijk Onderzook (NWO), Social Sciences and Humanities Research Council of Canada (SSHRCC/CRSH), Studienstiftung des Deutschen Volkes, Rotary Club, Schweizerischer Nationalfonds zur F�rderung der wissenschaftlichen Forschung (FNSNF), Ethik-Kommission der Deutschen Gesellschaft f�r Psychologie.
\item[$\bullet$] Member of the Society for Experimental Social Psychology
\end{enumerate}





\vspace{0.6cm}
\textbf{PhD-Students}\\[-0.25cm]

Laura Dannenberg\newline
\textit{Working title: Symbolic Goal Fulfillment}\\[-0.15cm]

Markus Denzler\newline
\textit{Cathartic Effects Revisited: The Impact of Goals on Aggressive Thoughts and Aggressive Behavior}\newline
Defense: May 2006}\\[-0.15cm]

Stefanie Kuschel\newline
\textit{Going Beyond Information Given: How Approach Versus Avoidance Motivational Cues Influence Encoding of Meaning and Details}\newline
Defense: May 2006}\\[-0.15cm]

Janina Margu\'{c}\newline
\textit{Working title: Strength of Engagement and Creativity}\\[-0.15cm]

Amina \"{O}zelsel\newline
\textit{Me, Myself - or We? A Goal-Systems Account of Gender Differences Found Following Independence Priming}\newline
Defense: May 2006}\\[-0.15cm]

Katrin Schimmel\newline
\textit{Self Regulatory Mechanisms in the Evaluation of Conventional and Unconventional Arts - Regulatory Focus and Distance Influences on Attitudes}\newline
Defense: May 2006}\\[-0.15cm]



\vspace{0.6cm}
\textbf{Research Personnel}\\[-0.25cm]

Janina Margu\'{c}\newline
Research Associate funded by Deutsche Forschungsgemeinschaft\\[-0.15cm]

Laura Dannenberg \newline
Research Associate funded by Deutsche Forschungsgemeinschaft\\[-0.15cm]

Markus Denzler \newline
PostDoc funded by Deutsche Forschungsgemeinschaft\\[-0.15cm]


