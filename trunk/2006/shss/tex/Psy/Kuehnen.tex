\subsection{Prof. Dr. Ulrich K\"{u}hnen }


\textbf{Main Research Interests}\\[-0.25cm]
\begin{enumerate}
\item[$\bullet$]	Self-concept and Social Information Processing 
\item[$\bullet$]	Cross-cultural Variations in Thinking, Feeling and Action
\item[$\bullet$]	Intercultural Understanding and Competence
\item[$\bullet$]	The Role of Cognitive Feelings in Judgment Formation
\end{enumerate}


\vspace{0.6cm}
\textbf{Research Activities}\\[-0.25cm]

Individuals can construe their identity either as an autonomous and independent entity or as being related to other people, thus stressing the interdependence with them. The differences in the way the self is defined have profound influences for the person's thinking, feeling and action. Understanding the exact mechanisms by which different self-views affect information processing is my most important research question. This research is closely related to cross-cultural studies, since it is a well-established finding that members of different cultural groups construe their identity primarily in either independent or interdependent terms. What is more, the individual construal of the self has been found to mediate the impact of culture on cognition in various domains. In 2006, Ulrich K\"{u}hnen extended this research in the following ways:
\begin{enumerate}
	\item[$\bullet$]While his previous research had focused primarily on the purely cognitive consequences of different self-views, most recently he started taking motivational implications of self-construals into account. Are independent and interdependent self-views related to approach and avoidance motivational states. What are the consequences of these states for independent or interdependent individuals? 
	\item[$\bullet$]Do the different self-views coincide with different agency concept and how do these differences attributions of observed and one's own behavior? Is the notion of free will more closely connected to independent rather than interdependent self-views?
\end{enumerate}
During the last year Ulrich K\"{u}hnen became more and more interested in applied aspects of Intercultural Psychology. This interest was partly fueled by the fact that he and his colleagues believe the multicultural IUB context offers great opportunities to study intercultural communication. Together they developed an initiative to improve intercultural understanding and competence on the IUB campus and beyond. This initiative includes intercultural workshops and trainings for IUB members, a new BA program, consultancy offers to external companies and accompanying applied research.

\vspace{0.6cm}



\textbf{Organization of Scientific Conferences}\\[-0.25cm]
\begin{enumerate}
\item[$\bullet$]	September 2006\newline
45. Kongress der Deutschen Gesellschaft f�r Psychologie\newline
N�rnberg\newline
Member of the Scientific Board  
\item[$\bullet$] September 2006\newline
International University Bremen \newline
Symposium on "Intercultural Understanding and Competence"\newline
funded by IUB
\end{enumerate}



\vspace{0.6cm}
\textbf{Other Professional Activities}\\[-0.25cm]
\begin{enumerate}
\item[$\bullet$] Chairman of the Social Psychology division (Fachgruppensprecher) of the Deutsche Gesellschaft f�r Psychologie
\item[$\bullet$] Member-at-large of the Executive Committees of the International Association for Cross-Cultural Psychology
\newpage
\item[$\bullet$] Together with Prof. Dr. Klaus Boehnke organizer (Head of the Local Organization Committee) of the $19^{\rm th}$ Congress of the International Association for Cross-Cultural Psychology in 2008 at IUB
\item[$\bullet$] Member of the Editorial Board of the journal \textit{Self \& Identity}
\item[$\bullet$] Ad hoc reviewer for the DFG, as well as for several scientific journals, including Personality and Social Psychology Bulletin; Psychological Science; Self and Identity
\end{enumerate}
