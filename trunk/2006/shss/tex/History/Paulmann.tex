\subsection[Prof. Dr. Johannes Paulmann] {Prof. Dr. Johannes Paulmann {\normalfont\newline Helmut Schmidt Chair of International History} {\normalfont\normalsize\newline (Left IUB in August 2006)} }


\textbf{Main Research Interests}\\[-0.25cm]
\begin{enumerate}
\item[$\bullet$]	International History
\item[$\bullet$]	European and German History
\item[$\bullet$]	German Cultural Diplomacy
\item[$\bullet$]	Environmental History
\end{enumerate}


\vspace{0.6cm}
\textbf{Research Activities}\\[-0.25cm]

The Helmut Schmidt Chair of International History was established during 2003 with funds from the ZEIT-Stiftung Ebelin und Gerd Bucerius. The chair holder Johannes Paulmann's research and teaching covers the historical development of today's connected world from the $18^{\rm th}$ century to 1990/91 based on a broad conception of International History. While also covering the classical field of international politics the emphasis is on topics of transnational relations between societies and cultures. Cross-border processes and structures are studied analysing historical precursors of the contemporary global entanglements. The scholarly aim is innovative in the context of recent scholarship and, on account of its topicality, attracts interest from a wider public. 

\vspace{0.6cm}


\textbf{Other Professional Activities}\\[-0.25cm]
\begin{enumerate}
\item[$\bullet$] Member of the Editorial Board and referee of \textit{Contemporary European History} (Cambridge University Press)
\item[$\bullet$] Referee for several foundations 
\end{enumerate}



\vspace{0.6cm}
\textbf{PhD-Students}\\[-0.25cm]

Daniel Leese\newline
\textit{Mao Cult: Rhetoric and Ritual during China's Cultural Revolution}\newline
funded by Studienstiftung des Deutschen Volkes\newline
Defense: December 2006\\[-0.15cm]


Bernhard Gi�ibl\newline
\textit{Germany's Imperial Game: Hunting, Wildlife Preservation, and Representations of Africa in Imperial Germany, 1880-1945}\newline
funded by Cusanuswerk and DHI London\\[-0.15cm]

Claus Ludl\newline
\textit{Die Geschichte der vergleichenden Verhaltensforschung in Deutschland (1945-1980): Wissenstransfer, Wissenschaftspopularisierung und internationale Beziehungen im Grenzgebiet von Psychologie und Biologie}\newline 
supplementary funding by DHI London\\[-0.15cm]
