%the function for the book publication could be as follows [it was not implemented here]
%1=author;2=year;3=title;4=publisher + comments
%\newcommand{\book}[4]{#1 (#2). \textit{#3}. #4}

\subsection{Publications}




	\paragraph{Books} \textit{ }
	
\bigskip	


Dooley, B. (Ed.) (2006). \textit{Energy and culture: Perspectives on the power to work}. Burlington, VT: Ashgate. \\ \\ 

Fischer-Tin\'{e}, H. (2006). \textit{Ausnahmezustand}. Novel by Nirmal Verma, translated from Hindi by H. Fischer-Tin\'{e}, \& H. Bauhaus-L\"{o}tzke. Heidelberg: Draupadi.\\ \\ 

Frey, M. (2006). \textit{Dekolonisierung in S\"{u}dostasien. Die Vereinigten Staaten und die Aufl\"{o}sung der europ\"{a}ischen Kolonialreiche}. M\"{u}nchen: Oldenbourg.\\ \\ 

Leutner, M., Spakowski, N., \& Yu, C.M. (Eds.) (2006). \textit{Gonghe shidai de Zhongguo fun\"{u} (Women in China. The Republican Period in Historical Perspective)}. Taipei: Rive Gauche Publishing, in press.\\ \\ 

Paulmann, J. (2006). \textit{Die Haltung der Zur\"{u}ckhaltung: Ausw\"{a}rtige Selbstdarstellungen nach 1945 und die Suche nach einem erneuerten Selbst-verst\"{a}ndnis in der Bundesrepublik} (= Schriftenreihe der Wilhelm und Helene Kaisen-Stiftung), Bremen: Kaisen-Stiftung.\\ \\ 

Spakowski, N., \& Milwertz, C. (Eds.) (2006). \textit{Women and gender in Chinese Studies. Special issue of Chinese History and Society/Berliner China-Hefte}, 29.\\ \\ 



\paragraph{Articles \& Chapters}\textit{ }

\bigskip


Dooley, B. (2006). Art and information brokerage in the career of Don Giovanni de' Medici. In H. Cools, M. Keblusek, \& B. Noldus (Eds.), \textit{Your Humble Servant} (pp. 81-96). Hilversum: Verloren.\\ \\ 

Fischer-Tin\'{e}, H. (2006). Global civil society and the forces of empire. The Salvation Army, British imperialism and the "pre-history" of NGOs. In S. Conrad, \& D. Sachsenmaier (Eds.), \textit{Conceptions of world order: Global historical approaches} (pp. 25-57). New York: Palgrave, in press.\\ \\ 

Fischer-Tin\'{e}, H. (2006). Stadt der Pal\"{a}ste? - Europ\"{a}ische Lebenswelten im kolonialen Kalkutta. In R. Ahuja, \& C. Brosius (Eds.), \textit{Megast\"{a}dte in Indien: Mumbai, Delhi, Calcutta} (pp. 241-256). Heidelberg: Draupadi. \\ \\ 

Fischer-Tin\'{e}, H. (2006). "Deep Occidentalism"? - Europa und der Westen in der Wahrnehmung hinduistischer Intellektueller und Reformer (ca. 1890-1930). \textit{Journal of Modern European History, 4}(2), 171-203.\\ \\ 

Fischer-Tin\'{e}, H. (2006). From \textit{Brahmacharya} to conscious race culture: Indian nationalism, Hindu tradition and Victorian discourses of science. In C. Bates (Ed.), \textit{Beyond representation. The construction of identity in colonial India }(pp. 230-59). New Delhi: Oxford University Press.\\ \\ 

Fischer-Tin\'{e}, H. (2006). Ananda Marga. In C. Auffahrt, H.-G. Kippenberg, \& A. Michaels (Eds.), \textit{W\"{o}rterbuch der Religionen }(p. 32). Stuttgart: Kr\"{o}ner.\\ \\ 

Fischer-Tin\'{e}, H. (2006). Arya Samaj. In C. Auffahrt, H.-G. Kippenberg, \& A. Michaels (Eds.), \textit{W\"{o}rterbuch der Religionen }(p. 45). Stuttgart: Kr\"{o}ner.\\ \\ 

Fischer-Tin\'{e}, H. (2006). Brahmo Samaj. In C. Auffahrt, H.-G. Kippenberg, \& A. Michaels (Eds.), \textit{W\"{o}rterbuch der Religionen }(p. 80). Stuttgart: Kr\"{o}ner.\\ \\ 

Fischer-Tin\'{e}, H. (2006). Dayanand Sarasvati. In C. Auffahrt, H.-G. Kippenberg, \& A. Michaels (Eds.), \textit{W\"{o}rterbuch der Religionen }(p. 103). Stuttgart: Kr\"{o}ner.\\ \\ 

Fischer-Tin\'{e}, H. (2006). Keshab Chandra Sen. In C. Auffahrt, H.-G. Kippenberg, \& A. Michaels (Eds.), \textit{W\"{o}rterbuch der Religionen} (p. 280). Stuttgart: Kr\"{o}ner.\\ \\ 

Fischer-Tin\'{e}, H. (2006). Krishnamurti. In C. Auffahrt, H.-G. Kippenberg, \& A. Michaels (Eds.), \textit{W\"{o}rterbuch der Religionen} (p. 296). Stuttgart: Kr\"{o}ner.\\ \\ 

Fischer-Tin\'{e}, H. (2006). Maharishi Mahesh Yogi. In C. Auffahrt, H.-G. Kippenberg, \& A. Michaels (Eds.), \textit{W\"{o}rterbuch der Religionen} (p. 318). Stuttgart: Kr\"{o}ner.\\ \\ 

Fischer-Tin\'{e}, H. (2006). Neohinduismus. In C. Auffahrt, H.-G. Kippenberg, \& A. Michaels (Eds.), \textit{W\"{o}rterbuch der Religionen} (pp. 370-371). Stuttgart: Kr\"{o}ner.\\ \\ 

Fischer-Tin\'{e}, H. (2006). Radhakrishnan. In C. Auffahrt, H.-G. Kippenberg, \& A. Michaels (Eds.), \textit{W\"{o}rterbuch der Religionen} (p. 419). Stuttgart: Kr\"{o}ner.\\ \\ 

Fischer-Tin\'{e}, H. (2006). Rajneesh. In C. Auffahrt, H.-G. Kippenberg, \& A. Michaels (Eds.), \textit{W\"{o}rterbuch der Religionen} (p. 419). Stuttgart: Kr\"{o}ner.\\ \\ 

Fischer-Tin\'{e}, H. (2006). Ramana Maharshi. In C. Auffahrt, H.-G. Kippenberg, \& A. Michaels (Eds.), \textit{W\"{o}rterbuch der Religionen} (p. 420). Stuttgart: Kr\"{o}ner.\\ \\ 

Fischer-Tin\'{e}, H. (2006). Sri Aurobindo. In C. Auffahrt, H.-G. Kippenberg, \& A. Michaels (Eds.), \textit{W\"{o}rterbuch der Religionen} (p. 496). Stuttgart: Kr\"{o}ner.\\ \\ 

Fischer-Tin\'{e}, H. (2006). Sundar Singh. In C. Auffahrt, H.-G. Kippenberg, \& A. Michaels (Eds.), \textit{W\"{o}rterbuch der Religionen} (p. 504). Stuttgart: Kr\"{o}ner.\\ \\ 

Fischer-Tin\'{e}, H. (2006). Tagore (Debendranath). In C. Auffahrt, H.-G. Kippenberg, \& A. Michaels (Eds.), \textit{W\"{o}rterbuch der Religionen} (p. 509). Stuttgart: Kr\"{o}ner.\\ \\ 

Fischer-Tin\'{e}, H. (2006). Tagore (Rabindranath). In C. Auffahrt, H.-G. Kippenberg, \& A. Michaels (Eds.), \textit{W\"{o}rterbuch der Religionen} (p. 509). Stuttgart: Kr\"{o}ner.\\ \\ 

Fischer-Tin\'{e}, H. (2006). Vivekananda. In C. Auffahrt, H.-G. Kippenberg, \& A. Michaels (Eds.), \textit{W\"{o}rterbuch der Religionen} (p. 558). Stuttgart: Kr\"{o}ner.\\ \\ 

Frey, M. (2006). Die Vereinigten Staaten und die Dritte Welt w\"{a}hrend des Kalten Krieges. In B. Greiner, \& D. Walther (Eds.), \textit{Hei�e Kriege im Kalten Krieg} (pp. 27-56). Hamburg: Hamburger Edition. \\ \\ 

Frey, M. (2006). Zivilgesellschaft in historischer Perspektive: Die Nie-derlande und Deutschland im 19. Jahrhundert. \textit{Zentrum f\"{u}r Niederlande-Stu-dien Jahrbuch f\"{u}r 2005, 16}, 11-32.\\ \\ 

Frey, M. (2006). Mao Zedong. Nationalist und Despot. In S. F\"{o}rster, \& M. P\"{o}hlmann (Eds.), \textit{Kriegsherren der Weltgeschichte. 22 Historische Por-traits} (pp. 357-372). M\"{u}nchen: C.H. Beck.\\ \\ 

Spakowski, N., \& Milwertz, C. (2006). Introduction: Women and gender in Chinese Studies and the Women and Gender in Chinese Studies Network (WAGNet). In N. Spakowski, \& C. Milwertz (Eds.), \textit{Women and Gender in Chinese Studies}. Special issue of \textit{Chinese History and Society/Berliner China-Hefte, 29}, 3-4.\\ \\ 

Spiekermann, U. (2006). From neighbour to consumer. The transformation of retailer-consumer relationships in twentieth-century Ger-many. In F. Trentmann (Ed.), \textit{The making of the consumer. Knowledge, power and identity in the modern world} (pp. 147-174). Oxford: Berg.\\ \\ 

Spiekermann, U. (2006). Warum scheitert Ern\"{a}hrungskommunikation? In Aid e.V. (Ed.), \textit{Ern\"{a}hrungskommunikation. Neue Wege - neue Chancen?}. (pp. 11-20). Bonn: Aid.\\ \\ 

Spiekermann, U. (2006). Warum scheitert die Ern\"{a}hrungskommunika-tion? Eine Antwort aus kulturwissenschaftlicher Perspektive. In E. Barl\"{o}sius, \& R. Rehaag (Ed.), \textit{Anforderungen an eine \"{o}ffentliche Ern\"{a}hrungskom-munikation} (pp. 39-51). Berlin: Wissenschaftszentrum.\\ \\ 

Spiekermann, U. (2006). Warenwelten. Die Normierung der Nah-rungsmittel in Deutschland 1880-1930. In R.-E. Mohrmann (Ed.), \textit{Essen und Trinken in der Moderne} (pp. 99-124). M\"{u}nster: Waxmann.\\ \\ 

Spiekermann, U. (2006). Brown bread for victory: German and British wholemeal politics in the interwar period. In F. Trentmann, \& F. Just (Ed.), \textit{Food and conflict in Europe in the age of the two World Wars} (pp. 143-171). Basingstoke: Palgrave Macmillan.\\ \\ 

