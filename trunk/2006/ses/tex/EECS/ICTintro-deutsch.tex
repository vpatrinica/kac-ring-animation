\documentclass[11pt,a4paper]{article}
\usepackage{german}
\usepackage{times,a4wide}
\begin{document}
{\Large\bf{Informations- und Kommunikationstechnologien}}\\[1ex]

Informations- und Kommunikationstechnologien haben bereits einen signifikanten und immer noch
wachsenden Einfluss auf unseren Alltag, unsere Gesellschaft und die Umwelt. Beispiele
daf"ur sind das World Wide Web, allgegenw"artige Computer, drahtlose Multimediager"ate,
Handys, Roboter, fahrerlose Transportsysteme und Computermodelle, die zur
Vorhersage komplexer Systeme wie dem globalen Klima verwendet werden. 

Die enorme Integrationsdichte von mikroelektronischen Schaltkreisen hat enorme
Fortschritte in den verschiedensten Sektoren erm"oglicht, und zu hochprofitablen
Technologien, Industrien, und Mehrwertdiensten gef"uhrt. Neben der Hardwareentwicklung tragen nat"urlich auch enorme Fortschritte in Bezug auf Software und auch grundlegende Aspekte wie die Entwicklung von neuen Algorithmen zu diese enorme Entwicklung bei. 

Das 21. Jahrhundert wird uns Technologien bringen, die Dinge erm"oglichen, die bisher
Menschen vorbehalten waren: Maschinen werden f"uhlen, handeln, sprechen, zuh"oren,
entscheiden und manchmal sogar verstehen k"onnen. Wir werden Autos haben, die miteinander
verhandeln, um den Verkehr zu organisieren, Treibstoff zu sparen und die Umwelt zu
entlasten. Unsere T-Shirts werden ihre eigene Internetadresse haben und vielleicht drahtlos
die Waschmaschine benachrichtigen, wann sie gewaschen werden m"ussen. Zus"atzlich zu ihrer
etablierten Rolle als programmierbare Werkzeuge werden Maschinen zunehmend in
Bereichen eingesetzt, wo Autonomie und Intelligenz notwendig sind. Roboter arbeiten immer mehr
unter Bedingungen, wo sie nicht st"andig von Menschen beaufsichtigt
werden k"onnen. Dabei m"ussen sie sich an wechselnde Gegebenheiten sowie an
Situationen anpassen, die ihre Entwickler nicht haben voraussehen k"onnen.

Die IUB beteiligt sich an diesem stark wachsenden, herausfordernden interdisziplin"aren
Unterfangen. Dabei liegt der Fokus auf den Gebieten
\begin{description}
    \item[Kommunikationssysteme und Netzwerke] Um jederzeit und "uberall auf jede
     Information zugreifen oder sie verteilen zu k"onnen, erforschen wir drahtlose
  Informations"ubertragung und Netzwerkdesign, Netzwerk- und "Ubertragungsprotokolle,
  Netzwerkinteroperabilit"at und Informationssicherheit.
\item[Robotik und Automatisierungstechnik] Um Autonomie und Adaptivit"at von Maschinen zu
  unterst"utzen, bauen wie intelligente Roboter und erforschen Methoden zur konstruktiven Modellierung von Kognition.
\item[Informations- und Wissensmanagementsysteme] Um den Benutzer mit ma"sgenschneiderten
  Informationen versorgen zu k"onnen, treiben wir die Konvergenz von Datenbanken, dem Web
  und verteilten Systemen voran, untersuchen wir semantische Repr"asentationsformate f"ur
  eScience-Anwendungen und optimieren das Management von Rasterdaten.
\item[Mikroelektronik] Um die rasante Leistungssteigerung bei elektronischen Ger"aten in
  den n"achsten Jahrzehnten sicherzustellen, untersuchen wir Grundlagenfragen in den
  Materialwissenschaften und Fertigungsprozessen. Wir brauchen neue Mikro- und
  Nanotechnologien, um Ger"ate weiter zu miniaturisieren und neue Anwendungen
  (z.B. flexible organische Displays) zu erm"oglichen.
\end{description}

\end{document}

%%% Local Variables: 
%%% mode: stex
%%% TeX-master: t
%%% Local IspellParsing: ~tex
%%% End: 

% LocalWords:  Informations World Wide Local mode stex master End  Rescue
% LocalWords:  Challenge

