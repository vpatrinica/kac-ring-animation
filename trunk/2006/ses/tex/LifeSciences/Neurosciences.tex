\shorttitle{Neurosciences}
\subsection{Neurosciences}

Neuroscientists at Jacobs University Bremen focus on the perception and interaction of animals, including primates and humans, with their environment. The structure and function of neurons in animal and primate brains along with theoretical and experimental studies on the function of complex neuronal networks are the main questions driving research in this field. The research projects cover basic sciences and clinical studies, and have several interesting connections to the groups working in the field of Molecular Medicine as described in chapter \ref{MolLifeSc}.\\ 

The landscape of primate brains is determined by anatomical studies and interpreted by employing systems neuroscience. Investigations of retinal fine structures in combination with electrophysiological recordings and behavioral tests allowed deduction of conclusions which were also of relevance for human health. Likewise, the research of Neuroscientists at Jacobs has become relevant for clinical applications in that the influence of visible light and electromagnetic radiation was analyzed experimentally. \textit{Virtual lesions} and experimental damaging of brain structures were performed, and the partial regeneration of the delicate neuronal structures from such damage was studied. The results demonstrated that primate brain structures can indeed be repaired to an unexpected extent. Although regeneration of brain structures remains limited in primates, adult neurogenesis in other vertebrates does take place. Investigations on repair after injury of brain structures revealed the detection of stem cells in the adult brain of bony fish. \\

The research groups in the Neurosciences at Jacobs aim to extend their studies towards the applicability of experimental findings from animals to humans.\\
