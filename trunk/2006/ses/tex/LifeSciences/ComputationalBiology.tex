\shorttitle{Computational Biology}
\subsection{Computational Biology}

Computational Biology at IUB uses numerical simulations, novel data analysis concepts and mathematical modeling to better understand biological data. These computational approaches allow relating information from very different levels of organization: e.g., protein binding and protein networks, genomes and ecology, regulation and patterns.\\

In 2006 the fields of research have been extended towards Systems Biology and now cover all major research areas of Computational Biology. The output of Computational Biology during this year ranges from articles in high-ranking scientific journals, scientific textbooks, software tools to the organization of conferences and summer schools. Progress has been made in analyzing metagenomic data, including flexibility in binding processes of biomolecules and understanding dynamic processes on abstract and real biological networks.\\

Bridging scales is the essence of a system-wide understanding and, at the same time, a source of a wide range of scientific collaborations. The vast methodological knowledge involved is relevant to virtually all parts of the life sciences. Work on Computational Biology is well embedded in the Complex Systems focus of Jacobs University.\\
