
\paragraph{Research Team}
Ben Godde (Professor of Neuroscience and Human Performance), Claudia Voelcker-Rehage (Lecturer, Human Performance), Mireille Trautmann (Postdoctoral Fellow, Neuroscience), Katja Dolge (Research Associate, Neuroscience).\newline \textbf {Former members}  Mikhail Babanin (Research Associate, Neuroscience), Laura Khil (Research Associate, Neuroscience), Johanna Stiehm (Research Associate, Human Performance), Alexander Kowanz (Research Associate, Human Performance).


\subsubsection{Promoting lifelong learning and development through detailed knowledge of underlying mechanisms}

Our research group is motivated by two central questions. First: How can lifelong learning and development be promoted? This means, how can we help people to gain optimal sensory, motor, and cognitive performance in different life phases and environments? Second: What are the neurobiological processes underlying lifelong learning and productive adult development, and how can these processes be modified and improved? For this purpose, our research focuses on identifying mechanisms underlying cortical plasticity and the structure/function relationships between cognitive, sensory, and motor performance and learning. The human brain remains plastic even at older age. However, plastic capacity declines with age. Thus, knowledge about the ability to facilitate plasticity is of high significance, particularly for the elderly learner. 

We address these research questions in several projects with a combination of neuroimaging, neuropsychology, physiology, and movement science. Although we examine human performance and its neurobiological basis during the whole life span, the relationships between sensory, motor and cognitive abilities, and the plastic-adaptive capacities of older adults are of particular interest for our research program. In this context, the research of our group is divided into the following main topics.


\subsubsection{Plastic-Adaptive Capacity of the Brain: Mechanisms, Effects of Age, Possible Interventions }

%\input{312}

Under this topic, we investigate the plastic adaptive capacity of the brain in order to understand basic mechanisms of plasticity and learning, as well as the effects of age. Based on experiments, possible interventions might be developed to improve learning and to interfere with age-related decline. In spite of the tremendous progress in brain imaging during the last two decades, which is now enabling detailed examination of brain activities related to complex cognitive functions, it is still rather difficult to unravel the underlying neuronal mechanisms and the relationships between neuronal activity, brain hemodynamics, and behavior. Thus, our research focuses on perception and motor action as more basic, but nevertheless well established, models of general brain functioning. Fields of research include vibrotactile perceptual learning, force control and force modulation, motor-cognitive dual-task performance, intermanual transfer of learned motor programs, and the sensory-motor coupling accompanying learning.

Within the sensory domain, we are mainly interested in plastic-adaptive competences related to tactile processing of younger and older adults. Using functional MRI and psychophysics, we investigate the cortical topography of tactile perception and the cortical mechanisms of perceptual learning. Within the motor domain, our focus is on upper extremity force control. Control of fine finger forces is an elementary component of movement production in many daily activities (e.g. dressing, eating, opening containers) and its assessment provides insight into movement control and coordination. A decline in force control is common in older adults. We examine age-related differences in force control, the practice // effect on force modulation, and age-related differences in dual-task performance. Besides force control, we are also interested in the intermanual transfer of learned motor programs.
Combining these two lines of research, we are going to focus on the interaction between motor and sensory performances. Sufficient sensorimotor functioning, in particular manual dexterity, is of major importance for independent living at old age. However, most older people suffer from a decline in their tactile and motor abilities. It is assumed that an important mechanism responsible for precise force control is tactile sensitivity. However, contradictory results also exist, and the neural control mechanisms responsible for precise manipulation of grasping forces are largely unknown. 
Using a protocol of high frequency tactile stimulation (tHFS), which has been shown to be very effective in inducing somatosensory cortical plasticity, our aim was: (a) to determine the effect of short-term tHFS on tactile and motor performance and (b) to examine tactile-motor interactions. Seventeen right-handed older adults (66-78 years) were randomly assigned to an experimental and a control group. Participants underwent a pre-test of tactile frequency and spatial discrimination performance, a test of manual dexterity, and a precision grip task with their left hands. They then received 30 minutes of tHFS on the tips of their left index fingers and thumbs, and performed a post-test. The control group received no stimulation. 



%insert figure

\begin{figure}[htp]
 \begin{center}
    \scalebox{0.5}{\includegraphics*[2.2in,4.3in][6.8in,6.8in]{figure1}}
    \caption{Left panel: Relative performance changes for the frequency discrimination task (a, FDT), spatial discrimination task (b, SDT), Purdue pegboard test (c, PBT), and force modulation task (d, FTT, RMSE) for the experimental (EG) and control group (CG) (means and standard errors). An asterisk marks significant performance changes different from zero (* p < .05). Note: Pre-to-post-test performance improvements are indicated by negative values for FDT, SDT, and FTT and by positive values for PBT. Right panel: Increased Beta-Power over the left sensorimotor cortex (marked by arrow) during the precision grip task was found after tHFS but not in the control group.  }\label{fig:fig2}
   \end{center}
\end{figure}

%insert figure

%insert figure

%insert figure


Results indicated an improvement in frequency and spatial discrimination performance in the experimental group but not in the control group (cf. Figure 1). For the precision grip task, however, improved performance, as found for the control group, did not occur for the experimental group. For the manual dexterity task, no effect was found either for the experimental or for the control group. Our data indicate that tHFS might influence tactile performance positively in older adults. Although it still remains open whether improved tactile function is functionally unimportant for a pure grip task as used in the current studies, or whether tHFS-induced cortical changes result in no improvement in the precision grip, our results give further evidence to the notion of an interrelation between sensory and motor performances.


\subsubsection{Effects of Cardiovascular and Motor Fitness on Cognitive 	Performance and Well-Being in Older Adults }

%\input{313}

Physical fitness is assumed to preserve cognitive functioning and psychological well-being in older adults. More and more animal and human studies stress the importance of a physically active lifestyle for successful aging (i.e. delayed cognitive decline, improved physiological and psychological health, etc.). However, the mechanisms, the dose-response relationship, and also the effects of different types of exercise are not well understood. This second area of research, which follows an interdisciplinary view on human performance that comprises motor, neurophysiological and psychological expertise and methods, deals with the role of aerobic and non-aerobic motor exercise on cognitive performance and well-being across the lifespan.

In a one-year longitudinal study, we investigated the effects of different types of physical activity on cognitive performance and well-being in older adults. 116 older men and women (61-79 years; M = 68.74; SD= 3.50) participated in our study. All participants were recruited from a member register of the German health insurance company DAK and were randomly assigned to one of three intervention groups (endurance training, strength training, or relaxation and stretching). Training groups met three times a week for 12 months. Before the start of the training program, every three months during the intervention, as well as 12 months after its conclusion, the participants spent two days in the lab for extended testing of cognitive and motor abilities, emotion regulation, and subjective well-being. In addition, participants were examined with functional MRI in the beginning, in the middle, and at the end of the intervention study, in order to visualize changes in brain processes during performance of cognitive, motor, and emotional tasks.

For a first cross-sectional analysis of the relationship between motor status and cognitive performances we calculated performance indices for primary intellectual abilities like executive control and perceptual speed (measured by a flanker task, a visual search task, a picture identification task, and an n-back task as memory task) as well as, based on an extended battery of motor tests and spirometric measures, motor status in respect to physical (cardiovascular fitness and strength) and motor fitness (balance, coordination, speed, flexibility). Both behavioral and imaging data revealed differential effects of physical and motor fitness on cognitive performance and related brain processes. At the behavioral level, physical fitness was mainly related to executive control processes (measured with the flanker and n-back task), whereas motor fitness showed a significant association with both the executive control and the perceptual speed tasks (measured with the identical picture and visual search task, cf. figure 2). 

Altogether, our behavioral results support the idea that fitness is differently associated with indicators of fluid intelligence such as perceptual speed and executive control. Imaging data for response competition (flanker task) also indicated that a high physical fitness level might be positively related to resources available for executive control processes. In particular, the right IFG has been shown to be important for interference control during the flanker task, and was activated more strongly in the physically fit participants. A high level of motor fitness, on the contrary, seems to require less effort exerted for the inhibition of distracting information and for motor preparation; this might cause more effective processing and integration of visuo-spatial information. This was indicated by less frontal but stronger and focal sensorimotor and parietal activation found for high versus low motor fit participants. Herewith, we were able to add important insights to the topic of brain aging. Based on the high inter-individual variability in age-related changes in cognition, fitness, and brain activation patterns, it seems useful to tailor preventive physical training interventions accordingly. Analysis of the longitudinal data is underway to test the specific cognitive and neurophysiological effects of a motor-fitness training program as compared to a classical cardiovascular training.


\subsubsection{Mobility and Developmental benefit }

%\input{314}

In a third line of research, we are interested in the effects of working life experiences on the cognitive functioning and plastic-adaptive capacity of middle-aged individuals (40-50 years of age). Within the joint project "Mobility and Developmental Benefit" we pursue the question: does job-related mobility have a cumulative effect on the development of cognitive and personality adaptivity (see Mobility project).

\subsubsection{DemoPass Project}

%\input{315}

Finally, the plastic-adaptive capacities of the adult and aging brain are of particular importance for the aging work force. Combining our expertise in the above mentioned fields of research as a part of joint JCLL efforts, we developed the aim to establish measures of older workers' adaptivity in the sensory, motor, and cognitive domains and to relate their adaptive capabilities to the demands of the work place. From the results, we will be able to identify mismatches between demands of the work place and individual preconditions. We assume that these mismatches might be used as predictors of low productivity or employee health problems (see DemoPass project). 

\subsubsection{Collaborators }

%\input{316}

\begin{itemize}
\item University of Bielefeld \\ Department of Sports Science \\ Prof. Dr. Klaus Willimczik.
\item University of Bremen\\ Movement Science and Training, \\Sports Science\\ Dr. Monika Fikus.
\item University of Bremen \\ Interdisciplinary Center for\\ Cognitive Sciences \\ Prof. Dr. Manfred Herrmann.
\item University of Bremen \\ Interdisciplinary Center for\\ Cognitive Sciences \\ Prof. Dr. Manfred Fahle.
\item University of T�bingen, MEG-Center \\ PD Dr. Christoph Braun.
\item Ruhr-University Bochum\\ Institute for Neuroinformatics\\ PD Dr. Hubert Dinse.
\item International School for Advanced Studies, Trieste, Italy\\  Cognitive Neuroscience Sector \\ Prof. Mathew E. Diamond, PhD.
\item Lerner Research Institute, USA\\  Department of Biomedical Engineering,\\ Cleveland Clinic Foundation\\ Jay L. Alberts, PhD.
\item Jacobs University, SES \\ Prof. Dr. Claus Hilgetag.
\item Jacobs University, JCLL \\ Prof. Dr. Ursula M. Staudinger.
\item University of Tulsa, Psychology Department\\ Nonconscious Information \\ Processing Laboratory\\ Pawel Lewicki, PhD.
\item Humboldt University Berlin\\ Department of Movement and\\ Training Science\\ Dr. Henning Budde.
\item University of Hamburg\\ Department of Human Movement\\ Prof. Dr. Volker Lippens.
\end{itemize}


\subsubsection{Other Professional Activities }

%\input{317}

Claudia Voelcker-Rehage is Junior Fellow of the Leopoldina-Acatech-Workgroup "Chancen und Probleme einer alternden Gesellschaft: Die Welt der Arbeit und des lebenslangen Lernens" (since 2005).

\paragraph{Ad-hoc Reviews}
\textbf{Godde:}
Journal of Cognitive Neuroscience, Psychophysiology, Neuroscience and Biobehavioral Reviews, Journal of Neurophysiology, Journal of Neuroscience Methods, The Quarterly Journal of Experimental Psychology.

\textbf{Voelcker-Rehage:}
Deutsche Zeitschrift f�r Sportmedizin, European Review on Aging and Physical Activity, Experimental Brain Research, Medical Science Monitor, Journal of Gerontology: Psychological Science, Psychological Reports Perceptual and Motor Skills, Zeitschrift f�r Sportpsychologie


\subsubsection{Publications }

%\input{318}
