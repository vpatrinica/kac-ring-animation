\paragraph{}
\begin{flushleft}
\textbf{Research Program}
\end{flushleft}

Max Kaase, Founding Dean of the School of Humanities and Social Sciences (SHSS) and Vice President of the International University Bremen (IUB) from  2000-2006, has been invited to become a Wisdom Professor of Political Science at the Jacobs Center on Lifelong Learning and Institutional Development (JCLL) starting September 1, 2007. Since then, he has given a presentation on "The European Social Survey (ESS) - A High-quality Instrument for Comparative Survey Research" in the context of the JCLL Distinguished Lecture Series, and has been involved in teaching in the PhD Program of the JCLL "Productive Adult Development" and at the Bremen International Graduate School of Social Sciences (BIGSSS) with the intent to add perspectives from political science to these programs.

\paragraph{}
Not the least because of his JCLL position and his role as Chair of the Kuratorium of the Max-Planck-Institute for Demographic Research in Rostock, he has increasingly become interested in the political dimension of demographic changes into the direction of an ever-aging population. A core interest here is to analyze to what extent this change (and recently the global financial crisis) will impact the future of the welfare state, especially in its Scandinavian variant (with a changing emphasis on more individual accountability instead of a further extension of welfare politics). This aspect deserves to be studied on the following two dimensions: the extent to which the above-mentioned changes will influence age-related voting and party membership patterns up to the potential emergence of a generation-based new conflict cleavage in contemporary democratic societies, and, in parallel with more general developments with respect to the structure of political participation, the extent to which older cohorts will turn more and more to elite-challenging actions (unconventional political participation) to represent and enforce their interests.

\paragraph{}
In order to study such developments, increasingly comparative micro-studies of political orientations and behavior are necessary. One particularly relevant survey database is the European Social Survey (ESS), which has come into being in the time (1995) since Kaase was a member of the Standing Committee of the Social Sciences (SCSS), and later Vice President of the European Science Foundation (ESF). The ESS, which is conducted every second year in about 30 European countries, is presently in Round Four.

\paragraph{}
The structure of the ESS foresees a 30-minute part with core questions which are regularly repeated, and (usually) two 15-minutes-each rotating modules, which are given to European research groups involved in comparative research without any cost to them on the basis of an open Europe-wide competition. For questions related to aging and to attitudes toward the welfare state, these rotating modules are particularly pertinent; in Round Three there is a module on the life course, and in Round Four there is a module on Experiences and Expressions of Ageism as well as one on  Welfare Attitudes in a Changing Europe. These data open up fascinating avenues for analyzing the described developments with a top-quality, up-to-date comparative database.

\subsubsection{Other Professional Activities}

\begin{flushleft}
\textit{Membership in Advisory Bodies or Academic Associations}
\end{flushleft}

\begin{itemize}

\item Past President and Member of the Executive Board, International Political Science Association (IPSA)  2006-2009

\item Advisory Board to the University Mannheim (Universitaetsrat) 2007 -

\item Chair, Scientific Advisory Board (SAB), European Social Survey 2000 -

\item Chair, Scientific Advisory Board of the Swiss Foundation for Research in the Social Sciences (FORS), 2008 -

\item Chair, Kuratorium des Max-Planck-Instituts fuer Demographische Forschung, Rostock 2004 -

\item Member of the Social Sciences and Humanities (SSH) Panel of the Finnish Research Infrastructure Survey and Roadmap Project of the Finnish Federation of Learned Societies, 2008

\item Member of the Review Panel of the NCCR Democracy for the Swiss National Science Foundation  (Schweizerischer Nationalfonds zur Foerderung der Wissenschaftlichen Forschung  (FNSNF), 2006 -

\item Member of the EUROCORES Review Panel, European Science Foundation (ESF), 2007
Reviewer for the German-Israeli Foundation for Scientific Research and Development (GIF), 2008

\item Member of the Scientific Advisory Board  for the European Commission FP7 Infrastructure Upgrade Project  "European Election Studies (EES)", 2008 -

\item Member of the Monitoring Group for the European Commission FP6 Project intune (Integrated and United? A quest for Citizenship in an "ever closer Europe"), 2006 -

\item Member SAB� for the Continuing Student Survey organized by the Working Group on University Research, University of Konstanz, and funded by the Federal Ministry of Education and� Research� (BMBF), since 1980

\item Member SAB for the Research Program on Social Stratification, Political Institutions and Human Orientations (SPION), directed by Stefan Svallfors , Umea University, Sweden, since 2008.

\end{itemize}

\paragraph{}
\begin{flushleft}
\textit{Reviews 2007/2008}
\end{flushleft}

International Political Science Review, British Journal of Political Science, American Journal of Political Science, European Political Science Review, Oxford University Press.

\paragraph{}
\begin{flushleft}
\textit{Editorial board 2007/2008}
\end{flushleft}

International Political Science Review, British Journal of Political Science, Journal of Media Psychology, Japanese Journal of Political Science, Acta Politica, Portuguese Journal of Social Science

\subsubsection{Publications}

\begin{itemize}
\item Kaase, M. (2008) Retrospect and Prospect. In: Heiner Meulemann (ed.), Social Capital in Europe - Similarity of countries and diversity of people? Multi-level analysis of the European Social Survey 2002, Brill Publishers, Leiden, 313-319. 
\item Kaase, M. (2008) Perspektiven der Forschung zur Politischen Kultur. In: Dieter Gosewinkel, Gunnar Folke Schuppert (Hg.), Politische Kultur im Wandel von Staatlichkeit. WZB-Jahrbuch 2007, edition sigma, Berlin, 387-397.
\item Kaase, M. (2007) The European Social Survey as a measurement model (with Roger Jowell, Rory Fitzgerald and Gillian Eva). In: Roger Jowell, Caroline Roberts, Rory Fitzgerald, Gillian Eva (eds.), Measuring Attitudes Cross-Nationally. Lessons from the European Social Survey, Sage Publications, London et al., 1-31.
\item Kaase, M., Gabriel A. Almond, Sydney Verba (2007) The Civic Culture. Political Attitudes and Democracy in Five Nations. In: Steffen Kailitz (Hrsg.), Schluesselwerke der Politikwissenschaft, Verlag fuer Sozialwissenschaften, Wiesbaden, 4-8.
\item Kaase, M. (2007) Perspectives on Political Participation. In: Russell J. Dalton, Hans-Dieter Klingemann (eds.), The Oxford Handbook of Political Behavior, Oxford University Press, Oxford, 783-796

\end{itemize}