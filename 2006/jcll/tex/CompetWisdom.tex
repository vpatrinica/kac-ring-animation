\index{Kuzmann, Ute}

\paragraph{Research Team}
Ute Kunzmann (Professor), David Richter (Doctoral Fellow).

 Although the psychology of wisdom is a relatively new field, several promising theoretical and operational definitions of wisdom have been developed during the last years (for reviews see Kunzmann, in press; Kunzmann \& Baltes, in press; Kunzmann \& Stange, in press). Our own work has been based on the Berlin wisdom paradigm that defines wisdom as expert knowledge about fundamental problems related to the meaning and conduct of life. In this paradigm, participants think aloud about difficult and uncertain life problems. Trained raters evaluate these think-aloud protocols according to five criteria indicating wisdom. The three core criteria are: (1) value relativism and tolerance, (2) awareness and management of uncertainty and (3) lifespan contextualism. 

 The goal of our research program has been to study the social and emotional dynamics of wisdom-related knowledge in the context of correlational field studies and experimental work. With this goal we intend to produce evidence that highlights the gains that come with the acquisition of wisdom-related knowledge during ontogenesis and especially the difference that this type of knowledge makes in adults' actual social and emotional behavior.

\null
\textbf{Research Highlights 2006}

 During the last year, Ute Kunzmann completed several publications dealing with the motivational, affective and social dynamics of wisdom-related knowledge. As to the social-emotional dynamics of wisdom-related knowledge, her experimental work suggests that people with high levels of wisdom-related knowledge tend to experience greater empathic sadness when being together with another person in need than people with low levels of wisdom-related knowledge. There is also evidence for a link between wisdom-related knowledge and empathic accuracy: wisdom-related knowledge seems to help people perceive other people's emotions accurately. 

 In her field studies Ute Kunzmann and her collaborators have demonstrated that the values and behaviors of people with high levels of wisdom-related knowledge indicate a striving for a good life in the sense of early Greek philosophy. One aspect of a good life in the early Greek tradition refers to the balancing of personal and common interests. A second aspect refers to the preference for personal growth and self-actualization -- even if this preference opposes happiness in a hedonistic and materialistic sense. Together the evidence from Ute Kunzmann and her collaborators strongly supports the general hypothesis that wisdom-related knowledge makes a difference in people's daily life and has important motivational and social functions for one's own and others' development. 

\newpage
\paragraph{Collaborations}
\begin{itemize}
\item Georgia Institute of Technology, Atlanta, USA  \\ Dr. Antje Stange
\item Tufts University, Medford, USA \\ Prof. Robert J. Sternberg, PhD
\end{itemize}

\enlargethispage*{0.2cm}

\begin{bibunit}[apalike]
\nocite{*}
\putbib[profUteKunzmann3]
\end{bibunit}