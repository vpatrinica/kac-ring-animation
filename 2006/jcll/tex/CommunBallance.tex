\index{Schwender, Clemens}

\paragraph{Research Team}
Clemens Schwender (Professor), Siegmar Otto (Doctoral Fellow), Dennis Mocigemba (Postdoctoral Fellow), Aynur Huylu (Dipl. Media Consultant), Claudia Hesping (Dipl. Media Consultant), Hatice Ecirli (Student Assistant), Florin Bora (Student Assistant)

 Project BALANCE: Why do people turn off the TV? This is the question that the BALANCE project tries to answer with regard to sustainability communication on TV. Sustainability communication faces the criticism that it reaches only those, who are informed anyway, and that a great part of society is not reached at all. Viewers turn off, or switch between channels and thereby refuse reception. The BALANCE project tries to change that by developing and implementing the concept of ecotainment: Messages about sustainability are to be combined with positive emotions.

 By applying content analysis (to identify narrative structure, offered messages and symbols for ecology and sustainability) as well as on- and offline questionnaires (to identify emotional impact, memory and attitudes toward the presented messages), we develop concepts for educational materials and inform journalists and production teams about our findings. The ultimate goal of the project is to broaden knowledge about sustainability in the population.

\null
\textbf{Research Highlights 2006}

\textit{The Avoidance of Information}

 When people watch TV, they use the remote control to decide about the information they will accept. So far, research on strategies of switching channels was centered mainly on the context of commercial avoidance. In my working group, we have developed a model that is able to determine reasons for switching channels that captures not only the content level, but also other factors, such as film aesthetics. The goal is to develop different categories to improve knowledge transfer, especially within the informal context of mass communication.

 Since Fall 2006 we finally have access to media ratings on a second-by-second basis. Now we will be able to find the moments (and reasons) when (and why) people switch channels while watching reports that contain knowledge information. The reasons can be clustered in three domains: (a) Context (like position of the episode, the position in relation to commercial breaks, the program of competitive channels), (b) Content (like interview situations, the presentation of big machines and production lines, the three dimensions - social, ecological and economical - of sustainability), and (c) Formal Aspects (like duration, number of cuts per minute). Our preliminary results indicate that an episode is accepted when it fits the viewers expectations in respect to the program. In the context of ``Welt der Wunder'' sustainability is an accepted topic. 

\textit{Investigation of Media Strategies for Audio-Visual Arguments}

 My working group is also investigating a representative sample of TV-Commercials to better understand what strategies media use for making audio-visual arguments. The term ``narrative function'' is able to explain how characters are used. For the first time, the categories age and gender are not used in content analysis to determine the respective demographic representation, but to determine their function as a stereotypical argument. The goal is to learn how commercials succeed in making a convincing audio-visual argument in a very short time.

 The content analysis of 698 TV-spots for products and services but also for social behavior or awareness and for political parties is finished. The spots present more than 5.000 agents.

\textit{Investigation of the Public Debate on Sustainability in TV and Print Media}

 With the support of ``google news'' we are able to check 700 papers daily for the terms ``sustainability'' and ``sustainable development''. Long-term studies were made with the daily paper ``Die Tageszeitung'' and the weeky paper ``Die Zeit'' from 2002 to 2005. The data show that great events like the Johannesburg World summit in 2002 had a verifiable impact on the coverage of sustainable development in Germany.

 Twice a year, we look at a TV Guide to detect all possible shows that may include issues that deal with sustainability. The media survey suggests a bigger potential for public debate on sustainability.

\newpage
\textit{Transfer of results}

 We are currently preparing a series of workshops for journalists as well as for experts in the field of sustainability research. The workshops will be organized by the Adolf-Grimme-Institut (Marl) and the Bundespressekonferenz (Berlin). 


\paragraph{Collaborations}
\begin{itemize}
\item Universit\"at Hohenheim \\ Lehrstuhl Umweltmanagement \\ Prof. Dr. Werner F. Schulz; Martin Kreeb; Volker Diffenhard 
\item Copenhagen Business School \\ nwd Institut \\ Dr. Lucia Reisch 
\item .lichtl Sustainability Communications \\ Martin Lichtl
\item Adolf-Grimme-Institut, Marl \\ Dr. Friedrich Hagedorn 
\end{itemize}

\begin{bibunit}[apalike]
\nocite{*}
\putbib[profClemensSchwender]
\end{bibunit}

\paragraph{Grants}

\begin{itemize}
\item BMBF (PI: C. Schwender). B.A.L.A.N.C.E. Development, Application and Promotion of a Communicational and Trendsetting Concept for a Sustainable Life and Management 2004-2006.
\end{itemize}