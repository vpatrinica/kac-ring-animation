\index{Voelpel, Sven}

\paragraph{Research Team}
Sven Voelpel (Professor), Eden Tekie (Doctoral Fellow).

Energy is the final driving force for the success of an organization that enables wisdom and innovation creation in a strategically aligned and integrated way. Research ranks managers' energy as the most crucial factor for organizational success - higher than integrity, knowledge or expertise. Consequently, energy is important on the individual, group and organizational levels to create and sustain a wise organization. Research in the area of energy entails projects on individual/human energy, group energy, organizational energy, energy efficiency.

Energy, in the organizational context, is the force that transforms ideas into action, and action into outcomes. It is widely recognized as one of the most important drivers of organization-wide action in the pursuit of organizational goals. The major source of organizational energy is people - their commitment, high performance and creativity. In light of the demographic changes and the aging of the workforce, it has accordingly become essential for organizations to address how best to motivate and mobilize the energy force of older employees (i.e., their emotional, cognitive, and behavioral potential) in achieving organizational goals. Research in this area focuses on organizational energy and aging of the workforce, age heterogeneous teams and organizational commitment, leadership and the aging workforce.

Organizational energy reflects the extent to which an organization is able to mobilize its full emotional, cognitive, and behavioral potential in pursuing its goals. People are the main source of organizational energy and, thus, are the most precious resource organizations have. Therefore, knowing how to create effective energy and, above all, leveraging and nurturing it through people management is at the heart of a company's long-term success in Asia. Mobilizing organizational energy becomes especially important when considering the challenges faced by international companies in dealing with Asian culture that is essentially characterized by high inter-personal loyalty and low loyalty towards organizations. Research on energy in Asia focuses on (1) Western and Chinese HR management practices and their relevance, (2) loyalty building and employee retention in Asia, (3) how to obtain, retain, motivate and develop employees in Asia. 

\null
\textbf{Research Highlights 2006}

During the past years, energy in and of organizations, energetic institutions, and energizing has been an emerging topic which received large attention among academics and practitioners from different fields.  Energy has been called ``the fuel that makes great organizations run'' (Dutton 2003, p. 7). Organizational energy can be seen as a force an organization purposefully works with and manifests in the intensity, pace, and endurance of a collective's work, change, and innovation processes (Bruch \& Ghoshal 2003). However, companies differ greatly in their ability to generate and maintain organizational energy. Thus, research on energy and its practical implications promises to improve the knowledge of core topics on the future management agenda such as growth, innovation, or change.

The current insights about organizational energy in academia and practice remain limited, although research on energy and energy-related aspects is growing, e.g. positive organizational scholarship, positive organizational behavior, or individual level research on vigor, engagement, or thriving. Cross, Baker \& Parker (2003) point out that ``while the term energy is pervasive in much of organizational life, it is also a highly elusive concept in that context'' (p. 51). Authors also state that ``[energy] is a construct that organizational scholars use but seldom define'' (Quinn \& Dutton, 2005, p. 36).  Despite its relevance to organizations, there are still many open questions regarding the construct of energy, its antecedents, its effects, and the implications for leadership in organizations. 

\newpage
Heike Bruch, Sven Voelpel and Bernd Vogel had been serving as conveners of the track ``Organizational Energy - Energizing Leadership'' at the 6th Annual European Academy of Management Conference, 4-7 May 2006, Norwegian School of Management Oslo, Norway. This was the first track where the new construct organizational energy was introduced. 

\paragraph{Collaborations}
\begin{itemize}
\item University of St. Gallen \\ Prof. Dr Heike Bruch
\item University of Stellenbosch \\ Prof. Dr. Marius Leibold
\end{itemize}

\paragraph{Grants}
\begin{itemize}
\item DAAD Scholarship (Eden Tekie).
\item DFG Travel Grant for Eden Tekie for Academy of Management Conference 2006.
\end{itemize}
