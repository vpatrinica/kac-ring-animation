\index{Name, First Name}

\paragraph{Research Team}
Sven Voelpel (Professor), Zheng Han (Postdoctoral Fellow), Chris Streb (Doctoral Fellow), Jan Meyer (Doctoral Fellow), Chunli Zhao (Doctoral Fellow).

 Innovation is currently regarded as a key factor for organizational success. This is supported by research indicating that innovative products create 80\% of companies' future revenues. Organizations aiming to survive and succeed in the highly competitive global innovation economy urgently need to adapt and innovate (reinvent) their organizational structure, business model and culture. They have to manage and enhance their innovation capabilities to create and sustain a wise organization. Research into innovation entails projects on: organizational adaptability and (disruptive) innovation, business model reinvention.

 High innovativeness has become the most important competitive edge of technology intensive firms, especially in developed economies. However, many developed economies, such as Germany, are facing a challenging phenomenon, i.e. a rapidly aging workforce. The critical question that arises is to what extent an aging of its workforce affects the innovativeness of a company and therefore their competitive advantage? Organizations that aim to sustain the pace of continuous innovations in a technology-intensive era urgently need to understand what threats and challenges are brought to them by significant demographic changes.

 Research in the area of innovation and aging focuses especially on: Aging workforce impact on innovation, enhancing innovation with an aging workforce, managing an aging workforce to sustain continuous innovations, age-heterogeneous work groups for enhanced innovation. 

 Until now, multinational companies (MNCs) in emerging markets of Asia, such as China, have mainly been concentrating on production-oriented investments. Due to an ongoing extension of competition, low cost production alone will likely fail to help firms to stay competitive in the long run. Differentiation through innovation will become a new and strategic move for many MNCs' operation in Asia. To succeed in this endeavor they need to realistically evaluate the major benefits and challenges and understand how innovations can be generated and R\&D activities can be managed under entirely different circumstances. This entails that local R\&D and innovation activities are also integrated into the MNC's global network.

 Research consequently focuses on benefits and challenges of building R\&D activities in emerging markets such as China and India, management of intellectual property, cooperation management, integration of local R\&D activity into the global R\&D network.

\null
\textbf{Research Highlights 2006}

 Further developing and deepening the innovation stream of research a number of articles have been published last year in collaboration with our collaborators, the prominent researchers in the field such as Professors Leibold and Davenport. The articles aim to bridge the innovation topic with important organizational issues such as leadership and strategic management.

 In a pursuit to tackle an impact of the aging workforce on the innovation process, our research group in collaboration with a new research partner Professor Van Der Vegt has designed a research project proposal ``The Effects of the Aging Workforce on the Innovation Process: A Large-Scale Study of Technology Intensive Companies''. Although research into the aging workforce is beginning to gain momentum, a conceptual framework that explains impacts of the aging workforce on the innovation process is nonexistent. This research product is expected to make an important contribution to academic research by explicitly addressing this issue. In this project, our research group aims to combine insights from organizational demography, innovation and organizational psychology research, to develop and test hypotheses that explain differences in innovative work behaviors across age cohorts at different activity levels. Knowledge of the differences in idea generation patterns, innovation promotion and implementation work behaviors across age cohorts will enable organizations to leverage the unique expertise and skills of their age diverse workforce. The project proposal is under review at one of Germany's major foundations. 

 The project proposal has been presented by Sven Voelpel and Polina Isichenko at the 66th Annual Academy of Management Meeting, August 2006, Atlanta, USA on ``Managing the Aging Workforce - Leadership towards a new Weltanschauung'' workshop organized by the track convenors Sven Voelpel and Chris Streb. The participants of the workshop and scholars that provided feedback included new academic collaborators of our research group were Professors David DeLong and Barbara Lawrence. 

 As a first stage of the project, interview guidelines and a survey were worked out, linking age with types of innovation, innovative activities, innovative factors and motivation. According to the preliminary agreement with Meyer Werft, the first application of this survey is currently in the pilot phase and will be tested at the company in the beginning of 2007.

 A new six months research and consulting project on innovation has been set up by Zheng Han and Sven Voelpel with five companies, i.e. Astrium Space Transportation, KAEFER Isoliertechnik, Meyer Werft, OHB Systems and Rheinmetall Defence Electronics. Aim of the project headed by Zheng Han will be to release innovation potentials of organizations. By working together with these innovation-oriented companies, the project will tackle questions such as, How to systematically identify new areas and ideas of innovation? Which kind of organizational design supports innovation? How to effectively invest and distribute corporate resources for innovation? How to create innovation through cooperation with customers, suppliers and other third parties? The kick-off event of this project was in December 2006. 

\newpage
\paragraph{Collaborations}
\begin{itemize}
\item  Harvard University \\ Technology and Operations Management \\ Prof. Alan MacCormack, PhD
\item  Harvard Business School, USA \\ Dorothy Leonard
\item  Tsinghua University, China \\ Prof. Dr. Max von Zedtwitz
\item  University of St. Gallen, Switzerland \\ Prof. Dr. Oliver Gassmann; Prof. Dr. Georg von Krogh
\item  Hitotsbashi University, Japan \\ Prof. Ikujiro Nonaka
\item  Babson College \\ Prof. Dr. Thomas H. Davenport
\item  University of Stellenbosch, South Africa \\ Prof. Dr. Marius Leibold
\item  University of Groningen, Netherlands \\ Prof. Dr. Gerben Van der Vegt
\item  AgeLab Massachusetts Institute of Technology (MIT), USA \\ Prof. David Delong PhD
\item  University of California, Los Angeles (UCLA), USA \\ Prof. Dr. Barbara S. Lawrence
\end{itemize}

\paragraph{Grants}
\begin{itemize}
\item Stiftung der Deutschen Wirtschaft (sdw) Scholarship (Chris Streb). 
\item Daimler Chrysler (Funding for PhD Thesis of Chris Streb).
\item DFG Travel Grant for Chris Streb for Academy of Management Conference.
\item VHB Travel Grant for Zheng Han for the R\&D Management Conference 2006.
\end{itemize}