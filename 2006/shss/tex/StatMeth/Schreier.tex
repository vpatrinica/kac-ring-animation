\subsection{Prof. Dr. Margrit Schreier}


\textbf{Main Research Interests}\\[-0.25cm]
\begin{enumerate}
\item[$\bullet$]	Qualitative Research Methods
\item[$\bullet$]	Media Psychology
\item[$\bullet$]	Gender and Reading
\item[$\bullet$]	Persuasive Effects of Fiction
\end{enumerate}

\vspace{0.6cm}
\textbf{Research Activities}\\[-0.25cm]

In 2006, Margrit Schreier has in the first place continued her work on the persuasive effects of fiction, with a special emphasis on degree of involvement as a moderator variable. A second focus has been on gender differences in the experience of reading fiction (funding for this research is provided by the DFG). The results of a first study are in line with previous research to the extent that women prefer to read in order to be drawn into a different world and in order to escape from everyday problems, whereas men are more likely to read for information. Even stronger differences in reading preferences and styles are obtained, however, for the comparison not of men and women, but of gender-schematic and a-schematic persons. A-schematic persons combine the reading interests usually ascribed to either men or women: they are more interested in reading in general, especially in reading nonfiction, and at the same time they read primarily in order to identify with the characters in the fictional world. Another study testing the effects of genre and substantive focus of a fictional narrative on degree of involvement, with both self-report and physiological measures of involvement taken into account, is in the process of being analyzed. A third focus has been on qualitative research methods and methodology, especially on content analysis and strategies of purposive sampling, resulting in a number of publications.


\vspace{0.6cm}
\textbf{Funded Projects}\\[-0.25cm]

"Gender and reading: the effect of text and reader characteristics on the experience of reading fiction" (Lesen und Geschlecht: Differentielles Lese-Erleben in Abh�ngigkeit von Text- und Personenmerkmalen), funded by the Deutsche Forschungsgemeinschaft.


\vspace{0.6cm}
\textbf{Other Professional Activities}\\[-0.25cm]
\begin{enumerate}
\item[$\bullet$]	Since January 2005: Editor-in-chief of the "Zeitschrift f�r Medienpsychologie"
\item[$\bullet$]	Membership in the Editorial Review Boards of the following journals: \textit{Forum Qualitative Social Research, Human Communication Research, Media Psychology, Poetics}
\item[$\bullet$]	Reviewer for the above journals and for DFG grant proposals
\item[$\bullet$]	Workshops on "Qualitative Content Analysis" at ZUMA
\item[$\bullet$]	Teaching cooperation on qualitative methods with the GSSS (Graduate School of Social Sciences, Bremen University)
\item[$\bullet$]	Member of the Advisory Board of INQUA (Institut f�r Qualitative Forschung, Berlin)
\end{enumerate}


\vspace{0.6cm}
\textbf{PhD-Students}\\[-0.25cm]

�zen Oda$\breve{\textrm{g}}$\newline
\textit{Unterschiede und Gemeinsamkeiten zwischen Frauen und M�nnern beim Lesen von Erz�hltexten - eine empirische Studie}\\[-0.15cm]

Yvonne Thies-Brandner\newline
\textit{Safer Sex through Entertainment Education? Realizing the Entertainment Education Approach with Computer Games about HIV/AIDS}\\[-0.15cm]

Bob Tsang\newline
\textit{The Hong Kong Museum of History: its Role in the Construction of Chinese Identity in Pre- and Post-1997 Hong Kong}
