\subsection{Prof. Dr. Klaus Boehnke}

\textbf{Main Research Interests}\\[-0.25cm]
\begin{enumerate}
\item[$\bullet$]	Political Socialization
\item[$\bullet$]	Value Transmission
\item[$\bullet$]	Social Science Methodology
\item[$\bullet$]	Educational Sociology
\end{enumerate}

\vspace{0.5cm}
\textbf{Research Activities}\\[-0.25cm]

As in prior years, research of Klaus Boehnke focused primarily on processes of political socialization among youth and young adults, often employing a cross-cultural perspective. Two major research projects have been conducted. The first project studied effects of the EU Eastern enlargement on the mobilization of right-wing extremism in Germany and bordering regions in Poland and the Czech Republic, with funding from the Deutsche Forschungsgemeinschaft. This study encompassed representative samples of the border regions of the three countries and is a longitudinal continuation of a study previously funded by the Federal Ministry of Education and Science (BMBF). The second project was a long-term panel study of young peace-movement sympathizers from the mid 1980s who have been studied every 3 � years since 1985. In 2006 the seven panel wave was conducted. That study focuses on the impact of societal context on individual psychosocial well-being and political activism. In the report phase, work has also continued on two other research projects, one on value transmission in families, another one on problems of social exclusion as a consequence of high achievement in schools ("Streber" study). In addition Klaus Boehnke has prepared two research proposals, one to the BMBF together with Professor Ariel Knafo from the Hebrew University of Jerusalem on identity development and value transmission among migrant youth, and the other to the Deutsche Forschungsgemeinschaft together with Professor Petra Lietz and Dr. Nongkran Wongsri from Saint Louis College in Thailand on learning styles of native and sojourner students in Germany and Thailand. He also prepared a proposal to the Deutsche Forschungsgemeinschaft together with Professor Ulrich K�hnen to obtain funding for the 19$^{\textrm{th}}$ International Congress of Cross-Cultural Psychology, which he hosts together with Ulrich K�hnen at IUB in 2008.

\vspace{0.5cm}
\textbf{Funded Projects}\\[-0.25cm]
\begin{enumerate}
\item[$\bullet$]	"Social integration after the fourth round of EU enlargement-border 	regions as test cases for Europe's path to a transnational civil society",	funded by the Deutsche Forschungsgemeinschaft
\end{enumerate}

\vspace{0.5cm}
\textbf{Organization of Scientific Conferences}\\[-0.25cm]
\begin{enumerate}
\item[$\bullet$]	June 2006\\
Multiliteracy and the European Educational Agenda, 1st international LANGSCAPE conference in conjunction with 3rd Bremen Conference on Content and Language Integrated Learning\\
(organized together with Professor Gerhard Bach from Universit�t Bremen)\\
funded by the Deutsche Forschungsgemeinschaft\\
international participants: 45\\[-0.15cm]

\item[$\bullet$]	July 2006\\
19th Biennial Meeting of the International Society for Behavioral Development\\
Melbourne, Australia\\
Member of the Review Panel\\
international participants: approx. 600\\[-0.15cm]

\item[$\bullet$]	July 2006\\
18th International Congress of Cross-Cultural Psychology\\
Spetses, Greece\\
Member of the International Scientific Program Committee\\
international participants: approx. 600\\[-0.15cm]

\item[$\bullet$]	July 2006\\
28th International Congress of Applied Psychology\\
Athens, Greece\\
Program Chair for the Political Psychology Program\\
international participants: approx. 1.800\\[-0.15cm]

\item[$\bullet$]	September 2006\\
45. Kongress der Deutschen Gesellschaft f�r Psychologie\\
N�rnberg\\
Senior Reviewer for Political Psychology
\end{enumerate}

\vspace{0.5cm}
\textbf{Other Professional Activities}\\[-0.25cm]
\begin{enumerate}
\item[$\bullet$]	College Master of Mercator College at the International University Bremen
\item[$\bullet$]	Secretary-General of the International Association for Cross-Cultural Psychology
\item[$\bullet$]	President of the Division of Political Psychology of the International Association for Applied Psychology
\item[$\bullet$]	President of the German Peace Psychology Association (\textit{Forum Friedenspsychologie})
\item[$\bullet$]	Member of the Executive Board of the Youth Sociology Section of the German Sociological Association (DGS)
\item[$\bullet$]	Ad hoc Expert Evaluator for the Deutsche Forschungsgemeinschaft
\item[$\bullet$]	Expert Evaluator for the European Commission in the 6th Framework Program, NEST, and the Marie Curie Program
\item[$\bullet$]	Expert Evaluator for INTAS
\item[$\bullet$]	Expert Evaluator for the Austrian National Science Foundation
\item[$\bullet$]	Expert Evaluator for the Israel Science Foundation
\item[$\bullet$]	Expert Evaluator for the Economic and Social Science Research Council, United Kingdom
\item[$\bullet$]	Consulting Editor of Peace and Conflict: The Journal of Peace Psychology
\item[$\bullet$]	Consulting Editor of the Journal of Cross-Cultural Psychology
\item[$\bullet$]	Member of the Editorial Board of the International Journal of Behavioral Development
\item[$\bullet$]	Member of the Editorial Board of \textit{Conflict \& Communication online}
\item[$\bullet$]	Senior Advisor of the Committee for Psychological Study of Peace (CPSP) of the International Union of Psychological Science (lUPsyS)
\item[$\bullet$]	Official Liaison (Vertrauensdozent) of Heinrich-B�ll-Stiftung at IUB
\item[$\bullet$]	Ad hoc reviewer for \textit{European Journal of Personality, European Journal for Psychology of Education, European Journal of Social Psychology, European Psychologist, European Societies, International Journal of Behavioral Development, International Journal of Psychology, Journal of Behavioral Science, Journal of Cross-Cultural Psychology, Journal of Public Health, Journal of Social and Clinical Psychology, Peace and Conflict: The Journal of Peace Psychology, Political Psychology, Politische Vierteljahresschrift, Social Psychology of Education: An International Journal, Zeitschrift f�r Medienpsychologie}
\item[$\bullet$]	IUB-Coordinator of the application for a Bremen International Graduate School of Social Sciences ($^{\rm\bf BI}${\itshape GSSS}) in the framework of the Excellence Initiative
\item[$\bullet$]	IUB-Fachkoordinator for the evaluation of sociology in Germany, conducted by the Wissenschaftsrat
\item[$\bullet$]	Member of the Promotion Committee of the Jacobs Center for Lifelong Learning and Institutional Development
\end{enumerate}

\vspace{0.5cm}
\textbf{PhD-Students}\\[-0.25cm]

Daniel Fuss\newline
\textit{Europa als Quelle sozialer Identit\"at--Eine international vergleichende\newline Analyse ihrer Voraussetzungen und Wirkungen bei jungen Erwachsenen}\newline
Defense: December 2006 \\[-0.15cm]

Simone Heil\newline
Working title: \textit{German-Israeli Youth Exchange Programs}\\[-0.15cm]

Angela Kindervater\newline
\textit{Stereotype versus Vorurteile: Welche Rolle spielt Autoritarismus? Ein empirischer Beitrag zur Begriffsbestimmung}\newline
Defense: September 2006\\[-0.15cm]

Olena Kornyeyeva\newline
Working title: \textit{Immigrant Adaptations and Modes of Upbringing}\\[-0.15cm]

Gukaah Brenda Nwana\newline
Working title: \textit{African Immigrants in Germany}\\[-0.15cm]

Martina Schreiber\newline
Working title: \textit{Football Hooliganism in a Social Psychological Perspective}


\vspace{0.6cm}
\textbf{Research Personnel}\\[-0.25cm]

Daniel Fuss\newline
Research Associate\newline
funded by Deutsche Forschungsgemeinschaft

\vspace{0.6cm}
\textbf{Guests}\\[-0.25cm]

Dr. Nongkran Wongsri\newline
Dean of the Faculty of Liberal Arts, Saint Louis College Bangkok, funded by the Deutsche Forschungsgemeinschaft (3 weeks)
