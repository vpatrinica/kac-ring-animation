\subsection{Prof. Dr. Welf Werner}

\vspace{0.3cm}
\textbf{Main Research Interests}\\[-0.25cm]
\begin{enumerate}
\item[$\bullet$]	International Monetary Regime Change
\item[$\bullet$]	International Trade in Services
\item[$\bullet$]	Trade Policy on Financial Services
\item[$\bullet$]	Regulation and Development of International Insurance Markets
\item[$\bullet$]	European Employment and Social Policies
\item[$\bullet$]	U.S. Economic Policy and Transatlantic Economic Relations
\item[$\bullet$]	History of Globalization
\item[$\bullet$]	U.S. Economic History
\end{enumerate}


\vspace{0.6cm}
\textbf{Research Activities}\\[-0.25cm]

Before joining IUB in 2004, Welf Werner worked in the fields of international economics, economic history, and U.S. economic policy. At IUB he has focused on Asia and particularly China, has engaged in a research project on European labor and social policies, and has broadened his interests in trade in services. Regarding international trade in services, Werner is interested in the compilation of trade statistics, the applicability of trade theory and in the ways and means of integrating services in regional and multilateral trade agreements. Topics covered in publications include welfare effects of services liberalization for developing countries, the status of liberalization commitments in the ongoing Doha negotiations at the World Trade Organization (co-authored with Walter Werner, the head of the German delegation in the Doha negotiations) and the conflicts created by trying to open financial markets while protecting the safety and soundness of national and international financial markets. With his work on European social and labor market policies, Werner crosses traditional academic boundaries. The VolkswagenStiftung is financing his research project "Globalization: Challenges for EU Social and Labor Market Policies" as part of its program "Welfare State Transformation: Bridging the Gap between Theory and Practice". While working at the Federal Ministry of Economics and Labor in Berlin for eight months Welf Werner advised on the European Employment Strategy (EES). After having focused on American and Atlantic policy issues at the Freie Universit�t Berlin, Harvard University and the School of Advanced International Studies of the Johns Hopkins University, Werner's shift of attention to Asia resulted in invitations to give presentations in Beijing and Seoul, supervision of a Chinese PhD student, and the publication of a paper on the rapid internationalization of Chinese financial markets in China's leading journal on international economics. Earlier interests, which Werner followed up with new papers, include the development of transatlantic merchandise trade and the dynamics of international reinsurance markets. 


\vspace{0.6cm}
\textbf{Funded Projects}\\[-0.25cm]
\begin{enumerate}
\item[$\bullet$]	"Globalization: Challenges for EU Social and Labor Market Policies," funded by VolkswagenStiftung in its program "Future Issues of our Society - Analysis, Advice and Communication between Academia and Practice. Welfare State Transformation: Bridging the Gap between Theory and Practice."
\end{enumerate}


\vspace{0.6cm}
\textbf{Organization of Scientific Conferences}\\[-0.25cm]
\begin{enumerate}
\item[$\bullet$]	March 2006\newline
Bonn\newline
"Jahrestagung of the Wirtschaftshistorischer Ausschuss, Verein f�r Socialpolitik"\newline
Member of the Scientific Board
\end{enumerate}


\vspace{0.6cm}
\textbf{Other Professional Activities}\\[-0.25cm]
\begin{enumerate}
\item[$\bullet$] Regular contributions of book reviews to \textit{Historische Zeitschrift, Vierteljahrschrift f�r Sozial- und Wirtschaftsgeschichte (VSWG)} and \textit{Harvard Business History Review}
\item[$\bullet$]	Referee for Studienstiftung des deutschen Volkes.
\item[$\bullet$]	Founding member of the Lions Club International University Bremen and the Center for International Studies, IUB.
\item[$\bullet$]	Member of Deutscher Hochschulverband, Wirtschaftshistorischer Ausschuss of the Verein f�r Socialpolitik, American Political Science Association, Verein f�r Unternehmensgeschichte, Verein f�r die gesamte Versicherungswissenschaft.
\item[$\bullet$]	Participating Researcher: Bremen International Graduate School of Social Sciences. Proposal for a Graduate School, Excellence Initiative by the German Federal and State Governments.
\item[$\bullet$]	Participating Researcher: Visual Hegemonies: Modeling and Decoding of Key Visuals in Intercultural Comparisons, IUB.
\item[$\bullet$]	Coordinator, Integrated Social Science Program, School of Humanities and Social Sciences, IUB. Together with Margrit Schreier.
\item[$\bullet$]	Consultant to the Federal Ministry of Economics and Labor, Berlin.
\end{enumerate}


\vspace{0.6cm}
\textbf{PhD-Students}\\[-0.25cm]

Okta Nofri\newline
\textit{Regional Integration: ASEAN and EU}
