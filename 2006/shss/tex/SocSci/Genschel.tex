\subsection{Prof. Dr. Philipp Genschel}


\textbf{Main Research Interests}\\[-0.25cm]
\begin{enumerate}
\item[$\bullet$]	International Political Economy
\item[$\bullet$]	Transformations of the State in Advanced Capitalist Societies
\item[$\bullet$]	Tax Policy
\item[$\bullet$]	Theories of Institutions and Institutional Change
\end{enumerate}


\vspace{0.6cm}
\textbf{Research Activities}\\[-0.25cm]

Philipp Genschel has been involved in two main research activities in 2006. First, working on a joint project on international tax policy together with PhD students Thomas Rixen, Ingo Rohlfing and Susanne Uhl, he analysed the interaction between international governance institutions and state sovereignty in the field of taxation. Focusing on the global double tax treaty regime and the EU tax policy regime, he argues that the extensive guarantees these regimes provide for legal sovereignty in taxation paradoxically contribute to undermining national policy autonomy in this field. Research results will appear in a series of forthcoming papers in 2007. Second, in his capacity as deputy director of the Bremen Collective Research Center on the "Transformation of the State" (Sfb 597), he co-authored the Center's research program for 2007-2010. This research program was a key element of the Center's successful re-application for funding by the Deutsche Forschungsgemeinschaft in the fall of 2006.


\vspace{0.6cm}
\textbf{Funded Projects}\\[-0.25cm]
\begin{enumerate}
\item[$\bullet$]	"Der Steuerstaat und die internationale Steuerpolitik"\newline
	funded by the Deutsche Forschungsgemeinschaft as part of the 	Collaborative Research Center 597 "Transformations of the State"
\end{enumerate}

\vspace{0.6cm}
\textbf{Other Professional Activities}\\[-0.25cm]
\begin{enumerate}
\item[$\bullet$]	Deputy Director of the DFG Sonderforschungsbereich 597 "Staatlichkeit im Wandel" (Collaborative Research Center 597 "Transformations of the State")
\item[$\bullet$]	Co-Director MA International Relations (joint program of IUB and Uni Bremen)
\item[$\bullet$]	Director of IUB's Center for International Studies (CIS)
\item[$\bullet$]	Reviewer for several journals (including \textit{European Union Politics, European Journal for Political Research, European Journal of International Relations, Journal of Common Market Studies, Journal of European Public Policy}).
\item[$\bullet$]	Reviewer for the Deutsche Forschungsgemeinschaft
\end{enumerate}


\vspace{0.6cm}
\textbf{PhD-Students}\\[-0.25cm]

Junghoon Choi\newline
\textit{International Cooperation against Money Laundering and Tax Evasion: Similar Problems, Different Outcomes?}\\[-0.15cm]

Thomas Rixen\newline
\textit{The Political Economy of International Tax Cooperation: Institutional Choice and Development in International Taxation}\\[-0.15cm]

Ingo Rohlfing\newline
\textit{Trading Interests: Bilateralism and Multilateralism in International Trade Cooperation, 1860-2005}\\[-0.15cm]

Susanne Uhl\newline
\textit{Die Transformation von Steuerstaatlichkeit in Europa}


\vspace{0.6cm}
\textbf{Research Personnel}\\[-0.25cm]

Thomas Rixen, Ingo Rohlfing and Susanne Uhl work as research associates in the project "Der Steuerstaat und die internationale Steuerpolitik"
