\subsection{Prof. Dr. Marion G. M�ller}

\vspace{0.3cm}
\textbf{Main Research Interests}\\[-0.25cm]
\begin{enumerate}
\item[$\bullet$]	Visual Communication
\item[$\bullet$]	The Use of Visuals in Times of War and Terrorism
\item[$\bullet$]	Visuals and Emotions
\item[$\bullet$]	Political Iconology
\item[$\bullet$]	Visual Competence
\item[$\bullet$]	Comparative Political Communication
\item[$\bullet$]	Symbolic and Ritual Communication
\item[$\bullet$]	Electoral Campaigns (US, Germany, EU)
\item[$\bullet$]	Comparative Parliamentary Studies
\end{enumerate}


\vspace{0.6cm}
\textbf{Research Activities}\\[-0.24cm]

Marion G. M�ller's major research activities in 2006 centered on further developing her visual archive: Politisch-ikonographisches Archiv der Vision (PIAV) with press-clippings and an image catalogue in both an electronic and in a printed postcard format. The archive is at the center of her current research interest in "Pathosformeln in den Massenmedien". Methodologically she applies the interpretative approach of political iconology in the tradition of the Kulturwissenschaftliche Bibliothek Warburg to the social sciences in an effort to make visual material accessible to the problem-oriented scrutiny of mass communication and political science. Her research questions center on the cohesion of democratic societies and the role of visuals in the process of creating and maintaining democratic belief systems. In this context she has devoted attention to the in-depth analysis of the cartoon conflict over Muhammad caricatures published by a Danish newspaper in late 2005 and eliciting violent responses in the Muslim world at the beginning of 2006. A second research focus, connected with the previously mentioned, is the use of images in times of war and terrorism, analyzing terrorist images published in the mass media and providing analyses of the ethical dilemma with which many journalists and editors are confronted in their decisions to publish images produced by terrorists. A third research focus in 2006 has been the scrutiny of perception and reception processes of press photography. The collaborative experimental study with her colleagues Professor Bettina Olk and Professor Arvid Kappas provides for a first and singular methodological combination of eye tracking, visual content analysis (iconology) and psychophysiological measurement of emotional reactions to press photography. The visual stimuli used in the first pilot experiment were violent protests in Nepal in spring 2006. Thus, this collaborative research project is additionally connected with Marion G. M�ller's interest in violent reactions to visuals. The theoretical framework for the collaborative study was formulated in an article published (with Arvid Kappas) in the peer-reviewed communication journal \textit{Publizistik} in spring 2006.

\vspace{0.6cm}
\textbf{Organization of Scientific Conferences}\\[-0.25cm]
\begin{enumerate}
\item[$\bullet$]	February 2006\newline
Hochschule f�r Philosophie, M�nchen\newline
"Bildethik" - Joint Annual Conference of the Visual Communication Division and the Division for Media and Communication Ethics of the DGPuK\newline
funded by Deutsche Gesellschaft f�r Publizistik und Kommunikationswissenschaft (DGPuK) and Hochschule f�r Philosophie M�nchen\newline
US-American keynote speaker
\end{enumerate}


\newpage
\textbf{Other Professional Activities}\\[-0.25cm]
\begin{enumerate}
\item[$\bullet$]	Chair of the Visual Communication Division of the German Communication Association (DGPuK)
\item[$\bullet$]	Vice Chair of the Visual Studies Division of the International Communication Association (ICA)
\item[$\bullet$]	Program Chair for organizing the Visual Studies Division's program for the Annual ICA Conference in San Francisco 2007
\item[$\bullet$]	Reviewer for the German Communication Association (DGPuK)
\item[$\bullet$]	Reviewer for the International Communication Association (ICA)
\item[$\bullet$]	Member of the adivsory board of the journal \textit{Politik \& Kommunikation}
\item[$\bullet$]	Member of the scientific advisory board of the Institut f�r Medienpolitik, Berlin
\item[$\bullet$]	Member of the DGPuK-"Selbstverst�ndnisausschuss"
\item[$\bullet$]	External member of the search committee for two W3 professorships "Kunstgeschichte" and "Kunst-Bildung-Vermittlung" at Universit�t Oldenburg
\item[$\bullet$]	Alumna 11$^{\textrm{th}}$ Transatlantic Forum (TAF), May 2006, BMW Stiftung Herbert Quandt
\end{enumerate}


\vspace{0.6cm}
\textbf{PhD-Students}\\[-0.25cm]

Ayse Esra �zcan\newline
\textit{Female Representation in Turkish Print Media}\\[-0.15cm]

Pinar Yildiz\newline
\textit{Honor Crimes and their Reception in the News}
