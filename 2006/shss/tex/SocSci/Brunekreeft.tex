\subsection[Prof. Dr. Gert Brunekreeft] {Prof. Dr. Gert Brunekreeft{\normalfont\normalsize\newline(Joined IUB in September 2006)}}

\vspace{0.3cm}
\textbf{Main Research Interests}\\[-0.25cm]
\begin{enumerate}
\item[$\bullet$]	Competition Policy
\item[$\bullet$]	Economics of Regulation
\item[$\bullet$]	Industrial Organization
\item[$\bullet$]	Network Industries
\item[$\bullet$]	Electricity Markets
\end{enumerate}


\vspace{0.6cm}
\textbf{Research Activities}\\[-0.25cm]

In 2006 Gert Brunekreeft has been setting up a 2-year joint research project (jointly with a number of European universities where IUB takes the lead) on the pro's and con's of ownership unbundling of energy companies. This issue has high practical relevance and has only been studied poorly in the literature. Secondly he has engaged in research on the economic relation between different types of economic regulation of monopolies (like energy or railway networks) and investment in the infrastructure. This is a front issue in the academic debate, with significant economic relevance. Furthermore he has conducted research on the economics of regulation, eyeing in particular the design of incentive regulation for the German energy sector, as this should be implemented by January 1, 2008. As director of the Bremer Energie Institut, which is associated to IUB, Gert Brunekreeft--jointly with the colleagues at the institute--does research in energy-related environmental economics and policy. Hereby, the cost effective supply of combined heat and power and the energy efficiency both take pivotal roles. There are always a number of projects in process in the institute, funded by public bodies and industry alike.


\vspace{0.6cm}
\textbf{Funded Projects}\\[-0.25cm]

As Director of the Bremer Energie Institut (since September 1, 2006) Gert Brunekreeft was nominal principal investigator of the following funded research projects:
\begin{enumerate}
\item[$\bullet$]	"Investitionsanreize und die Randnummern 18 und 218 im Bericht der BNetzA nach �112a EnWG zur Einf�hrung der Anreizregulierung nach �21a EnWG."
\item[$\bullet$]	"Unbundling of Energy Companies: Is it worth it?"
\item[$\bullet$]	"Optimal networks"
\item[$\bullet$]	"Ermittlung von Effekten des KfW-CO2-Geb�udesanierungsprogramms"\newline
funded by KfW Bankengruppe, Frankfurt
\end{enumerate}


\vspace{0.6cm}
\textbf{Organization of Scientific Conferences}\\[-0.25cm]
\begin{enumerate}
\item[$\bullet$]	October 2006\newline
	Berlin\newline
	"5$^{\textrm{th}}$ Conference on Applied Infrastructure Research (INFRADAY)"\newline
	funded by: a number of sponsors, among them the European 	Commission.\newline
	international participants: 200  
\end{enumerate}


\vspace{0.6cm}
\textbf{Other Professional Activities}\\[-0.25cm]
\begin{enumerate}
\item[$\bullet$]	Senior Economist at EnBW AG Karlsruhe
\item[$\bullet$]	Director of the Bremer Energie Institut
\item[$\bullet$]	Co-editor of the journal \textit{Competition and Regulation in Network Industries} (CRNI)
\item[$\bullet$]	Research Fellow at a number of research institutions
\end{enumerate}


\vspace{0.6cm}
\textbf{Research Personnel}\\[-0.25cm]

As Director of the Bremer Energie Institut (since September 1, 2006) Gert Brunekreeft was nominal supervisor of the research associates at the Bremer Energie Institut (all of which are funded through diverse third-party grants).



