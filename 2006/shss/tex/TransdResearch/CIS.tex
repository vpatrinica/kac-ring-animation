\subsection{Center for International Studies (CIS)}


\paragraph{Speaker/Coordinator:} Philipp Genschel


\vspace{0.5cm}
The Center was established in 2004 as an institutional framework for research in international affairs. The aim was to bridge disciplinary boundaries within and between the humanities and the social sciences. The Center is open to all relevant theoretical traditions and methodological approaches as long as the research focuses on issues with a transborder aspect be it international relations, cross-national comparison, or domestic politics and culture with cross-boundary effects. Considerable research activities have been going on in these fields from the very beginning of the SHSS. The Center provides an institutional platform to better integrate, synthesize and develop these activities. 


\vspace{0.5cm}
To meet its general goal of encouraging, promoting, and coordinating international studies at IUB, the Center pursues two particular tasks: 
First, to serve as a forum of debate and intellectual exchange among scholars interested in International Studies. In this role CIS helped to facilitate the following conferences at IUB:
\begin{enumerate}
\item[$\bullet$] "Intercultural Understanding and Competence" organized by Ulrich K\"uhnen and funded by IUB.
\item[$\bullet$]	"Multiliteracy and the European Educational Agenda". 1st international LANGSCAPE conference in conjunction with 3rd Bremen Conference on Content and Language Integrated Learning organized by Gerhard Bach from Universit\"at Bremen and Klaus Boehnke and funded by the Deutsche Forschungsgemeinschaft
\item[$\bullet$]	"Art and Metaphysics in the Twentieth Century and Beyond" organized by Isabel W\"unsche, Paul Crowther and Ursula Frohne, funded by the Deutsche Forschungsgemeinschaft.
\item[$\bullet$]	"Ferne \& N\"ahe: Der Beitrag der russischen Kunst zur europ\"aischen Moderne" organized by Isabel W\"unsche and Ada Raev (Berlin), funded by the Karin und Uwe Hollweg Foundation, Bremen
\end{enumerate}
The second task is to provide an institutional platform to coordinate ongoing research on international issues and integrate future research plans. In this function, the Center helped to coordinate and prepare the following successful grant applications: 

\begin{enumerate}
\item[$\bullet$]	Matthijs Bogaards together with Matthias Basedau (Hamburg), Peter Niesen (Darmstadt), and Christoph Hartmann (Bochum): "Managing Ethnic Conflict through Institutional Engineering: Ethnic Party Bans in Africa" funded by the Thyssen Foundation.
\item[$\bullet$]	Klaus Boehnke: "Social integration after the fourth round of EU enlargement-border regions as test cases for Europe's path to a transnational civil society" funded by the Deutsche Forschungsgemeinschaft.
\item[$\bullet$]	Philipp Genschel: "International tax policy and the national �tax state" funded by the Deutsche Forschungsgemeinschaft as part of the Collaborative Research Center 597 "Transformations of the State" second phase (2007-2010).
\item[$\bullet$]	Hartmut Wessler: "The Transnationalization of Public Spheres: The Case of the EU" funded by the Deutsche Forschungsgemeinschaft as part of the Collaborative Research Center 597 "Transformations of the State" second phase (2007-2010).
\item[$\bullet$]	Christian Welzel: "Gr\"unde, Ursachen und Triebkr\"afte des postindustriellen Wertewandels: Deutschland im globalen Vergleich", funded by the Deutsche Forschungsgemeinschaft.
\end{enumerate}

Finally, CIS also facilitated IUB's participation in the joint application with Universit\"at Bremen in the framework of the Excellence Initiative of the Deutsche Forschungsgemeinschaft and the Wissenschaftsrat. Klaus Boehnke coordinates the Center's activities in this initiative. 


\vspace{0.5cm}
Members of CIS in 2006:
Profs. K. Boehnke, M. Bogaards, P. Genschel (Director), U. K\"uhnen, P. Lietz, J. Paulmann, N. Spakowski, C. Welzel, W. Werner, H. Wessler, and I. W\"unsche.
