\subsection{"Transformations of the State"\\
The Collaborative Research Center (CRC) 597 \newline
Deutsche Forschungsgemeinschaft}


\paragraph{Speakers/Coordinators:} Philipp Genschel and Hartmut Wessler


\vspace{0.6cm}
The Collaborative Research Center 597 studies the "Transformations of the State" in the OECD-world since the 1970s. The influence of the state on human lives is more comprehensive and sustained than that of any other organizational construct. It regulates the economy, fights crime, provides education, manages traffic, sustains democracy, enters wars, guarantees social welfare, collects taxes and deploys some forty percent of the gross domestic product. The Research Center conceptualizes the multi-faceted modern state in four intersecting dimensions: resources, or control of the use of force and revenues; law, or jurisdiction and the courts; legitimacy, or the acceptance of political rule by the populace; and welfare, or the facilitation of economic efficiency and social equity.


\vspace{0.6cm}
The remarkable feature of the 20$^{\textrm{th}}$ century nation state was to almost completely monopolize activities in these four dimensions and merge them in one tightly woven fabric that appeared so well-suited to the specific conditions of the post-WWII era that people nowadays often look back on that period as the golden age of the state. But what has become of the Golden-Age state since the 1970s? Is its fabric worn out, is it unraveling? Will it be rewoven and restyled - perhaps as one gigantic world state of uniform pattern, or perhaps in the miniature, as a multitude of semi-sovereign regional governments? Or will the fibers simply separate, each following its individual fate in postmodern fashion, the rule of law moving into the international arena and the nation state clinging to its resources, while the intervention state comes completely unspun and goes every which way?


\vspace{0.6cm}
In its first phase (2003-2006), the Collaborative Research Center has focused on providing descriptive answers to these questions. Some preliminary results were published in 2005 (see Stephan Leibfried/Michael Z\"urn (eds.): Transformations of the State? Cambridge: Cambridge University Press). In 2006, the Center successfully applied for a second phase of funding (2007-2010). The major task in this phase will be to analyze the causal structure of the transformations of the state. What are the drivers? What are the critical junctures? Is the state just a victim of external forces of change or is it also an active agent of its own transformation? 


\vspace{0.6cm}
IUB Participants in the CRC:
Profs. P. Genschel (deputy director), M. Jachtenfuchs (until 2006), and H. Wessler (since 2005).
