\subsection{Cultures, Markets and Citizens (CMC)}


\paragraph{Speaker/Coordinator:} Brendan Dooley


\vspace{0.6cm}
The Cultures, Markets and Citizens Research Group (CMC) studies the ways in which individuals in modern societies are affected by crucial intersections between the field of culture and the structures of the marketplace. At present the research of the CMC group is focused on three projects - one rooted in a deep chronological perspective (Art and Value), another casting a broad gaze over contemporary Europe (Energy and Culture), and finally, yet another focused on the origin of mechanisms of exchange (Culture and Exchange).


\vspace{0.6cm}
Art and Value attempts to assess the ways in which a market has evolved for some of the most precious material values existing in our societies: namely, our artworks. It analyzes the effect of market structures on the valuation of artworks. At the same time, it examines how artists, in a period of rapid development of a market for their work, have themselves become commentators and, in effect, evaluators, of the "market" existing in their societies. The project covers the period from the early nineteenth century, when market structures began to impinge particularly on the art world, up to the 1960s and beyond, when art becomes a billion dollar business. Work on this project will reach public diffusion via a major international exhibition to be held in Bremen (or elsewhere), and a fully illustrated catalogue collecting contributions from the organizers, their collaborators, and other scholars who have worked on these problems.


\vspace{0.6cm}
The project "Energy and Culture" analyzes the ways in which cultural factors affect public opinion in regard to energy issues. Opinion surveys such as the Eurobarometer and the World Value Survey have demonstrated significant differences across national boundaries concerning questions such as the importance of protecting the environment or the need to find new energy sources. No study has yet attempted to collect and compare information on the cultural and structural bases of these national differences. The core of the project is a series of country studies, selected for representativeness as well as for contrast. These will be Germany, Sweden, France, Switzerland, Italy, Spain, Bulgaria, Romania, and the United States. For each country in the study, we will be examining several basic variables that can yield a profile of the energy availability and use resulting from the specific social, cultural, and political factors present in that country, on which future expectations can be based. These variables include Efficiency, Usage per person, Structure of energy system, Policies for energy sector, Natural environment, Human resources, Marketing campaigns by energy firms, by government and by other sectors. Data sources will include official statistics, personal interviews (for instance, with IUB students), the World value survey, secondary data sources, special, surveys and experiments. Basic methods include qualitative analysis, scenario simulations, local statistical models, data mining techniques, and hierarchical models. Results will be divulged in an international conference and by a series of collective publications including individually and collectively authored articles in major journals. A major international conference for establishing partnerships in this research was held on 18-20 March 2004, and the proceedings of this conference were published by Ashgate Publishers (UK) this year (2006).


\vspace{0.6cm}
The project "Culture and Exchange" focuses a novel combination of humanities, social science and information science methods on unique materials in Bremen and elsewhere, to study a phenomenon of worldwide significance: namely, the historic transition to modern communication networks, which took place in the early modern period. Our purpose is to provide the first detailed description of the political information networks that began to join the cities of Germany to other parts of Europe in this period. Project funding will be used to apply modern information technology facilitating a systematic analysis of the unique microfilm collection of early German newspapers in the State Library in Bremen, in comparison with document collections existing elsewhere, including the Medici Archive in Florence Italy, which contain evidence for how news stories were born, how they grew and how they came to be widely distributed. Major partners in this initiative are the Presseforschung unit at Uni Bremen, the University of Venice, the University of Louvain. A conference is planned for 15-16 December, 2006, entitled "Time and Space on the Road to Modernity: The Emergence of Contemporaneity in Early Modern Culture." The purpose of the conference is to study one of the major features of the shared consciousness of European reality in modern times: namely, the emergence of Contemporaneity, made possible by the widespread creation of public and private networks for the transmission of political news, both in manuscript and print. How were stories born, how did they grow and mature during their transmission from source to source, from country to country? This workshop will study the sources and methods that will make possible the first systematic international collaborative effort to approach the problem. 


\vspace{0.6cm}
Participants in CMC initiatives:\\
Within IUB: Profs. B. Dooley, P. Lietz, P. Crowther, I. W�nsche, C. Welzel, A. Wilhelm, H. Wessler (all SHSS), R. Richards, A. Boetius (SES). Outside collaborators include Prof. H. B�ning, Universit�t Bremen; Dr. A. Blome, Universit�t Bremen; Dr. J. Weber, Universit�t Bremen; Prof. Dr. Th. W. Gaehtgens, Freie Universit�t Berlin and Centre allemand d'histoire de l'art/Deutsches Forum fuer Kunstgeschichte, Paris; Dr. S. Flach, Literaturwissenschaftliche Zentren Berlin; Prof. M. Bal, Director of the Amsterdam School for Critical Analysis at the University of Amsterdam; Prof. R. Koolhaas, Harvard University, Cambridge, Mass. and Office for Metropolitan Architecture OMA, Rotterdam; Prof. S. Ciriacono, University of Padua; Prof. J. Beckert, Max-Planck-Institut f�r Gesellschaftsforschung, K�ln; Dr. O. Velthuis, Erasmus University Rotterdam; Prof. J. D�ez Medrano, University of Barcelona, Prof. M. Infelise, University of Venice; Dr. P. Arblaster, University of Louvain; Dr. J. Bos, Royal Library of The Hague; Dr. P. Ries, Oxford, M. Mendle, University of Alabama; C. Beltr�n (History, Burgos, Spain); J.-P. Vittu , Literature, Orl�ans, M. Infelise, Venice; A. Hardie, Lancaster; M. van Otegem, The Hague; S. Schulthei�-Heinz, Universit�t Bayreuth; C.-G. Holmberg; N. Brownlees, Florence; Z. Barbarics, University of P�cs; F. Mauelshagen, Zurich.
