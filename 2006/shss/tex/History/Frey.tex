\subsection[Prof. Dr. Marc Frey] {Prof. Dr. Marc Frey {\normalfont\newline Helmut Schmidt Chair of International History} {\normalfont\normalsize\newline (Joined IUB in August 2006)} }

\textbf{Main Research Interests}\\[-0.25cm]
\begin{enumerate}
\item[$\bullet$]	International Relations History $19^{\rm th}$ and $20^{\rm th}$ Century
\item[$\bullet$]	Colonialism and Decolonization
\item[$\bullet$]	History of Development
\item[$\bullet$]	History of Southeast Asia in the $20^{\rm th}$ Century
\end{enumerate}


\vspace{0.6cm}
\textbf{Research Activities}\\[-0.25cm]

Marc Frey's research activities have broadly centered on North-South relations in the $19^{\rm th}$ and $20^{\rm th}$ centuries and on aspects of international and transnational history of Southeast Asia in the $20^{\rm th}$ century. In the area of North-South relations, research has been focusing on four projects: (a) \textit{Colonial Law, Colonial Injustice}, which analyses forms of plural legalism in European and Japanese colonies in Africa and Asia in light of its implications for the development of international law and traces the connections between state law, public law and criminal law on the one hand and the changing character of international law in the period from c. 1870 to the onset of decolonization; (b) two projects are related to nation building, economic development and social change in the era of decolonization and early independence, one traces the role of elites in the decolonization process from a transdisciplinary perspective, the other analyzes development policies of Asian countries in the period 1945-1975; (c) \textit{Development Policies in Global Perspective}, is intended as a textbook, to be published with Beck Publishers Munich, on the history of development policies in contemporary history and traces the origins of development doctrines, theories and regimes, evaluates the results and shortcomings of development policies, and draws conclusions from past experiences with development policies for our time; (d) finally, together with two IUB colleagues, the project \textit{Asianisms in the $20^{th}$ Century} looks at processes of integration and differentiation, of condensation and demarcation within Asia from a cultural, societal, economic, and political perspective. 

\vspace{0.6cm}
\textbf{Organization of Scientific Conferences}\\[-0.25cm]
\begin{enumerate}
\item[$\bullet$]June 2006\newline
National University of Singapore\newline
"Asian Experiences of Development, 1945-1975"\newline
funded by the National University of Singapore, the Canada Research Council, the Deutsche Forschungsgemeinschaft, and Regional Office of Konrad Adenauer Foundation in Singapore\newline
international participants: 25
\item[$\bullet$]November 2006\newline
Finland-Institute Berlin\newline
"Modernization and Social Change in Finland and West-Germany in the Post-war Period"\newline
funded by Finland-Institute Berlin\newline
international participants: 15
\end{enumerate}



\vspace{0.6cm}
\textbf{Other Professional Activities}\\[-0.25cm]
\begin{enumerate}
\item[$\bullet$] Regular contributions of book reviews to International History Review, H-Net, Neue Politische Literatur
\item[$\bullet$] Member of Advisory Board, Graduiertenkolleg "Zivilgesellschaftliche Verst�ndigungsprozesse vom 19. Jahrhundert bis zur Gegenwart. Die Niederlande und Deutschland im Vergleich", Center for Dutch Studies, University of M�nster
\item[$\bullet$] Anonymous reviewer for Harvard University Press and Singapore University Press
\item[$\bullet$] Professional collaboration with National University of Singapore, University of Toronto, Ohio State University, Die ZEIT-Stiftung and with the Bucerius Law School, Hamburg
\end{enumerate}


\vspace{0.6cm}
\textbf{PhD-Students}\\[-0.25cm]

Christoph Meyer\newline
\textit{The League of Nations, the Debate on Economic and Social Development, and the Emergence of Global Governance}\\[-0.15cm]

\vspace{0.6cm}
\textbf{Research Personnel}\\[-0.25cm]

Stefanie J\"{u}rries, M.A.\newline
Research Associate for the project \textit{Asianisms in the ${20^{th}}$ Century}
