\subsubsection{Analytic Number Theory}
\index{Baier, Stephan}

\paragraph{Research Team}
Stephan Baier (Visiting Lecturer)

\medskip

My field of research is analytic number theory. So far my research has been
centered around the following main topics: detecting primes in sparse sets,
applications and improvements of the large sieve, and applications of methods
of analytic number theory to elliptic curves. More in particular, I~am
interested in the distribution of primes in sequences of polynomial growth
(Piatetski-Shapiro primes, primes represented by polynomials), in improvements
of the large sieve inequality from analytic number theory for sparse sets of
moduli and in  the Lang-Trotter and Sato-Tate conjectures on elliptic curves.
Since taking up my position as a Visiting Lecturer at Jacobs University Bremen
in September~2006, I~have mainly worked together with Liangyi Zhao (KTH
Stockholm, Sweden) on primes in quadratic progressions.


\paragraph{Highlights}

It was due to Dirichlet that any linear polynomial represents infinitely many primes provided the coefficients are co-prime.  Though long been conjectured, analogous statements are not known for any polynomial of higher degree.  
G.H.~Hardy and J.E.~Littlewood conjectured an asymptotic formula for
the number of primes $\le N$ represented by any quadratic polynomial that may conceivably represent infinitely many primes. In a recent joint preprint with L.~Zhao~\cite{BAZO}, we looked at quadratic polynomials of the form $f(x)=x^2+k$, $k$ being a fixed natural number.
More precisely, we proved that, as $N\rightarrow\infty$, the asymptotic formula for the number of primes $\le N$ of the form $x^2+k$ predicted by Hardy and Littlewood holds, in a certain sense, for {\it almost all} $k\le N/(\log N)^A$, where $A$ is any given positive constant. In this preprint, we used Hardy-Littlewood's circle method (a trigonometrical method) together with some modern tools developed by H.~Mikawa in studying the twin primes conjecture on average.

During the last months, we have continued our collaboration on this problem. Using a completely different approach which has some similarities to Linnik's dispersion method, we have been able to prove the above almost-all result for the much shorter range $k\le N^{2/3+\varepsilon}$. A key ingredient in this work are zero density estimates for Dirichlet L-functions. Under the Riemann hypothesis for Dirichlet L-functions we have further reduced the $k$-range to $k\le N^{1/2+\varepsilon}$.
A preprint on these results is in preparation.  In the near future, we want to try to improve the above results by refining our techniques.  Moreover, we aim to extend our results to tuplets of quadratic progressions employing methods developed by A.~Balog.

Another of my research projects is to improve previously obtained results on the Lang-Trotter conjecture about elliptic curves. These are non-singular algebraic curves of genus ~1. They are equipped with a natural group structure which makes them an interesting object of research in number theory. Elliptic curves have applications in cryptography and fast factorization of integers.

The Lang-Trotter conjecture predicts an asymptotic formula for the number of primes $p\le x$ such that $\lambda_E(p)$ equals a fixed integer $r$, where  $\lambda_E(p)$ is the difference between $p+1$ and the number of points on the elliptic curve $E_p$ obtained from an elliptic curve $E$ over the rationals by reduction modulo $p$.

Although the Lang-Trotter conjecture seems to be out of reach of currently available methods, it was proved by C.~David and F.~Pappalardi that in a certain sense it holds true on average over families of elliptic curves. In a recent preprint~\cite{BAILAN} I improved the results of David-Pappalardi by expressing certain error terms as character sums and estimating them non-trivially. In future work, L.~Zhao and I aim to improve these estimates further.

Similar methods as in~\cite{BAILAN} were used in~\cite{BAZO2} to prove a
result on the Sato-Tate conjecture for small intervals (resp.\ angles) on average over families of elliptic curves. This conjecture predicts a certain limit distribution of $\lambda_E(p)/(2\sqrt{p})$, as $p$ varies over the primes. In a recent preprint, R.~Taylor has announced a proof of the Sato-Tate conjecture for all elliptic curves over totally real fields satisfying some mild condition.

\paragraph{Collaborations}
\begin{enumerate}
\item {\sl Jacobs University Bremen} \\
  Prof.~Dr.~M.~Stoll \\  
  Rational Points on Curves
\item {\sl KTH Stockholm, Sweden} \\ 
  Dr.~L.~Zhao \\ 
  Primes in Quadratic Progressions, Sato-Tate and Lang-Trotter Conjectures 
  on Elliptic Curves
\item {\sl Harish-Chandra Research Institute in Allahabad, India} \\
  Prof.~Dr.~S.~Das Adhikari, Dr.~P.~Rath \\  
  Extremal Problems in Lattice Point Combinatorics
\end{enumerate}

