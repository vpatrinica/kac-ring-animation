%\documentclass[10pt,a4paper,twocolumn]{article}
%
%\usepackage{times}
%\usepackage{graphicx}
%
%\newcommand{\mycaption}{\caption}
%
%\begin{document}
%
%\setcounter{section}{4}

\section[Mathematics and Theoretical Physics]{Mathematics and \\ Theoretical Physics}

Research in Mathematics and Theoretical Physics is fundamental
research. Pure Mathematics explores the world of mathematical
objects primarily for its own sake, thus preparing the ground for
sometimes surprising applications. For example, Number Theory, some
of whose proponents were even proud of the complete lack of
applications not so long ago, is nowadays heavily used in the design
of (and attacks on) cryptosystems. Applied Mathematics draws its
motivation more directly from applications; this ranges from
theoretical investigations of the underlying mathematical structures
over the modeling of specific systems to the simulation of these
systems and the visualization of the results. Theoretical Physics
expresses the fundamental laws of nature in a mathematical language
that is suitable for the prediction and quantitative understanding
of new phenomena. Methods of Theoretical and Computational Physics
are applicable to a wide range of topics, often going well beyond
the realm for which they were originally developed.

Despite the small size of the university, a fairly broad spectrum of
research topics within these areas is covered at Jacobs University Bremen. Research in
Pure Mathematics mainly focuses on Geometry and Dynamics, Algebra,
and Number Theory. The more theoretical side of Applied Mathematics
is represented by Harmonic Analysis and Approximation Theory. Other
groups do modeling or study the Numerical Analysis necessary to
obtain reliable simulations, and there is a group working on
visualization. Research in Theoretical Physics at
Jacobs University ranges from Mathematical Physics over
Statistical and Computational Physics to Astrophysics and Biophysics. Its
transdisciplinary nature is also reflected in this research report by contributions
to almost all the other major topics.





% \end{document}
