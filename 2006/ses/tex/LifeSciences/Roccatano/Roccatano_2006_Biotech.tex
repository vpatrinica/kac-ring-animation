\subsubsection{Computational Bioengineering}
\index{Roccatano, Danilo}

\paragraph{Research Team}
Danilo Roccatano (University Lecturer (Chemistry))\\

The investigation of physico-chemical properties of
biological  systems of industrial relevance using computational methods
is one of the major aspects of  my research. I am interested in
applying theoretical model and using molecular modelling to provide useful information
that drives enzyme evolution and protein engineering for biotechnological applications. Various aspects of my research activities are summarized below:\\

Study of the effects of non-aqueous solvents and their water mixtures on the
conformation, diffusive dynamics and reactivity of macromolecular systems for
bioengineering. In particular, the solvent effect on
protein folding and enzymatic activity is of high commercial and academic interest.\\

In collaboration with Prof. U. Schwaneberg and collaborators, we are committed to
developing computational tools to assist directed evolution experiments.~\cite{Wong06M,Schenk06M,Wong06bM}
As a long term project, we are developing a Computational Bioengineering Portal (map.iu-bremen.de)
comprising various programs that assist random mutagenesis experiments. This publicly available server is dedicated to directed evolution community.



\paragraph{Highlights}
Main research achievements in year 2006 are summarized as follow:\\


The  collaboration with Prof. U. Schwaneberg and his PhD student T.
S. Wong resulted in  important achievements:  MD simulations study
provided a clue on the mechanism of DMSO inactivation of heme
monooxygenase P450 BM-3~\cite{Roccatano06aM}.  A subsequent
resolution of the crystal structure of P450 BM-3 heme domain in
presence of DMSO, performed by T. S. Wong in collaboration with
Prof. M. Wilmanns at EMBL Hamburg, confirmed the theoretical
hypothesis and showed the
coordination of DMSO to the active site heme iron. \\

Setting up a Computational Bioengineering Portal at Jacobs University (map.iu-bremen.de) (Fig. \ref{fig:Roccatano2}).
The portal is dedicated to providing various computational
support for directed evolution. It includes Mutagenesis Assistant Program (MAP)~\cite{Wong06M,Schenk06M}
(running since beginning 2006, Fig. \ref{fig:Roccatano2}) that allows analyzing amino acid substitution patterns of 19 random mutagenesis methods upon input of a DNA sequence. Other two programs will be
available on the portal by the end of 2006, SeSaM-Tv and ExPoSeS. SeSaM-Tv is computational tool
to support Sequence Saturation Mutagenesis (SeSaM), a novel random mutagenesis method developed by T. S. Wong and Prof. U. Schwaneberg. ExPoSeS (Exploring Protein Sequence Space) provides guidance to efficiently explore protein sequence space using combinations of random mutagenesis methods.


\begin{figure}[ht]
  \begin{center}
   \includegraphics[width=\hsize]{Roccatano/DrRoccatano-fig2.pdf}
     \mycaption{The Computational Bioengineering Portal at Jacobs University (map.iu-
bremen.de).}\label{fig:Roccatano2}
   \end{center}
\end{figure}



Danilo Roccatano is also involved in ``Nano and Material Science''.


\myparagraph{Collaborations}
%
Bremen Area Collaborations:
\begin{enumerate}
\item {\sl International University Bremen} \\ Prof. W. Nau \\ Study of the dynamics in solution of small peptides by Molecular Dynamics simulation and time-resolved spectroscopic tecniques.
 \\ Prof. U. Schwaneberg \\ Study of  the organic solvent effects on monooxygenase P450 BM-3 and Mutagenesis Assistant Program.
 \\ Prof. M. Zacharias \\ Molecular Dynamics simulation study of the nucleosome core particle.
\end{enumerate}
National \& International Collaborations:
\begin{enumerate}
\item {\sl University of Salerno, Italy} \\ Prof. G. Milano \\ Interaction of polymer with biological membranes
\item {\sl University of East Anglia, Norwich, UK} \\ Dr. S. Hayward  \\ Theoretical investigation of domain motion in proteins
\item {\sl Max Planck Institute for Biophysical Chemistry, G�ttingen } \\Dr. B. de Groot \\ Theoretical investigation of domain motion in proteins
\end{enumerate}


\nocite{Pal06M,Roccatano06bM,Sahoo06M}
