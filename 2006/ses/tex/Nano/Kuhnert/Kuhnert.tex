
\subsubsection{Supramolecular and Analytical Chemistry}
\index{Kuhnert, Nikolai}

\paragraph{Research Team}
Dr. Nikolai Kuhnert (Professor), Fatemeh Jami (PhD Student), Warren Drynan (PhD Student), Birgul Surucu (PhD Student)\\

The Kuhnert group is interested in two main areas: The chemistry
of biologically active compounds from food and the supramolecular
chemistry of novel macrocyclic systems. Food can be exploited as a
major source for biologically active compounds for medicinal
purposes. We are working on the structural elucidation, synthesis
and biological evaluation of polyphenolics a major class of
compounds from plant based food.  Polyphenolics are produced by
plants as complex mixtures of isomeric compounds whose separation,
isolation and structural elucidation pose a major scientific
challenge. In particular we focus on tea and coffee as sources of
complex polyphenolics. We develop new methods for their structural
elucidation mainly based on tandem mass spectrometry. Along with
structure elucidation we undertake synthetic chemistry to access
selected compounds and evaluate their biological activity with a
series of collaborators. In the field of supramolecular chemistry
we develop new synthetic methodology for the synthesis of a new
class of chiral macrocyclic compounds we have named trianglimines
due to their unique triangular molecular architecture. We exploit
the supramolecular chemistry of these systems in a series of
applications ranging from chiral recognition over sensing
applications to the development of nano-devices and machines.


\paragraph{Highlights}

\emph{Structural elucidation of chlorogenic acids}
Following our previous observation that isomeric natural products
such as chlorogenic acids (defined as esters of quinic acids and
its epimers) can be readily distinguished by tandem mass
spectrometry and we have continued our effort in this field. We
have developed a hierarchichal key for the interpretation of
tandem MS spectra of chlorogenic acids that allows easy and
unambiguous identification of many members of this ubiquitous
class of secondary natural products [2]. We have identified
further structural classes of chlorogenic acids in green coffee
beans [2, 4] and expanded our work to other food based plants such
as Hemerocallis and Aster [1, 6]. The total number of new
chlorogenic acids we have identified in plants has now stands at
over a hundred. The biological evaluation of selected members of
chlorogenic acids is currently under progress. We have continued
our efforts in the general field of stereochemical analysis
including analysis of enantiomers using tandem mass spectrometry
and a series of publications in this field are to be expected
shortly.

\emph{Macrocyclic and supramolecular chemistry}\\
Following our previous work on the synthesis of chiral
trianglamine and trianglimine macrocycles we have continued our
synthetic efforts. We have developed a method for the
functionalisation of trianglamine macrocycles furnishing highly
functionalized enantiomerically pure synthetic macrocycles by
acylation and alkylation reactions [5]. Furthermore we have
developed the synthesis of a first trianglamine based simple
molecular device a trianglamine-cyclodextrin [2] catenane [3].
This catenane synthesis is in particular remarkable due to the
fact that we can use either enantiomer of the trianglamine
macrocycle and hence obtain for the first time diastereomeric
catenanes in a controlled fashion.\\



    %Pictures are to be included via:

    %\begin{figure}[ht]
    %  \begin{center}
    %    \includegraphics[width=6cm]{profxxx-figx.jpg}
    %    \mycaption{ xxx )}\label{fig:profxxx}
    %\end{center}
    %\end{figure}

    %\paragraph{Organization}
    % list the (research) events you have organized, if any,

    %\begin{enumerate}
    %\item  xxx
    %\item  xxx
    %\end{enumerate}

\paragraph{Collaborations}
\begin{enumerate}
\item {\sl The University of Surrey} \\ Professor Mike Clifford \\
Food Safety, analysis of polyphenolics\\
\item {\sl The University of Surrey} \\ Professor Costas Ioannides \\
Toxicology and animal studies of dietary bioactive compounds\\
\item {\sl Universit\"{a}t Erlangen} \\ Professor Nicolai Burzlaff
\\ Einkristallr\"{o}ntgenstrukturanalyse\\
\item {\sl Veterinary Laboratories Agency Weybridge} \\Dr Maurice
Sauer and Dr Nick Coldham \\ Mass spectrometry analysis of animal
derived samples\\
\item {\sl Unilever drinks research, Colworth} \\ Dr. Jacek
Obuchowicz \\ Structural analysis of black tea polyphenolics
\end{enumerate}


%\paragraph{Grants}
% list the running grants in 2005, if none have been received, please delete this
% subsection.
%\begin{enumerate}
%\item {Funded by} ``Proposal ��
%\item {Funded by} ``Proposal ��
%\end{enumerate}


%\paragraph{Awards, Prizes}
% list the grants you have received in 2005, if none have been received, please delete this
% subsection.
%\begin{enumerate}
%\item
%\item
%\end{enumerate}

\nocite{Kuhnert1}
\nocite{Kuhnert2}
\nocite{Kuhnert3}
\nocite{Kuhnert4}
\nocite{Kuhnert5}
\nocite{Kuhnert6}
