
\subsubsection{Simulation and Control of Complex Dynamical Systems}\label{ict:modeling:antoulas}
\index{Antoulas, A.C.}

\paragraph{Research Team}
Athanosios Antoulas (Professor)

Model reduction seeks to replace a large-scale system of differential
or difference equations (the outcome of discretizing a PDE, perhaps) by a
system of substantially lower dimension, that ideally, has the same
response characteristics as the original system,
yet requires far less computational resources for realization than the
potentially unmanageable levels that may be required by the larger original
system.

Two main currents can be identified among methodologies for model reduction.
Balanced approximation methods are built upon a family of
ideas with very close connection to the singular value decomposition.
These methods preserve stability and allow for global error bounds but
often do not scale well in terms of
computational efficiency and stability when applied to large scale problems.
Moment matching methods are based principally on Pad\'e-like
approximations and for large-scale problems have lead naturally to the
use of Krylov and rational Krylov subspace methods. These methods generally
enjoy greater efficiency  and numerical stability.
A strong current trend aims at combining these two approaches by deriving
iterative methods which achieve approximate balanced reduction.

\paragraph{Highlights}

%%% give a short (500 words)description of the research highlights. 1 figure costs 100 words

% to include a figure, generate a file xxx.pdf and integrate the following lines
%% \begin{figure}[ht]
%%   \begin{center}
%%     \includegraphics[width=10cm]{xxx}
%%     \caption{The caption of the figure}
%%     \label{fig:xxx}
%%   \end{center}
%% \end{figure}
% to reference it use ``Figure.~\ref{fig:xxx}''; the numbers will be computed automatically.

Our research activities are concerned with the circle of ideas surrounding
model reduction. It provides efficient
and robust methods for producing reduced order models of
large state space systems.  These activities
have an impact both in system theory of complex systems
as well as in applied mathematics and in particular numerical methods for
large-scale problems.  Once the theory and computational
methods are developed, we expect that high quality software will result and
have applications in many areas of engineering.
This will enable at a later stage,
the design of real time controllers for complex systems.

{\it Broader impact resulting from the proposed activities}.
In today's technological world, physical processes are
described mainly by mathematical models, which are used to
simulate the behavior of the physical processes in question.
Sometimes, they are also used to modify or control their behavior.
In this framework, there is an ever increasing need for
improved accuracy which leads to models of high complexity.

The basic motivation for system approximation is the need in many
instances for a simplified model of a dynamical system, which
however captures the main features of the original complex model.
This need arises from limited computational, accuracy, and storage
capabilities. The simplified model is then used in place of the
original complex model, either for {\it simulation}, or {\it control}.

Important areas of application of model reduction are: VLSI
(Very Large Scale Integration) design, weather prediction,
air quality management, molecular dynamics simulations, simulation and
control of chemical (e.g. CVD - Chemical Vapor Deposition) reactors,
car windscreen quality management, simulation and control of MEMS (Micro
Electro Mechanical Systems) devices, e.g. micromirrors, to name but a few.
Thus the proposed activities have potential benefits for society at large.

%\paragraph{Organization}
% list the (research) events you have organized, if any,

%\begin{enumerate}
%\item  ....
%\item   ...
%\item  ...
%\end{enumerate}

%\paragraph{Collaborations}
%\begin{enumerate}
%\item ...
%\item ...
%\item ...
%\end{enumerate}

\paragraph{Grants at RICE University}
% list the running grants in 2005, if none have been received, please delete this
% subsection.
\begin{enumerate}
\item
Funded by US National Science Foundation, R38570-776000, NSF CCR-0306503, \emph {Model
  Reduction for Structured Dynamical Systems}, (August 2003 - July 2006)
% Amount: \$$436,607.00$ 

\item
Funded by US National Science Foundation,  ITR (Information Technology Research) Grant,
collaborative with Purdue and Florida State, R38670-776000, NSF ACI-0325081, \emph
{Research on Model Reduction of Dynamical Systems for Real-time Control}, (January 2003 -
August 2007)
      %Amount: \$$1,826,959.00$\\

\item
Funded by US National Science Foundation, OSR No.: 06052204, NSF CCF-0634902,
\emph{Advanced Projection Techniques for Dimension Reduction of Large Scale Dynamical
  Systems}, (October 2006 - September 2009)

\end{enumerate}

%\paragraph{Patents}
% list the grants you have received in 2005, if none have been received, plese delete this
% subsection.
%\begin{enumerate}
%\item
%\item
%\end{enumerate}


%\paragraph{Awards, Prices}
% list the grants you have received in 2005, if none have been received, plese delete this
% subsection.
%\begin{enumerate}
%\item
%\item
%\end{enumerate}

%\paragraph{Publications}
% list the publications of 2005 (also accepted and in press), if none have been received, plese delete this
% subsection. Enter the publications into the SES publications database at
% http://kwarc.eecs.iu-bremen.de/ses-pubs/index.php and only reference them here.
\nocite{aca1}

% \begin{description}
% \item[Book]
% A.C. Antoulas,
% ``Approximation of large-scale dynamical systems'',
% Book Series: {\it Advances in Design and Control} {\bf DC 06}, 479 pages,
% SIAM, Philadelpia (2005).
% \item[Journal Paper]
% D.C. Sorensen and A.C. Antoulas,
% {\it On model reduction of structured systems}, in
% "Dimension Reduction of Large-Scale Systems",
% Edited by P. Benner, V, Mehrmann and D.C. Sorensen,
% Springer Verlag, pages 125-138, (2005).
% \item A.C. Antoulas, {\it A new result on passivity preserving
% model reduction}, Systems and Control Letters,
% Volume 54, Issue 4, Pages 361-374, April (2005).
% \item[Journal Paper]
% S. Gugercin and A.C. Antoulas,
% {\it Model reduction of large-scale systems by least squares},
% Linear Algebra and Applications,
% Special Issue on Order Reduction of Large-Scale Systems,
% Edited by P. Benner, D.C. Sorensen, R. Freund, and A. Varga, accepted
% for publication, 2005.
% \item[Journal Paper]
% A.C. Antoulas, {\it Frequency domain representation and singular
% value decomposition}, UNESCO EOLSS (Encyclopedia for the Life Sciences),
% Contribution 6.43.13.4, 52 pages, in press, 2005.
% \item[Journal Paper]
% A.C. Antoulas, {\it An overview of model reduction methods for large-scale
% dynamical systems}, IFAC Annual Reviews in Control, Invited Paper,
% JARAP vol. {\bf 228} (2005).
% \item[Report]
% Q. Zhou, K. Mohanram, and A.C. Antoulas,
% {\it Structure Preserving Reduction of Frequency-dependent Interconnect},
% Technical Report, June (2005).
% \item[Journal Paper]
% A.C. Antoulas, E. Gildin, R.H. Bishop and D.C. Sorensen,
% {\it Model and controller reduction for seismically excited buildings},
% submitted to "Structural Control and Monitoring", November (2005).
% \item[Journal Paper]
% S. Gugercin, A.C. Antoulas, and C. Beattie, Technical Report,
% {\it An iterative rational Krylov approach to optimal H2 model reduction},
% SIMAX (SIAM J. Matrix Analysis and Applications), 2006 (submitted).
% \item[Journal Paper]
% A.J. Mayo and A.C. Antoulas,
% {\it A framework for the general realization problem},
% {Linear Algebra and Its Applications}, Special Issue in honor of P.A. Fuhrmann,
% Edited by A.C. Antoulas, U. Helmke, J. Rosenthal, V. Vinnikov, and E. Zerz.
% 2006 (sumbitted).
% \item[Journal Paper]
% S. Gugercin, A.C. Antoulas, and C.A. Beattie,
% {\it Krylov-based controller reduction for large-scale systems},
% {Automatica}, 2006 (submitted).
%\end{description}
%%\end{document}
%%% Local Variables:
%%% mode: latex
%%% TeX-master: "report"
%%% End:
