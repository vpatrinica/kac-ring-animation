\shorttitle{Smart Systems}
\subsection{Smart Systems}
As humans we can sense, act, speak, listen, decide and sometimes
understand.  The 21st century will witness technologies that can do
the same. 
Robots are for example more and more used in domains where some autonomy and 
intelligence is necessary. They work under conditions where the robot is 
not constantly supervised by a human operator and where it has to be 
adaptive as its developer can not fully predict which situations it will 
encounter in its application environment. 
But also information services are rapidly changing from 
statically served data to information that is dynamically tailored to each individual user's
current needs. This information is fetched instantaneously from networked, distributed sources which
themselves change and evolve continually. At the same time, new representation formats
allow to discover and specify the internal and functional structure of information and
drive services that previously required (human) understanding. Technically, we observe the
convergence of databases, Internet, and distributed systems into semantically enriched
knowledge systems which allow the casual as well as the professional user to deal with the
ever-growing amount of information available.  
Another issue is the presentation of
information to the user.  Data visualization is concerned with the management of large
data, the filtering and extraction of salient features, and their visual representation in
an expressive, intuitive, and interactive manner.

%%% Local Variables:
%%% mode: latex
%%% TeX-master: "report"
%%% End:
